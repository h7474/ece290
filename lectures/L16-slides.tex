
\documentclass[letterpaper,hide notes,xcolor={table,svgnames},pdftex,10pt]{beamer}
\def\showexamples{t}


%\usepackage[svgnames]{xcolor}

%% Demo talk
%\documentclass[letterpaper,notes=show]{beamer}

\usecolortheme{crane}
\setbeamertemplate{navigation symbols}{}

\usetheme{MyPittsburgh}
%\usetheme{Frankfurt}

%\usepackage{tipa}

\usepackage{hyperref}
\usepackage{graphicx,xspace}
\usepackage[normalem]{ulem}
\usepackage{multicol}

\newcommand\SF[1]{$\bigstar$\footnote{SF: #1}}

\usepackage[default]{sourcesanspro}
\usepackage[T1]{fontenc}

\newcounter{tmpnumSlide}
\newcounter{tmpnumNote}

% old question code
%\newcommand\question[1]{{$\bigstar$ \small \onlySlide{2}{#1}}}
% \newcommand\nquestion[1]{\ifdefined \presentationonly \textcircled{?} \fi \note{\par{\Large \textbf{?}} #1}}
% \newcommand\nanswer[1]{\note{\par{\Large \textbf{A}} #1}}


 \newcommand\mnote[1]{%
   \addtocounter{tmpnumSlide}{1}
   \ifdefined\showcues {~\tiny\fbox{\arabic{tmpnumSlide}}}\fi
   \note{\setlength{\parskip}{1ex}\addtocounter{tmpnumNote}{1}\textbf{\Large \arabic{tmpnumNote}:} {#1\par}}}

\newcommand\mmnote[1]{\note{\setlength{\parskip}{1ex}#1\par}}

%\newcommand\mnote[2][]{\ifdefined\handoutwithnotes {~\tiny\fbox{#1}}\fi
% \note{\setlength{\parskip}{1ex}\textbf{\Large #1:} #2\par}}

%\newcommand\mnote[2][]{{\tiny\fbox{#1}} \note{\setlength{\parskip}{1ex}\textbf{\Large #1:} #2\par}}

\newcommand\mquestion[2]{{~\color{red}\fbox{?}}\note{\setlength{\parskip}{1ex}\par{\Large \textbf{?}} #1} \note{\setlength{\parskip}{1ex}\par{\Large \textbf{A}} #2\par}\ifdefined \presentationonly \pause \fi}

\newcommand\blackboard[1]{%
\ifdefined   \showblackboard
  {#1}
  \else {\begin{center} \fbox{\colorbox{blue!30}{%
         \begin{minipage}{.95\linewidth}%
           \hspace{\stretch{1}} Some space intentionally left blank; done at the blackboard.%
         \end{minipage}}}\end{center}}%
         \fi%
}



%\newcommand\q{\tikz \node[thick,color=black,shape=circle]{?};}
%\newcommand\q{\ifdefined \presentationonly \textcircled{?} \fi}

\usepackage{listings}
\lstset{%
  keywordstyle=\bfseries,
  aboveskip=15pt,
  belowskip=15pt,
  captionpos=b,
  identifierstyle=\ttfamily,
  escapeinside={(*@}{@*)},
  stringstyle=\ttfamiliy,
  frame=lines,
  numbers=left, basicstyle=\scriptsize, numberstyle=\tiny, stepnumber=0, numbersep=2pt}

\usepackage{siunitx}
\newcommand\sius[1]{\num[group-separator = {,}]{#1}\si{\micro\second}}
\newcommand\sims[1]{\num[group-separator = {,}]{#1}\si{\milli\second}}
\newcommand\sins[1]{\num[group-separator = {,}]{#1}\si{\nano\second}}
\sisetup{group-separator = {,}, group-digits = true}

%% -------------------- tikz --------------------
\usepackage{tikz}
\usetikzlibrary{positioning}
\usetikzlibrary{arrows,backgrounds,automata,decorations.shapes,decorations.pathmorphing,decorations.markings,decorations.text}

\tikzstyle{place}=[circle,draw=blue!50,fill=blue!20,thick, inner sep=0pt,minimum size=6mm]
\tikzstyle{transition}=[rectangle,draw=black!50,fill=black!20,thick, inner sep=0pt,minimum size=4mm]

\tikzstyle{block}=[rectangle,draw=black, thick, inner sep=5pt]
\tikzstyle{bullet}=[circle,draw=black, fill=black, thin, inner sep=2pt]

\tikzstyle{pre}=[<-,shorten <=1pt,>=stealth',semithick]
\tikzstyle{post}=[->,shorten >=1pt,>=stealth',semithick]
\tikzstyle{bi}=[<->,shorten >=1pt,shorten <=1pt, >=stealth',semithick]

\tikzstyle{mut}=[-,>=stealth',semithick]

\tikzstyle{treereset}=[dashed,->, shorten >=1pt,>=stealth',thin]

\usepackage{ifmtarg}
\usepackage{xifthen}
\makeatletter
% new counter to now which frame it is within the sequence
\newcounter{multiframecounter}
% initialize buffer for previously used frame title
\gdef\lastframetitle{\textit{undefined}}
% new environment for a multi-frame
\newenvironment{multiframe}[1][]{%
\ifthenelse{\isempty{#1}}{%
% if no frame title was set via optional parameter,
% only increase sequence counter by 1
\addtocounter{multiframecounter}{1}%
}{%
% new frame title has been provided, thus
% reset sequence counter to 1 and buffer frame title for later use
\setcounter{multiframecounter}{1}%
\gdef\lastframetitle{#1}%
}%
% start conventional frame environment and
% automatically set frame title followed by sequence counter
\begin{frame}%
\frametitle{\lastframetitle~{\normalfont(\arabic{multiframecounter})}}%
}{%
\end{frame}%
}
\makeatother

\makeatletter
\newdimen\tu@tmpa%
\newdimen\ydiffl%
\newdimen\xdiffl%
\newcommand\ydiff[2]{%
    \coordinate (tmpnamea) at (#1);%
    \coordinate (tmpnameb) at (#2);%
    \pgfextracty{\tu@tmpa}{\pgfpointanchor{tmpnamea}{center}}%
    \pgfextracty{\ydiffl}{\pgfpointanchor{tmpnameb}{center}}%
    \advance\ydiffl by -\tu@tmpa%
}
\newcommand\xdiff[2]{%
    \coordinate (tmpnamea) at (#1);%
    \coordinate (tmpnameb) at (#2);%
    \pgfextractx{\tu@tmpa}{\pgfpointanchor{tmpnamea}{center}}%
    \pgfextractx{\xdiffl}{\pgfpointanchor{tmpnameb}{center}}%
    \advance\xdiffl by -\tu@tmpa%
}
\makeatother
\newcommand{\copyrightbox}[3][r]{%
\begin{tikzpicture}%
\node[inner sep=0pt,minimum size=2em](ciimage){#2};
\usefont{OT1}{phv}{n}{n}\fontsize{4}{4}\selectfont
\ydiff{ciimage.south}{ciimage.north}
\xdiff{ciimage.west}{ciimage.east}
\ifthenelse{\equal{#1}{r}}{%
\node[inner sep=0pt,right=1ex of ciimage.south east,anchor=north west,rotate=90]%
{\raggedleft\color{black!50}\parbox{\the\ydiffl}{\raggedright{}#3}};%
}{%
\ifthenelse{\equal{#1}{l}}{%
\node[inner sep=0pt,right=1ex of ciimage.south west,anchor=south west,rotate=90]%
{\raggedleft\color{black!50}\parbox{\the\ydiffl}{\raggedright{}#3}};%
}{%
\node[inner sep=0pt,below=1ex of ciimage.south west,anchor=north west]%
{\raggedleft\color{black!50}\parbox{\the\xdiffl}{\raggedright{}#3}};%
}
}
\end{tikzpicture}
}


%% --------------------

%\usepackage[excludeor]{everyhook}
%\PushPreHook{par}{\setbox0=\lastbox\llap{MUH}}\box0}

%\vspace*{\stretch{1}

%\setbox0=\lastbox \llap{\textbullet\enskip}\box0}

\setlength{\parskip}{\fill}

\newcommand\noskips{\setlength{\parskip}{1ex}}
\newcommand\doskips{\setlength{\parskip}{\fill}}

\newcommand\xx{\par\vspace*{\stretch{1}}\par}
\newcommand\xxs{\par\vspace*{2ex}\par}
\newcommand\tuple[1]{\langle #1 \rangle}
\newcommand\code[1]{{\sf \footnotesize #1}}
\newcommand\ex[1]{\uline{Example:} \ifdefined \presentationonly \pause \fi
  \ifdefined\showexamples#1\xspace\else{\uline{\hspace*{2cm}}}\fi}

\newcommand\ceil[1]{\lceil #1 \rceil}


\AtBeginSection[]
{
   \begin{frame}
       \frametitle{Outline}
       \tableofcontents[currentsection]
   \end{frame}
}



\pgfdeclarelayer{edgelayer}
\pgfdeclarelayer{nodelayer}
\pgfsetlayers{edgelayer,nodelayer,main}

\tikzstyle{none}=[inner sep=0pt]
\tikzstyle{rn}=[circle,fill=Red,draw=Black,line width=0.8 pt]
\tikzstyle{gn}=[circle,fill=Lime,draw=Black,line width=0.8 pt]
\tikzstyle{yn}=[circle,fill=Yellow,draw=Black,line width=0.8 pt]
\tikzstyle{empty}=[circle,fill=White,draw=Black]
\tikzstyle{bw} = [rectangle, draw, fill=blue!20, 
    text width=4em, text centered, rounded corners, minimum height=2em]
    
    \newcommand{\CcNote}[1]{% longname
	This work is licensed under the \textit{Creative Commons #1 3.0 License}.%
}
\newcommand{\CcImageBy}[1]{%
	\includegraphics[scale=#1]{creative_commons/cc_by_30.pdf}%
}
\newcommand{\CcImageSa}[1]{%
	\includegraphics[scale=#1]{creative_commons/cc_sa_30.pdf}%
}
\newcommand{\CcImageNc}[1]{%
	\includegraphics[scale=#1]{creative_commons/cc_nc_30.pdf}%
}
\newcommand{\CcGroupBySa}[2]{% zoom, gap
	\CcImageBy{#1}\hspace*{#2}\CcImageNc{#1}\hspace*{#2}\CcImageSa{#1}%
}
\newcommand{\CcLongnameByNcSa}{Attribution-NonCommercial-ShareAlike}

\newenvironment{changemargin}[1]{% 
  \begin{list}{}{% 
    \setlength{\topsep}{0pt}% 
    \setlength{\leftmargin}{#1}% 
    \setlength{\rightmargin}{1em}
    \setlength{\listparindent}{\parindent}% 
    \setlength{\itemindent}{\parindent}% 
    \setlength{\parsep}{\parskip}% 
  }% 
  \item[]}{\end{list}} 




\title{Lecture 16 --- Tort: Negligent Misstatement }

\author{Jeff Zarnett \\ \small \texttt{jzarnett@uwaterloo.ca}}
\institute{Department of Electrical and Computer Engineering \\
  University of Waterloo}
\date{\today}


\begin{document}

\begin{frame}
  \titlepage

\begin{center}
  \small{Acknowledgments: Douglas Harder~\cite{dwh}, Julie Vale~\cite{jv}}
  \end{center}
\end{frame}




\begin{frame}
\frametitle{``Whoops.''}

What happens if an engineer says something that is incorrect and someone else acts on that information?

In a contract, one might be liable for damages if another party is affected.

A deliberate intent to deceive might be fraud.

Suppose there is no contract and it is not fraud...

Does a professional have a responsibility to others to be accurate?

\end{frame}



\begin{frame}
\frametitle{Negligent Misstatement/Misrepresentation}

A person owes a general duty of honesty, and a deliberate misstatement or misrepresentation is, of course, fraud.

The duty did not extend to being careful.

In 1951, the English Court of Appeal heard a case~\cite{lba}.

An accountant who carelessly prepared a misleading financial document for the company and this document led someone to invest in the company.

The court held that the accountant was not liable to compensate the investor. 

Is this decision reasonable?


\end{frame}



\begin{frame}
\frametitle{Negligent Misstatement/Misrepresentation}

The court was concerned that a liability for careless words would have a chilling effect on everyday business~\cite{lba}.

If you overheard a stockbroker talking about a company at lunch, and you acted on that advice, would you be able to sue the stockbroker for ``bad advice''?

The dissenting view in the case: the duty should be confined to situations where it it foreseeable that the other party would rely on the statement.

This dissenting view came to the fore later in an important decision.

\end{frame}


\begin{frame}
\frametitle{\textit{Hedley Byrne \& Co Ltd. v. Heller \& Partners}}

The key decision here is \textit{Hedley Byrne \& Co Ltd. v. Heller \& Partners} in 1964 and it is applicable to all professionals.

Hedley Byrne was an advertising firm and a client, Easipower Ltd., placed a large order.

Hedley Byrne requested a check of Easipower's financial situation and creditworthiness

Without consideration, Heller replied, with a letter indicating Easipower was\\
\quad``considered good for its ordinary business engagements.''

Heller's letter included the statement:\\
\quad ``without responsibility on the part of this bank''

\end{frame}



\begin{frame}
\frametitle{\textit{Hedley Byrne \& Co Ltd. v. Heller \& Partners}}

Soon thereafter, Easipower liquidated (went out of business and their assets were sold); Hedley Byrne lost \textsterling17~000.

This is the equivalent of half a million Canadian dollars (in 2015)!

Hedley Byrne sued for negligent and misleading information.

Heller \& Partners claimed: \\
\quad (1) there was no duty of care and\\
\quad (2) their disclaimer should protect them from liability.

Who was right?

\end{frame}



\begin{frame}
\frametitle{When all is won and lost...}

For the most part, the defendant won.

The court held that the disclaimer was adequate to limit liability.

But the court also found that their first claim was not true -- that professionals would be liable for statements that they make.

It was reasonable for the defendant to foresee that the plaintiff would rely on the advice given, and that establishes a duty of care.

\end{frame}



\begin{frame}
\frametitle{\textit{Hedley Byrne \& Co Ltd. v. Heller \& Partners}}

Lord Morris wrote in the ruling:

\begin{quote}
I consider that it follows and that it should now be regarded as settled that if someone possessing special skill undertakes, quite irrespective of contract, to apply that skill for the assistance of another person who relies upon such skill, a duty of care will arise. The fact that the service is to be given by means of or by the instrumentality of words can make no difference. Furthermore, if in a sphere in which a person is so placed that others could reasonably rely upon his judgment or his skill or upon his ability to make careful inquiry, a person takes it upon himself to give information or advice to, or allows his information or advice to be passed on to, another person who, as he knows or should know, will place reliance upon it, then a duty of care will arise.
\end{quote}


\end{frame}


\begin{frame}
\frametitle{The First Duty}

Where one person relies on the special skill and judgement of another, and the second person knew of the reliance, this establishes a duty.

The expert is expected to act with reasonable care in exercising that skill.

\end{frame}



\begin{frame}
\frametitle{Limitation Clauses}

The previous case turned on the limitation clause; the disclaimer.

\textit{Wolverine Tube (Canada) Inc. v Noranda Metal Industries Ltd. et al}, 1994.

An environmental consultant prepared a report with respect to environmental compliance audits and the possibility of environmental liabilities on the lands.

The report was prepared for the owner, Noranda Metal Industries.

\end{frame}



\begin{frame}
\frametitle{Limitation Clauses}

The report had the following disclaimer:

\begin{quote}
This report was prepared by Arthur D. Little of Canada, Limited for the account of Noranda, Inc. The material in it reflects Arthur D. Little's best judgment in light of the information available to it at the time of preparation. Any use which a third party makes of this report, or any reliance on or decisions to be made based on it, are the responsibility of such third parties. Arthur D. Little accepts no responsibility for damages, if any, suffered by any third party as a result of decisions made or actions based on this report.
\end{quote}


\end{frame}



\begin{frame}
\frametitle{\textit{Wolverine Tube Inc. v Noranda Metal Industries Ltd. et al}}

When selling the land to Wolverine Tube, Noranda passed on the report and indicated that Wolverine could rely on the report.

Five years after the report was prepared, Wolverine determined that the report contained significant errors.

The errors detrimentally affected Wolverine who then sued Noranda and the consultant
Wolverine claimed the consultant was negligent.

Was this disclaimer sufficient?

\end{frame}



\begin{frame}
\frametitle{\textit{Wolverine Tube Inc. v Noranda Metal Industries Ltd. et al}}

Yes; the court held that this disclaimer was adequate.

The court found the language more comprehensive than that in the \textit{Hedley Byrne} case and affirmed that decision.

1979 case: \textit{Trident Construction Ltd. v. W.L. Wardrop and Assoc. et al.}~\cite{lpe}.

An engineer designed a sewage plant but it had serious problems and it caused problems for the contractor. Was the engineer liable?

\end{frame}



\begin{frame}
\frametitle{Further Cases on Misstatements}

Yes, the engineer was liable. His special skill (design) was was applied here, without a suitable disclaimer of liability.

Thus, a duty of care attached and the engineer was responsible.

\end{frame}



\begin{frame}
\frametitle{Mistakes in Tendering}

\textit{Brown \& Hudson Ltd. v The Corporation of the City of York et al.}, 1983.

A consulting engineer prepared a tendering package.

The package omitted information relevant to the soil and groundwater levels.

The contractor assumed that, as no such information was included in the package, there would be no issues with respect to water during construction.

The contractor did not specifically ask for such reports in preparing his tender.

Did the contractor (Brown \& Hudson) succeed in their suit?

\end{frame}



\begin{frame}
\frametitle{Not Quite All or Nothing}

Partially! The court determined that the contractor was partly negligible for not having looked into the soils report.

The court decided that the contractor was 25\% liable and the consulting engineer 75\% liable.

(Sometimes it's not all or nothing: concurrent tortfeasors again...)

\end{frame}



\begin{frame}
\frametitle{Dirty Business}
In the judgment, the court indicated that:

\begin{quote}
   The engineer must have known that tenderers would rely on the tender package; particularly when the contract documents did not require the contractor to satisfy itself about the subsurface conditions.
\end{quote}
\begin{quote}
   Was the lack of reference to the soil reports and the change of a sketch and plan a negligent omission to convey necessary information?  Information concerning the water and sub-surface conditions was of great significance to any tenderer.  I can think of no good reason why the engineers did not refer to the soils reports in the tender package and no reason for this omission was advanced at trial.
\end{quote}


\end{frame}



\begin{frame}
\frametitle{One More Case}

\textit{Canama Contracting Ltd. v Huffman et al.}, 1983.

An engineer is employed by the Dept. of Agriculture of Ontario.

A contractor had occasionally relied on his advice.

The contractor passed on its design of a barn to be built over a manure pit.

The only communications were by phone, but the engineer looked at the plans and sent the message:

	     ``Good set of plans.  I like the detail.  Wish I could spend that 	    amount of time on each project.  Keep up the good work.''


The engineer, however, failed to note deficiencies in the plan with respect to the placement of rebar.

Part of the walls failed as a result.


\end{frame}



\begin{frame}
\frametitle{\textit{Canama Contracting Ltd. v Huffman et al.}}

The engineer claimed that he was not under the impression he was being consulted by the contractor.


The court found that the engineer did refer to the plans as ``good'', which would consequently ``lull the plaintiff into thinking the plans were adequate.''

The court rejected the idea that the engineer's impression of not being consulted on his opinion was enough.

Instead, the important determination is ``what our conduct objectively makes the other person believe we feel or think''.

Still, was the contractor blameless in this situation?

\end{frame}



\begin{frame}
\frametitle{\textit{Canama Contracting Ltd. v Huffman et al.}}

No -- the contractor should have made it clear that it was intending to ask the engineer for advice upon which it was going to rely.

They were each 50\% at fault in the outcome.

Note also that the court pointed out that a disclaimer of responsibility would have absolved the engineer from negligent misstatement~\cite{lpe}.

\end{frame}



\begin{frame}
\frametitle{Negligent Design}

\textit{SEDCO and Hospital Laundry Services of Regina v William Kelly Holdings Ltd. et al.,} 1988.


The owner hired an architect to design a building.

The architect subcontracted with mechanical engineers to design the ventilation system.

The cooling system was deficient and this resulted in workers having to take ``heat breaks'' throughout the day.

This resulted in financial losses for the owner who sued the engineers for negligence.

Did they succeed?

\end{frame}



\begin{frame}
\frametitle{Negligent Design}

Yes they did.

The courts found the engineers were aware of the working conditions and knew (or should have known) this defect would have harmed the work environment.

There was a breach of the duty of care.

This breach resulted in economic loss of the owner.


\end{frame}


\begin{frame}
\frametitle{Takeaways}

It has been further affirmed in subsequent supreme court decisions that liability may arise without a contract between the two parties.

That a loss is purely economic does not disqualify it from being recoverable.

Similarly, disclaimers have generally been upheld where they exist.\\
\quad But they must be specific and explicit.


Consequently: you must be extremely careful about giving professional advice! 


\end{frame}

\begin{frame}
\frametitle{References \& Disclaimer}
\bibliographystyle{alphaurl}
\setbeamertemplate{bibliography item}{\insertbiblabel}
{\scriptsize
\bibliography{290}
}
\vfill

{\tiny Disclaimer: the material presented in these lectures slides is intended for use in the course ECE~290 at the University of Waterloo and should not be relied upon as legal advice. Any reliance on these course slides by any party for any other purpose are the responsibility of such parties.  The author(s) accept(s) no responsibility for damages, if any, suffered by any party as a result of decisions made or actions based on these course slides for any other purpose than that for which it was intended.\par}


\end{frame}


\end{document}

