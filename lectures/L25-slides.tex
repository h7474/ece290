
\documentclass[letterpaper,hide notes,xcolor={table,svgnames},pdftex,10pt]{beamer}
\def\showexamples{t}


%\usepackage[svgnames]{xcolor}

%% Demo talk
%\documentclass[letterpaper,notes=show]{beamer}

\usecolortheme{crane}
\setbeamertemplate{navigation symbols}{}

\usetheme{MyPittsburgh}
%\usetheme{Frankfurt}

%\usepackage{tipa}

\usepackage{hyperref}
\usepackage{graphicx,xspace}
\usepackage[normalem]{ulem}
\usepackage{multicol}

\newcommand\SF[1]{$\bigstar$\footnote{SF: #1}}

\usepackage[default]{sourcesanspro}
\usepackage[T1]{fontenc}

\newcounter{tmpnumSlide}
\newcounter{tmpnumNote}

% old question code
%\newcommand\question[1]{{$\bigstar$ \small \onlySlide{2}{#1}}}
% \newcommand\nquestion[1]{\ifdefined \presentationonly \textcircled{?} \fi \note{\par{\Large \textbf{?}} #1}}
% \newcommand\nanswer[1]{\note{\par{\Large \textbf{A}} #1}}


 \newcommand\mnote[1]{%
   \addtocounter{tmpnumSlide}{1}
   \ifdefined\showcues {~\tiny\fbox{\arabic{tmpnumSlide}}}\fi
   \note{\setlength{\parskip}{1ex}\addtocounter{tmpnumNote}{1}\textbf{\Large \arabic{tmpnumNote}:} {#1\par}}}

\newcommand\mmnote[1]{\note{\setlength{\parskip}{1ex}#1\par}}

%\newcommand\mnote[2][]{\ifdefined\handoutwithnotes {~\tiny\fbox{#1}}\fi
% \note{\setlength{\parskip}{1ex}\textbf{\Large #1:} #2\par}}

%\newcommand\mnote[2][]{{\tiny\fbox{#1}} \note{\setlength{\parskip}{1ex}\textbf{\Large #1:} #2\par}}

\newcommand\mquestion[2]{{~\color{red}\fbox{?}}\note{\setlength{\parskip}{1ex}\par{\Large \textbf{?}} #1} \note{\setlength{\parskip}{1ex}\par{\Large \textbf{A}} #2\par}\ifdefined \presentationonly \pause \fi}

\newcommand\blackboard[1]{%
\ifdefined   \showblackboard
  {#1}
  \else {\begin{center} \fbox{\colorbox{blue!30}{%
         \begin{minipage}{.95\linewidth}%
           \hspace{\stretch{1}} Some space intentionally left blank; done at the blackboard.%
         \end{minipage}}}\end{center}}%
         \fi%
}



%\newcommand\q{\tikz \node[thick,color=black,shape=circle]{?};}
%\newcommand\q{\ifdefined \presentationonly \textcircled{?} \fi}

\usepackage{listings}
\lstset{%
  keywordstyle=\bfseries,
  aboveskip=15pt,
  belowskip=15pt,
  captionpos=b,
  identifierstyle=\ttfamily,
  escapeinside={(*@}{@*)},
  stringstyle=\ttfamiliy,
  frame=lines,
  numbers=left, basicstyle=\scriptsize, numberstyle=\tiny, stepnumber=0, numbersep=2pt}

\usepackage{siunitx}
\newcommand\sius[1]{\num[group-separator = {,}]{#1}\si{\micro\second}}
\newcommand\sims[1]{\num[group-separator = {,}]{#1}\si{\milli\second}}
\newcommand\sins[1]{\num[group-separator = {,}]{#1}\si{\nano\second}}
\sisetup{group-separator = {,}, group-digits = true}

%% -------------------- tikz --------------------
\usepackage{tikz}
\usetikzlibrary{positioning}
\usetikzlibrary{arrows,backgrounds,automata,decorations.shapes,decorations.pathmorphing,decorations.markings,decorations.text}

\tikzstyle{place}=[circle,draw=blue!50,fill=blue!20,thick, inner sep=0pt,minimum size=6mm]
\tikzstyle{transition}=[rectangle,draw=black!50,fill=black!20,thick, inner sep=0pt,minimum size=4mm]

\tikzstyle{block}=[rectangle,draw=black, thick, inner sep=5pt]
\tikzstyle{bullet}=[circle,draw=black, fill=black, thin, inner sep=2pt]

\tikzstyle{pre}=[<-,shorten <=1pt,>=stealth',semithick]
\tikzstyle{post}=[->,shorten >=1pt,>=stealth',semithick]
\tikzstyle{bi}=[<->,shorten >=1pt,shorten <=1pt, >=stealth',semithick]

\tikzstyle{mut}=[-,>=stealth',semithick]

\tikzstyle{treereset}=[dashed,->, shorten >=1pt,>=stealth',thin]

\usepackage{ifmtarg}
\usepackage{xifthen}
\makeatletter
% new counter to now which frame it is within the sequence
\newcounter{multiframecounter}
% initialize buffer for previously used frame title
\gdef\lastframetitle{\textit{undefined}}
% new environment for a multi-frame
\newenvironment{multiframe}[1][]{%
\ifthenelse{\isempty{#1}}{%
% if no frame title was set via optional parameter,
% only increase sequence counter by 1
\addtocounter{multiframecounter}{1}%
}{%
% new frame title has been provided, thus
% reset sequence counter to 1 and buffer frame title for later use
\setcounter{multiframecounter}{1}%
\gdef\lastframetitle{#1}%
}%
% start conventional frame environment and
% automatically set frame title followed by sequence counter
\begin{frame}%
\frametitle{\lastframetitle~{\normalfont(\arabic{multiframecounter})}}%
}{%
\end{frame}%
}
\makeatother

\makeatletter
\newdimen\tu@tmpa%
\newdimen\ydiffl%
\newdimen\xdiffl%
\newcommand\ydiff[2]{%
    \coordinate (tmpnamea) at (#1);%
    \coordinate (tmpnameb) at (#2);%
    \pgfextracty{\tu@tmpa}{\pgfpointanchor{tmpnamea}{center}}%
    \pgfextracty{\ydiffl}{\pgfpointanchor{tmpnameb}{center}}%
    \advance\ydiffl by -\tu@tmpa%
}
\newcommand\xdiff[2]{%
    \coordinate (tmpnamea) at (#1);%
    \coordinate (tmpnameb) at (#2);%
    \pgfextractx{\tu@tmpa}{\pgfpointanchor{tmpnamea}{center}}%
    \pgfextractx{\xdiffl}{\pgfpointanchor{tmpnameb}{center}}%
    \advance\xdiffl by -\tu@tmpa%
}
\makeatother
\newcommand{\copyrightbox}[3][r]{%
\begin{tikzpicture}%
\node[inner sep=0pt,minimum size=2em](ciimage){#2};
\usefont{OT1}{phv}{n}{n}\fontsize{4}{4}\selectfont
\ydiff{ciimage.south}{ciimage.north}
\xdiff{ciimage.west}{ciimage.east}
\ifthenelse{\equal{#1}{r}}{%
\node[inner sep=0pt,right=1ex of ciimage.south east,anchor=north west,rotate=90]%
{\raggedleft\color{black!50}\parbox{\the\ydiffl}{\raggedright{}#3}};%
}{%
\ifthenelse{\equal{#1}{l}}{%
\node[inner sep=0pt,right=1ex of ciimage.south west,anchor=south west,rotate=90]%
{\raggedleft\color{black!50}\parbox{\the\ydiffl}{\raggedright{}#3}};%
}{%
\node[inner sep=0pt,below=1ex of ciimage.south west,anchor=north west]%
{\raggedleft\color{black!50}\parbox{\the\xdiffl}{\raggedright{}#3}};%
}
}
\end{tikzpicture}
}


%% --------------------

%\usepackage[excludeor]{everyhook}
%\PushPreHook{par}{\setbox0=\lastbox\llap{MUH}}\box0}

%\vspace*{\stretch{1}

%\setbox0=\lastbox \llap{\textbullet\enskip}\box0}

\setlength{\parskip}{\fill}

\newcommand\noskips{\setlength{\parskip}{1ex}}
\newcommand\doskips{\setlength{\parskip}{\fill}}

\newcommand\xx{\par\vspace*{\stretch{1}}\par}
\newcommand\xxs{\par\vspace*{2ex}\par}
\newcommand\tuple[1]{\langle #1 \rangle}
\newcommand\code[1]{{\sf \footnotesize #1}}
\newcommand\ex[1]{\uline{Example:} \ifdefined \presentationonly \pause \fi
  \ifdefined\showexamples#1\xspace\else{\uline{\hspace*{2cm}}}\fi}

\newcommand\ceil[1]{\lceil #1 \rceil}


\AtBeginSection[]
{
   \begin{frame}
       \frametitle{Outline}
       \tableofcontents[currentsection]
   \end{frame}
}



\pgfdeclarelayer{edgelayer}
\pgfdeclarelayer{nodelayer}
\pgfsetlayers{edgelayer,nodelayer,main}

\tikzstyle{none}=[inner sep=0pt]
\tikzstyle{rn}=[circle,fill=Red,draw=Black,line width=0.8 pt]
\tikzstyle{gn}=[circle,fill=Lime,draw=Black,line width=0.8 pt]
\tikzstyle{yn}=[circle,fill=Yellow,draw=Black,line width=0.8 pt]
\tikzstyle{empty}=[circle,fill=White,draw=Black]
\tikzstyle{bw} = [rectangle, draw, fill=blue!20, 
    text width=4em, text centered, rounded corners, minimum height=2em]
    
    \newcommand{\CcNote}[1]{% longname
	This work is licensed under the \textit{Creative Commons #1 3.0 License}.%
}
\newcommand{\CcImageBy}[1]{%
	\includegraphics[scale=#1]{creative_commons/cc_by_30.pdf}%
}
\newcommand{\CcImageSa}[1]{%
	\includegraphics[scale=#1]{creative_commons/cc_sa_30.pdf}%
}
\newcommand{\CcImageNc}[1]{%
	\includegraphics[scale=#1]{creative_commons/cc_nc_30.pdf}%
}
\newcommand{\CcGroupBySa}[2]{% zoom, gap
	\CcImageBy{#1}\hspace*{#2}\CcImageNc{#1}\hspace*{#2}\CcImageSa{#1}%
}
\newcommand{\CcLongnameByNcSa}{Attribution-NonCommercial-ShareAlike}

\newenvironment{changemargin}[1]{% 
  \begin{list}{}{% 
    \setlength{\topsep}{0pt}% 
    \setlength{\leftmargin}{#1}% 
    \setlength{\rightmargin}{1em}
    \setlength{\listparindent}{\parindent}% 
    \setlength{\itemindent}{\parindent}% 
    \setlength{\parsep}{\parskip}% 
  }% 
  \item[]}{\end{list}} 




\title{Lecture 25 --- Discipline \& Enforcement}

\author{Jeff Zarnett, based on original by Douglas Harder \\ \small \texttt{jzarnett@uwaterloo.ca} / \texttt{dwharder@uwaterloo.ca}}
\institute{Department of Electrical and Computer Engineering \\
  University of Waterloo}
\date{\today}


\begin{document}

\begin{frame}
  \titlepage

\begin{center}
  \small{Acknowledgments: Douglas Harder~\cite{dwh}, Julie Vale~\cite{jv}}
  \end{center}
\end{frame}




\begin{frame}
\frametitle{Complaints and Discipline}

Suppose an engineer has potentially committed professional misconduct.

When an incident occurs, or when a member of the public feels that a practitioner has been less than competent, they may register a complaint.

All complaints are reviewed by a Complaints Committee.

Those complaints that are found to have merit are forwarded on to the Discipline Committee.

\end{frame}



\begin{frame}
\frametitle{Complaints and Discipline}

With any complaint, the complainant can discuss the complaint with the Complaint Review Councillor.

This councillor can review how a complaint was handled.

He or she may offer additional information to the complainant in the case where a complaint was not found to have merit.

This position is there for transparency; to help the public understand why a particular complaint was not forwarded on for discipline.

\end{frame}



\begin{frame}
\frametitle{Discipline}

The Discipline Committee will hold a hearing for each complaint and determine whether there has been an incident of either incompetence or professional misconduct.

If a practitioner is found guilty of either of these, the Discipline Committee may then apply a penalty.


\end{frame}



\begin{frame}
\frametitle{Discipline Committee Penalties}

(4) Where the Discipline Committee finds a practitioner guilty of professional misconduct or to be incompetent it may, by order,

(a) revoke any licence or Certificate of Authorization of the practitioner



\end{frame}

\begin{frame}
\frametitle{Discipline Committee Penalties}

(4) Where the Discipline Committee finds a practitioner guilty of professional misconduct or to be incompetent it may, by order,


(b) suspend any licence or certificate of the practitioner for a stated period, not exceeding 24 months


\end{frame}

\begin{frame}
\frametitle{Discipline Committee Penalties}

(4) Where the Discipline Committee finds a practitioner guilty of professional misconduct or to be incompetent it may, by order,


(c) accept the undertaking of the practitioner to limit the
		professional work of the practitioner in the practice
		of professional engineering to the extent specified in the
		undertaking


\end{frame}

\begin{frame}
\frametitle{Discipline Committee Penalties}

(4) Where the Discipline Committee finds a practitioner guilty of professional misconduct or to be incompetent it may, by order,


(d) impose terms, conditions or limitations on the licence or
 		certificate of the practitioner including but not limited to the
		successful completion of a particular course or courses of
		study, as are specified by the Discipline Committee


\end{frame}

\begin{frame}
\frametitle{Discipline Committee Penalties}

(4) Where the Discipline Committee finds a practitioner guilty of professional misconduct or to be incompetent it may, by order,

(e) impose specific restrictions on the licence or certificate
		including but not limited to,

\begin{enumerate}[i)]
\item requiring the practitioner to engage in the practice of professional engineering only under the personal supervision and direction of a member,
\item requiring the Member to not alone engage in the practice of professional engineering,
\item requiring the practitioner to accept periodic inspections by the Committee of documents and records in the possession or under the control of the practitioner in connection with the practice of professional engineering,
\item requiring the practitioner to report to the Registrar on such matters in respect of the practitioner's practice for such period of time, at such times and in such form, as the Discipline Committee may specify
\end{enumerate}

\end{frame}

\begin{frame}
\frametitle{Discipline Committee Penalties}

(4) Where the Discipline Committee finds a practitioner guilty of professional misconduct or to be incompetent it may, by order,


(f) require that the practitioner be reprimanded, admonished
		or counselled and, if considered warranted, direct that the
		fact of the reprimand, admonishment or counselling be
		recorded on the register for a stated or unlimited period of
		time


\end{frame}

\begin{frame}
\frametitle{Discipline Committee Penalties}

(4) Where the Discipline Committee finds a practitioner guilty of professional misconduct or to be incompetent it may, by order,


(g) revoke or suspend for a stated period of time the
		designation of the member or holder by the Association as
		a specialist, consulting engineer or otherwise


\end{frame}

\begin{frame}
\frametitle{Discipline Committee Penalties}

(4) Where the Discipline Committee finds a practitioner guilty of professional misconduct or to be incompetent it may, by order,


(h) impose such fine to a maximum of \$5,000, to be paid by
		the practitioner to the Treasurer of Ontario for payment into
		the Consolidated Revenue Fund


\end{frame}

\begin{frame}
\frametitle{Discipline Committee Penalties}

(4) Where the Discipline Committee finds a practitioner guilty of professional misconduct or to be incompetent it may, by order,

(i) subject to subsection (5) in respect of orders of revocation
		or suspension, direct that the finding and the order of the
		Discipline Committee be published in detail or in summary
		and either with or without including the name of the
		practitioner in the official publication of the Association and
		in such other manner or medium as the Discipline
		Committee considers appropriate in the particular case



\end{frame}

\begin{frame}
\frametitle{Discipline Committee Penalties}

(4) Where the Discipline Committee finds a practitioner guilty of professional misconduct or to be incompetent it may, by order,


(j) fix and impose costs to be paid by the practitioner to the
		Association


\end{frame}

\begin{frame}
\frametitle{Discipline Committee Penalties}

(4) Where the Discipline Committee finds a practitioner guilty of professional misconduct or to be incompetent it may, by order,

(k) direct that the imposition of a penalty be suspended or
		postponed for such period and upon such terms or for such
		purpose as the Discipline Committee may specify, including
		but not limited to,
\begin{enumerate}[i)]
\item the successful completion by the member or the holder of the temporary licence, provisional licence or limited licence of a particular course or courses of study,
\item the production to the Discipline Committee of evidence satisfactory to it that any physical or mental incapacity in respect of which the penalty was imposed has been overcome
\end{enumerate}



\end{frame}

\begin{frame}
\frametitle{Discipline Committee Penalties}

(4) Where the Discipline Committee finds a practitioner guilty of professional misconduct or to be incompetent it may, by order,


\textbf{Or any combination of the above!}


\end{frame}



\begin{frame}
\frametitle{Discipline Case}

Let's consider some sample cases from the PEO Gazette:



``In the matter of a hearing under the Professional Engineers Act, R.S.O. 1990, c. P.28; and in the matter of a complaint regarding the conduct of JIRI KRUPKA, P.ENG., a member of the Association of Professional Engineers of Ontario, and CAELLIOTT INC., a holder of a Certificate of Authorization'':\\
\url{ http://www.peo.on.ca/index.php/ci_id/28567/la_id/1.htm }



\end{frame}


\begin{frame}
\frametitle{The Flexner Report}

Consider the practice of medicine.

At the start of the $20^{th}$ century, ``diploma mills'' not associated with any accredited university were churning out ``medical doctors''.

In this environment, the quality of doctors was suspect at best.

The poor quality of the medical services provided resulted in the expansion of pseudo-scientific practices such as chiropractic and homeopathy.

\end{frame}



\begin{frame}
\frametitle{The Flexner Report}

Published by the Carnegie Foundation, Abraham Flexner's report made a number of recommendations:

\begin{itemize}
	\item Reduce the number of medical schools
	\item Increase the prerequisites to enter medical training
	\item Train physicians to practice in a scientific manner and engage medical faculty in research
	\item Give medical schools control of clinical instruction in hospitals
	\item Strengthen state regulation of medical licensure
\end{itemize}

Modeled on the Johns Hopkins University School of Medicine

\end{frame}



\begin{frame}
\frametitle{Licensing Requirement}

From the Act, section 12:


(1) No person shall engage in the practice of professional engineering or hold himself, herself or itself out as engaging in the practice of professional engineering unless the person is the holder of a licence, a temporary licence, a provisional licence or a limited licence.

(2) No person shall offer to the public or engage in the business of providing to the public services that are within the practice of professional engineering except under and in accordance with a certificate of authorization.

What are the penalties for this?

\end{frame}




\begin{frame}
\frametitle{Licensing Requirement}

Every person who contravenes section 12 is guilty of an offence and on conviction is liable for the first offence to a fine of not more than \$25,000 and for each subsequent offence to a fine of not more than \$50,000. 

(Is this ``enforcement'' or ``discipline''?)

\end{frame}



\begin{frame}
\frametitle{Identification}

It is important that the public can identify engineers.

The Act prevents individuals from holding themselves out as or identifying themselves as being professional engineers or the holder of other licences.

\end{frame}



\begin{frame}
\frametitle{Identification}

\textbf{Offence, use of term ``professional engineer'', etc.}

Every person who is not a holder of a licence or a temporary licence and who,


      (a)	uses the title ``professional engineer'' or ``ing\'enieur'' or an abbreviation or
	variation thereof as an occupational or business designation;\\
      (a.1)	uses the title ``engineer'' or an abbreviation of that title in a
	manner that will lead to the belief that the person may 
	engage in the practice of professional engineering;\\
      (b)	uses a term, title or description that will lead to the belief 
	that the person may engage in the practice of professional 
	engineering; or\\
	        (c)	uses a seal that will lead to the belief that the person is a
			professional engineer,


	is guilty of an offence and on conviction is liable for the first offence to a fine of not more than \$10,000 and for each subsequent offence to a fine of not more than \$25,000.

\end{frame}



\begin{frame}
\frametitle{Identification}

Every person who is not acting under and in accordance with a certificate of authorization and who,


	    (a)	uses a term, title or description that will lead to the belief that the person
		may  provide to the public services that are within the practice of 
		professional engineering; or\\
	    (b) 	uses a seal that will lead to the belief that the person may provide to the
		public services that are within the practice of professional engineering,


	is guilty of an offence and on conviction is liable for the first offence to a fine of not more than \$10,000 and for each subsequent offence to a fine of not more than\$25,000. 

\end{frame}



\begin{frame}
\frametitle{Further Liabilities}

\textbf{Liability of directors and officers}\\
	40. (5)  Where a corporation is guilty of an offence under subsection (1), (2), (3) or (4), every director or officer of the corporation who authorizes, permits or acquiesces in the offence is guilty of an offence and on conviction is liable to a fine of not more than \$50,000.

\textbf{Liability of partners}\\
	40. (6)  Where a person who is guilty of an offence under subsection (1), (2), (3) or (4) is a member or an employee of a partnership, every member of the partnership who authorizes, permits or acquiesces in the offence is guilty of an offence and on conviction is liable to a fine of not more than \$50,000.


\end{frame}



\begin{frame}
\frametitle{Enforcement}

Enforcement is when an individual who does not hold a license or certificate of authorization engages in the practice of professional engineering. 

The Association will enforce the terms of the PEA by asking the courts to step in and grant an injunction.

\end{frame}



\begin{frame}
\frametitle{Enforcement}

\textbf{Order directing compliance}\\
	39. (1) Where it appears to the Association that any person does not comply with this Act or the regulations, despite the imposition of any penalty in respect of such non-compliance and in addition to any other rights it may have, the Association may apply to a judge of the Superior Court of Justice for an order directing the person to comply with the provision, and upon the application the judge may make the order or such other order as the judge thinks fit.


\end{frame}



\begin{frame}
\frametitle{The Case of Mr. Stolarchuk}

An application was brought under Section 39 of the Professional Engineers Act in the Ontario Superior Court of Justice at 130 Queen Street West, Toronto, Ontario, on July 23, 2002 before the Honourable Mr. Justice Somers.  The association obtained the following Order against Dan Stolarchuk of Toronto:

1. A DECLARATION that Dan Stolarchuk breached s. 12(1) of the Act in that, without a licence, he held himself out as engaging in the business of providing, to the Ontario public, services that are within the practice of professional engineering

2. AN ORDER that Stolarchuk refrain from holding himself out as engaging in the business of providing, to the public in Ontario, services that are within the practice of professional engineering, unless and until he obtains a licence from PEO;


\end{frame}

\begin{frame}
\frametitle{The Case of Mr. Stolarchuk}

An application was brought under Section 39 of the Professional Engineers Act in the Ontario Superior Court of Justice at 130 Queen Street West, Toronto, Ontario, on July 23, 2002 before the Honourable Mr. Justice Somers.  The association obtained the following Order against Dan Stolarchuk of Toronto:

3. A DECLARATION that Stolarchuk breached s. 40(2)(a) of the Act in that, without a licence, he used the title ``professional engineer'' and the abbreviated title ``P.Eng.'' as occupational or business designations

4. AN ORDER that Stolarchuk refrain from using the title ``professional  engineer'' or any abbreviations or variation thereof as an occupational or  business designation in Ontario unless and until he obtains a licence

\end{frame}



\begin{frame}
\frametitle{The Case of Mr. Stolarchuk}

5. A DECLARATION that Stolarchuk breached s. 40(2)(b) of the Act in that, without a licence, he used the titles ``professional engineer'' and ``P.Eng.'', terms which would lead to the belief that he could engage in the practice of professional engineering

6. AN ORDER that Stolarchuk refrain from using, by any medium, the term ``professional engineer'' or any variation or abbreviation thereof that will lead to the belief that he provides, to the public in Ontario, services within the practice of professional engineering, unless and until he obtains a licence

7. AN ORDER that Stolarchuk turn over to a representative of PEO all promotional materials, business cards, and any other business stationery or printed materials and signage using the title ``professional engineer'' and/or ``P.Eng.'', in combination with his name and/or any other term in violation of this Order, within 21 days of the date of this Order

8. AN ORDER that Stolarchuk pay to PEO its costs of this matter fixed at \$6,750.

\end{frame}



\begin{frame}
\frametitle{The Case of Mr. Stolarchuk}

The investigation leading to the subsequent application began after PEO received information that Stolarchuk had misrepresented himself as a professional engineer to fellow employees at a Toronto company, as well as describing himself as a ``Field Applications Engineer'' and ``R.D. Engineer'' in a resume.


\end{frame}



\begin{frame}
\frametitle{Wiktor Kwiatek}

In November 2003, 73-year old Kwiatek, P.Eng., was found guilty of incompetence and professional misconduct under 72(2)(a), (b), (d), (g), (h) and (j) and his licence was revoked.

He inspected a fence and determined it was ``structurally adequate, built in accordance with prevailing construction practice in Ontario''.

One month later, it was blown over in a wind storm.

Is this discipline or enforcement?

\end{frame}



\begin{frame}
\frametitle{Wiktor Kwiatek}

With his license revoked, Kiwatek continued to perform engineering work.

In 2006, he used a facsimile of a seal to approve a crane inspection despite there being visible cracks in the structure.

A subsequent investigation found he had sealed approximately 60 other inspections since his licence was revoked.

Is this a matter for discipline or enforcement?

\end{frame}



\begin{frame}
\frametitle{Wiktor Kwiatek}

Enforcement was used to obtain a court order preventing him from using the term ``professional engineer'', ``P.Eng.'' or using the seal.

He was required to pay \$2,500.

(This case was in the PEO Gazette Nov/Dec. 2006)

\end{frame}



\begin{frame}
\frametitle{That Escalated Quickly}


Mohammad Hafeez, of Toronto, was jailed June 10, 2005 for 30 days and ordered to pay costs to PEO of \$19,863.81. 

He was found in contempt of a previous Order of the Ontario Superior Court of Justice for violating the Professional Engineers Act. 

Mr. Hafeez is not, and has never been, licensed as a professional engineer in the Province of Ontario.

in the spring of 2000, Mr. Hafeez had described himself as a ``structural engineer''.  

\end{frame}



\begin{frame}
\frametitle{Mohammad Hafeez}

Under the terms of the 1995 Order, Mr. Hafeez was ordered to:
\begin{itemize}
\item refrain from using the title ``professional engineer'' or an abbreviation or variation thereof as an occupational or business designation;
\item refrain from using a term, title or description that will lead to the belief that he may engage in the business of professional engineering; and
\item surrender to PEO, site signs, seals or title blocks in his possession containing the words any business cards ``professional engineer'', ``engineer'', ``engineering'', or any abbreviation thereof.
\end{itemize}

\end{frame}


\begin{frame}
\frametitle{References \& Disclaimer}
\bibliographystyle{alphaurl}
\setbeamertemplate{bibliography item}{\insertbiblabel}
{\scriptsize
\bibliography{290}
}
\vfill

{\tiny Disclaimer: the material presented in these lectures slides is intended for use in the course ECE~290 at the University of Waterloo and should not be relied upon as legal advice. Any reliance on these course slides by any party for any other purpose are the responsibility of such parties.  The author(s) accept(s) no responsibility for damages, if any, suffered by any party as a result of decisions made or actions based on these course slides for any other purpose than that for which it was intended.\par}


\end{frame}


\end{document}

