
\documentclass[letterpaper,hide notes,xcolor={table,svgnames},pdftex,10pt]{beamer}
\def\showexamples{t}


%\usepackage[svgnames]{xcolor}

%% Demo talk
%\documentclass[letterpaper,notes=show]{beamer}

\usecolortheme{crane}
\setbeamertemplate{navigation symbols}{}

\usetheme{MyPittsburgh}
%\usetheme{Frankfurt}

%\usepackage{tipa}

\usepackage{hyperref}
\usepackage{graphicx,xspace}
\usepackage[normalem]{ulem}
\usepackage{multicol}

\newcommand\SF[1]{$\bigstar$\footnote{SF: #1}}

\usepackage[default]{sourcesanspro}
\usepackage[T1]{fontenc}

\newcounter{tmpnumSlide}
\newcounter{tmpnumNote}

% old question code
%\newcommand\question[1]{{$\bigstar$ \small \onlySlide{2}{#1}}}
% \newcommand\nquestion[1]{\ifdefined \presentationonly \textcircled{?} \fi \note{\par{\Large \textbf{?}} #1}}
% \newcommand\nanswer[1]{\note{\par{\Large \textbf{A}} #1}}


 \newcommand\mnote[1]{%
   \addtocounter{tmpnumSlide}{1}
   \ifdefined\showcues {~\tiny\fbox{\arabic{tmpnumSlide}}}\fi
   \note{\setlength{\parskip}{1ex}\addtocounter{tmpnumNote}{1}\textbf{\Large \arabic{tmpnumNote}:} {#1\par}}}

\newcommand\mmnote[1]{\note{\setlength{\parskip}{1ex}#1\par}}

%\newcommand\mnote[2][]{\ifdefined\handoutwithnotes {~\tiny\fbox{#1}}\fi
% \note{\setlength{\parskip}{1ex}\textbf{\Large #1:} #2\par}}

%\newcommand\mnote[2][]{{\tiny\fbox{#1}} \note{\setlength{\parskip}{1ex}\textbf{\Large #1:} #2\par}}

\newcommand\mquestion[2]{{~\color{red}\fbox{?}}\note{\setlength{\parskip}{1ex}\par{\Large \textbf{?}} #1} \note{\setlength{\parskip}{1ex}\par{\Large \textbf{A}} #2\par}\ifdefined \presentationonly \pause \fi}

\newcommand\blackboard[1]{%
\ifdefined   \showblackboard
  {#1}
  \else {\begin{center} \fbox{\colorbox{blue!30}{%
         \begin{minipage}{.95\linewidth}%
           \hspace{\stretch{1}} Some space intentionally left blank; done at the blackboard.%
         \end{minipage}}}\end{center}}%
         \fi%
}



%\newcommand\q{\tikz \node[thick,color=black,shape=circle]{?};}
%\newcommand\q{\ifdefined \presentationonly \textcircled{?} \fi}

\usepackage{listings}
\lstset{%
  keywordstyle=\bfseries,
  aboveskip=15pt,
  belowskip=15pt,
  captionpos=b,
  identifierstyle=\ttfamily,
  escapeinside={(*@}{@*)},
  stringstyle=\ttfamiliy,
  frame=lines,
  numbers=left, basicstyle=\scriptsize, numberstyle=\tiny, stepnumber=0, numbersep=2pt}

\usepackage{siunitx}
\newcommand\sius[1]{\num[group-separator = {,}]{#1}\si{\micro\second}}
\newcommand\sims[1]{\num[group-separator = {,}]{#1}\si{\milli\second}}
\newcommand\sins[1]{\num[group-separator = {,}]{#1}\si{\nano\second}}
\sisetup{group-separator = {,}, group-digits = true}

%% -------------------- tikz --------------------
\usepackage{tikz}
\usetikzlibrary{positioning}
\usetikzlibrary{arrows,backgrounds,automata,decorations.shapes,decorations.pathmorphing,decorations.markings,decorations.text}

\tikzstyle{place}=[circle,draw=blue!50,fill=blue!20,thick, inner sep=0pt,minimum size=6mm]
\tikzstyle{transition}=[rectangle,draw=black!50,fill=black!20,thick, inner sep=0pt,minimum size=4mm]

\tikzstyle{block}=[rectangle,draw=black, thick, inner sep=5pt]
\tikzstyle{bullet}=[circle,draw=black, fill=black, thin, inner sep=2pt]

\tikzstyle{pre}=[<-,shorten <=1pt,>=stealth',semithick]
\tikzstyle{post}=[->,shorten >=1pt,>=stealth',semithick]
\tikzstyle{bi}=[<->,shorten >=1pt,shorten <=1pt, >=stealth',semithick]

\tikzstyle{mut}=[-,>=stealth',semithick]

\tikzstyle{treereset}=[dashed,->, shorten >=1pt,>=stealth',thin]

\usepackage{ifmtarg}
\usepackage{xifthen}
\makeatletter
% new counter to now which frame it is within the sequence
\newcounter{multiframecounter}
% initialize buffer for previously used frame title
\gdef\lastframetitle{\textit{undefined}}
% new environment for a multi-frame
\newenvironment{multiframe}[1][]{%
\ifthenelse{\isempty{#1}}{%
% if no frame title was set via optional parameter,
% only increase sequence counter by 1
\addtocounter{multiframecounter}{1}%
}{%
% new frame title has been provided, thus
% reset sequence counter to 1 and buffer frame title for later use
\setcounter{multiframecounter}{1}%
\gdef\lastframetitle{#1}%
}%
% start conventional frame environment and
% automatically set frame title followed by sequence counter
\begin{frame}%
\frametitle{\lastframetitle~{\normalfont(\arabic{multiframecounter})}}%
}{%
\end{frame}%
}
\makeatother

\makeatletter
\newdimen\tu@tmpa%
\newdimen\ydiffl%
\newdimen\xdiffl%
\newcommand\ydiff[2]{%
    \coordinate (tmpnamea) at (#1);%
    \coordinate (tmpnameb) at (#2);%
    \pgfextracty{\tu@tmpa}{\pgfpointanchor{tmpnamea}{center}}%
    \pgfextracty{\ydiffl}{\pgfpointanchor{tmpnameb}{center}}%
    \advance\ydiffl by -\tu@tmpa%
}
\newcommand\xdiff[2]{%
    \coordinate (tmpnamea) at (#1);%
    \coordinate (tmpnameb) at (#2);%
    \pgfextractx{\tu@tmpa}{\pgfpointanchor{tmpnamea}{center}}%
    \pgfextractx{\xdiffl}{\pgfpointanchor{tmpnameb}{center}}%
    \advance\xdiffl by -\tu@tmpa%
}
\makeatother
\newcommand{\copyrightbox}[3][r]{%
\begin{tikzpicture}%
\node[inner sep=0pt,minimum size=2em](ciimage){#2};
\usefont{OT1}{phv}{n}{n}\fontsize{4}{4}\selectfont
\ydiff{ciimage.south}{ciimage.north}
\xdiff{ciimage.west}{ciimage.east}
\ifthenelse{\equal{#1}{r}}{%
\node[inner sep=0pt,right=1ex of ciimage.south east,anchor=north west,rotate=90]%
{\raggedleft\color{black!50}\parbox{\the\ydiffl}{\raggedright{}#3}};%
}{%
\ifthenelse{\equal{#1}{l}}{%
\node[inner sep=0pt,right=1ex of ciimage.south west,anchor=south west,rotate=90]%
{\raggedleft\color{black!50}\parbox{\the\ydiffl}{\raggedright{}#3}};%
}{%
\node[inner sep=0pt,below=1ex of ciimage.south west,anchor=north west]%
{\raggedleft\color{black!50}\parbox{\the\xdiffl}{\raggedright{}#3}};%
}
}
\end{tikzpicture}
}


%% --------------------

%\usepackage[excludeor]{everyhook}
%\PushPreHook{par}{\setbox0=\lastbox\llap{MUH}}\box0}

%\vspace*{\stretch{1}

%\setbox0=\lastbox \llap{\textbullet\enskip}\box0}

\setlength{\parskip}{\fill}

\newcommand\noskips{\setlength{\parskip}{1ex}}
\newcommand\doskips{\setlength{\parskip}{\fill}}

\newcommand\xx{\par\vspace*{\stretch{1}}\par}
\newcommand\xxs{\par\vspace*{2ex}\par}
\newcommand\tuple[1]{\langle #1 \rangle}
\newcommand\code[1]{{\sf \footnotesize #1}}
\newcommand\ex[1]{\uline{Example:} \ifdefined \presentationonly \pause \fi
  \ifdefined\showexamples#1\xspace\else{\uline{\hspace*{2cm}}}\fi}

\newcommand\ceil[1]{\lceil #1 \rceil}


\AtBeginSection[]
{
   \begin{frame}
       \frametitle{Outline}
       \tableofcontents[currentsection]
   \end{frame}
}



\pgfdeclarelayer{edgelayer}
\pgfdeclarelayer{nodelayer}
\pgfsetlayers{edgelayer,nodelayer,main}

\tikzstyle{none}=[inner sep=0pt]
\tikzstyle{rn}=[circle,fill=Red,draw=Black,line width=0.8 pt]
\tikzstyle{gn}=[circle,fill=Lime,draw=Black,line width=0.8 pt]
\tikzstyle{yn}=[circle,fill=Yellow,draw=Black,line width=0.8 pt]
\tikzstyle{empty}=[circle,fill=White,draw=Black]
\tikzstyle{bw} = [rectangle, draw, fill=blue!20, 
    text width=4em, text centered, rounded corners, minimum height=2em]
    
    \newcommand{\CcNote}[1]{% longname
	This work is licensed under the \textit{Creative Commons #1 3.0 License}.%
}
\newcommand{\CcImageBy}[1]{%
	\includegraphics[scale=#1]{creative_commons/cc_by_30.pdf}%
}
\newcommand{\CcImageSa}[1]{%
	\includegraphics[scale=#1]{creative_commons/cc_sa_30.pdf}%
}
\newcommand{\CcImageNc}[1]{%
	\includegraphics[scale=#1]{creative_commons/cc_nc_30.pdf}%
}
\newcommand{\CcGroupBySa}[2]{% zoom, gap
	\CcImageBy{#1}\hspace*{#2}\CcImageNc{#1}\hspace*{#2}\CcImageSa{#1}%
}
\newcommand{\CcLongnameByNcSa}{Attribution-NonCommercial-ShareAlike}

\newenvironment{changemargin}[1]{% 
  \begin{list}{}{% 
    \setlength{\topsep}{0pt}% 
    \setlength{\leftmargin}{#1}% 
    \setlength{\rightmargin}{1em}
    \setlength{\listparindent}{\parindent}% 
    \setlength{\itemindent}{\parindent}% 
    \setlength{\parsep}{\parskip}% 
  }% 
  \item[]}{\end{list}} 




\title{Lecture 11  --- Contracts: Breach }

\author{Jeff Zarnett \\ \small \texttt{jzarnett@uwaterloo.ca}}
\institute{Department of Electrical and Computer Engineering \\
  University of Waterloo}
\date{\today}


\begin{document}

\begin{frame}
  \titlepage

\begin{center}
  \small{Acknowledgments: Douglas Harder~\cite{dwh}, Julie Vale~\cite{jv}}
  \end{center}
\end{frame}


 

\begin{frame}
\frametitle{Breach of Contract}

Terms of a contract that require performance are termed \textit{obligations}.

Failure to perform is called a \alert{breach} of the contract.

Breach of contract may result in discharge, but it is not guaranteed.

The party that breaches the contract is the \alert{defaulting} party.

Any other parties to the contract are \alert{innocent} parties or \alert{injured} parties.

\end{frame}



\begin{frame}
\frametitle{Once More Unto the Breach}

1: Imagine A agrees to sell B 10~000 bags of potatoes, and deliver them in yellow paper bags with green labels~\cite{lba}.

A makes a mistake and prints the labels in blue rather than green. 

Did A breach the contract? Can B reject the potatoes? Does B still have to pay?

2: Now imagine B has written in the contract the potatoes must be delivered on Wednesday, but A attempts to deliver them on Friday.


Did A breach the contract? Can B reject the potatoes? Does B still have to pay?

\end{frame}



\begin{frame}
\frametitle{Once More Unto the Breach}

In both cases, A has made a mistake and failed to deliver on the promise.\\
\quad So A has breached the contract, in both scenarios.

First scenario: the breach of the contract is ``minor''; it's a non-essential term.

B cannot reject the potatoes (without committing a breach) or refuse payment.\\
\quad He can sue for damages (that he can demonstrate exist).

Second scenario: the breach of the contract is major: it is an essential term.

B can reject the potatoes and discharge the contract (and does not have to pay).\\
\quad Or, he can accept them, pay for them, and sue for damages from late delivery.

\end{frame}




\begin{frame}
\frametitle{Not Necessary... Or Is It?}

What if one party says or does something that will cause the other party to breach the contract?

Suppose there is a contract to build a house with a shed in the back yard.

The homeowner says verbally to the contractor that the contractor does not have to build the shed in the back yard; the house will suffice.

The contractor does not build the shed.

The homeowner then later sues the contractor, claiming that the written contract includes the shed and its absence constitutes a breach of contract.

Can the homeowner sue for breach of contract?

\end{frame}



\begin{frame}
\frametitle{Not Necessary... Or Is It?}

The verbal agreement did not constitute itself a valid contract because the homeowner did not receive consideration in it.

Thus, the agreement to not require the contractor to build the shed is a gratuitous promise (recall that from earlier) and not legally enforceable.

And yet, ``but for'' this agreement, the contractor would have built the shed.

This is called an \alert{induced breach} of contract.\\
\quad The conduct of the homeowner induced the contractor to breach.

\end{frame}



\begin{frame}
\frametitle{Induced Breach}

The courts will not allow the homeowner in this situation to profit from the induced breach under a doctrine called \alert{equitable estoppel}.

The court ``estops'' the homeowner from enforcing the strict terms of the contract on the grounds that it would be unfair.

The promisee must demonstrate:

\begin{enumerate}
\item There was a promise,
\item That promise was reasonably relied upon,
\item This reliance was legally detrimental to the promisee, and
\item Equity or fairness (justice) requires enforcement of the promise
\end{enumerate}

\end{frame}



\begin{frame}
\frametitle{Induced Breach}

On its own, failure to fulfil a gratuitous promise is not a breach of contract.

However, if there is a direct causal chain between the promise and the breach of contract, then the courts will prevent enforcement of that term.


Note that inducing breach of contract can fall under tort law.\\
\quad A subject we are soon going to examine!

\end{frame}


\begin{frame}
\frametitle{Types of Breaches}

Breach does not automatically discharge the contract.

If the breach is of a minor term, then the contract is still binding on both parties.

If the breach is of a fundamental term, the party committing the breach is still bound, but the injured party can choose to discharge or affirm the contract.

An essential obligation of a contract is said to be a \alert{condition} of the contract.

Other obligations non-essential to the contract are said to be \alert{warranties}.

\end{frame}




\begin{frame}
\frametitle{Remedies}

The innocent/injured parties may seek \alert{remedies} for the breach. 

Breach of a warranty is called a \alert{non-material breach}. 

This may allow suing for damages (money) or specific performance (we'll return to this later).

Breach of a condition is called a \alert{material breach} and it may, as we have discussed, result in discharge of the contract (at the option of the injured party).

The remedies of damages and specific performance are still available.

\end{frame}



\begin{frame}
\frametitle{\textit{Pigott Construction Co. Ltd. v W.J. Crowe Ltd.}, 1963 }

The plaintiff, a general contractor, sued a subcontractor for non-performance.

The subcontractor expected to begin work on 1 Jan. 1957, but this did not occur.

The subcontractor claimed that the delay in the start date constituted a material breach of the contract and they were able to discharge the contract.

The other side, obviously, disagreed.

Who was correct?

\end{frame}



\begin{frame}
\frametitle{\textit{Pigott Construction Co. Ltd. v W.J. Crowe Ltd.}, 1963 }

The Ontario Court of Appeal found that the contract should not be discharged.

The breaches would not have prevented the subcontractor from substantial performance on the part of his obligations.

Thus the breaches are grounds for damages, but not for discharge of contract.

This decision was upheld by the Supreme Court of Canada.

\end{frame}



\begin{frame}
\frametitle{\textit{Pigott Construction Co. Ltd. v W.J. Crowe Ltd.}, 1963 }

In the decision:

\begin{quote}
A breach of contract is a cause of discharge only if its effect is to render it purposeless for the innocent party to proceed further with performance.  Further performance is rendered purposeless if one party either shows an intention no longer to be bound by the contract or breaks a stipulation of major importance to the contract...
\end{quote}

\end{frame}

\begin{frame}
\frametitle{\textit{Pigott Construction Co. Ltd. v W.J. Crowe Ltd.}, 1963 }

In the decision:

\begin{quote}
It may, indeed, be said in general that any breach which prevents substantial performance is a cause of discharge.  Whether performance is substantially prevented or only partially affect is, of course, a question that depends upon the circumstances of each case.\end{quote}


\end{frame}



\begin{frame}
\frametitle{\textit{Jacob \& Youngs, Inc. v. Kent}, 1921}

The contractor used a different brand of pipe than was specified in the contract.

The owner claimed a material breach and refused to make the last payment.

The owner wanted to sue for the cost of removing the wrong piping and installing new piping.

Was this a material or non-material breach?

\end{frame}



\begin{frame}
\frametitle{\textit{Jacob \& Youngs, Inc. v. Kent}, 1921}

The court found it was non-material: the brand of pipe itself was not considered important and did not affect the value of the house.

The owner was only able to sue for the difference in the cost of the pipes.

In this case, that difference was... \$0. Whoops.

Takeaway: it is very difficult to determine what is a material breach and what is a non-material breach?

Due to the difficulty, construction contracts may include a term allowing termination if a professional engineer judges a contractor's work inadequate.

\end{frame}



\begin{frame}
\frametitle{How a Breach May Occur}

There are three ways that a party may breach a contract:

\begin{enumerate}
	\item Express repudiation.
	\item Acting in a way that renders performance impossible.
	\item Failing to perform or tendering a performance that is not equivalent.
\end{enumerate}

We will examine each of these.

\end{frame}



\begin{frame}
\frametitle{Express Repudiation}

Express repudiation is simply a declaration by one of the parties that they will not perform as promised.

The promisee is entitled to treat the contract as immediately ended, find another party to perform, and sue for whatever damages (e.g., higher costs)~\cite{lba}.

If the injured party takes this option, the defaulting party should be advised that the contract is being terminated by the breach and to expect a lawsuit.

A promisee can attempt to insist on performance up until the date originally agreed upon for the performance of the obligation. 

\end{frame}



\begin{frame}
\frametitle{\textit{Avery v. Bowden}, 1855}

A ship was chartered to sail from England to Odessa to pick up cargo~\cite{lba}.

On reaching Odessa, the ship wanted to pick up the cargo, but the agent in Odessa refused to hand it over.

Custom was that there was usually some period of grace to deliver the cargo, but the outright refusal was grounds to immediately discharge the contract.

The captain did not do that. Before the period of grace elapsed, the Crimean War broke out, making the contract impossible.

The contract was discharged by frustration, not breach. No damages awarded.\\
\quad It would have been different if the captain had treated repudiation as breach!

\end{frame}



\begin{frame}
\frametitle{Impossibility}

Performance can be rendered impossible by the actions of one of the parties.

An action, can therefore, be a form of repudiation.\\
\quad Recall the example about selling one's own truck to prevent performance.

Sometimes there is the possibility of \alert{anticipatory breach}.

If A makes a contract to sell B a car on a certain date, but a week before delivery, A sells and delivers it to C instead.

B need not wait until the delivery date to sue; nor can A defend by claiming he can buy it back to deliver it on time~\cite{lba}.

\end{frame}

\begin{frame}
\frametitle{Failure of Performance}

Unlike impossibility and express repudiation, failure is not evident until the time of performance arrives (or during it).

Failure can be~\cite{lba}:

\begin{itemize}
	\item A total failure to perform
	\item A grossly inadequate performance
	\item A failure of a minor detail
	\item Partial performance of a major item
\end{itemize}

The nature of the failure determines the remedies available to the injured party.

\end{frame}



\begin{frame}
\frametitle{Remedies}

The court may be compensated for the breach in several ways:

\begin{itemize}
	\item Damages (monetary)
	\item \textit{Quantum Meruit} 
	\item Specific performance
	\item Injunctions
\end{itemize}

Remember also that the injured party may be freed from its obligations.

\end{frame}



\begin{frame}
\frametitle{Prerequisites for Damages}

The purpose of damages is not to punish the defaulting party, but to restore the injured party to its original state (as much as possible).

Damages must ``flow naturally from the breach'': the loss resulting from the breach must have been what the parties would expect as a consequence.


In \textit{Hadley v. Baxendale}, 1854, the plaintiff operated a mill and required the defendants to transport a broken crankshaft for repairs.

It was delayed in transit.

The crankshaft was essential to the operation of the mill, but this was not impressed upon the defendants at the time the contract was established.

The plaintiff was seeking damages for lost productivity. Did they succeed?

\end{frame}



\begin{frame}
\frametitle{\textit{Hadley v. Baxendale}}

The justice indicated:

\begin{quote}
Where two parties have made a contract which one of them has broken, the damages which the other party ought to receive in respect of such breach of contract should be such as may fairly and reasonably be considered either arising naturally, i.e., according to the usual course of things, from such a breach of contract itself, or such as may be reasonably be supposed to have been in the contemplation of both parties, at the time they made the contract, as the probable result of the breach of it.
\end{quote}

The defendants must be told of the consequences if the plaintiffs seek damages as a result of any breach. 

\end{frame}



\begin{frame}
\frametitle{Direct and Indirect Damages}

Valid damages include any immediate costs that the injured party must incur to complete the terms of the contract.

This may include any additional costs, over and above the costs of the current contract to find another party to complete the terms of the contract.

Damages may also be speculative, if a profit that might have been earned is not realized as a result of a breach. Profit is rarely certain; this can be hard to prove.

The court may also award \alert{nominal damages}. An award of \$1 establishes the validity of the claim, even if no monetary losses were suffered.

\end{frame}


\begin{frame}
\frametitle{Duty to Mitigate}

The injured party has a \alert{duty to mitigate} damages: to do whatever is reasonable to limit the extent of the loss suffered.

In a construction that is partially completed, the injured party would have to, for example, guard against possible water damage.

The courts will consider compensating the party for any costs incurred in protecting the structure. 

But they will not consider water damage to the structure if the injured party took no reasonable steps to protect it.

\end{frame}



\begin{frame}
\frametitle{Penalty and Limitation Clauses}

If a contract includes a \alert{penalty clause}, that is, one indicating damages to be paid in the event of a breach must be commensurate with the actual damage.

Such clauses that are purely punitive will not be upheld by courts.

Many contracting parties will often attempt to limit damages to those that directly result from any breach.

Example: ``In no event whatsoever will the manufacturer be responsible for any indirect or consequential damages howsoever caused.''

\end{frame}



\begin{frame}
\frametitle{\textit{Quantum Meruit}}

When one person requests the services of another and obtains performance, the law implies a promise of reasonable pay for these services~\cite{lba}.

Basically: the performing party is permitted to sue to be paid a reasonable rate for the work already completed.

The law permits a \textit{quantum meruit} (Latin: ``what one has earned'') remedy, only if there is no agreed price stated in a contract.

If one of the parties breaks a contract with a fixed price, and thus discharges it, the law allows the injured party to act as if no contract existed.

Thus, a \textit{quantum meruit} remedy is allowed.

\end{frame}



\begin{frame}
\frametitle{Substantial Compliance}

When a performance has completed the essential objective of a contract, the performing party has provided \alert{substantial compliance}.

Even though minor obligations have not been met.

The completion of E5 was satisfied, even though the developer never bothered to correctly code the light switches consistently.

It does not appear to be worth the effort of the University of Waterloo to require the contractor to fix such minor details.

\end{frame}



\begin{frame}
\frametitle{Specific Performance}

Specific performance is unusual. It is a remedy of equity (fairness).

The court orders that the party breaching the contract, rather than paying damages, must fulfill specific terms of the contract.

Under what circumstances will the court order this?

\end{frame}




\begin{frame}
\frametitle{Specific Performance}

If your answer was something about building something or performing a service, the court will not order specific injunction.

For one thing, slavery is illegal and a performance that is done less than willingly is likely to be a disgruntled performance.

A construction contract would require supervision or other oversight; the court will not make a contractor do it when they could do a bad job.

\end{frame}



\begin{frame}
\frametitle{Specific Performance}

The most common case is: sale of land.\\
\quad The court may order the land to be transferred.

Specific performance can occur in the sale of other items.

A failure to provide a vehicle in a contract to sell it is a breach of contract.

If the vehicle is common (e.g., a 2015 Volkswagen Golf), the injured party can be compensated with damages to purchase a similar or equivalent vehicle.

If the vehicle is rare, unique, antique, etc (e.g., a 1967 Ford Mustang), the court may order the vehicle be transferred to fulfill the contract.

\end{frame}



\begin{frame}
\frametitle{Injunction}

\alert{Injunction}: a court order to restrain a party from acting in a specific manner. 

For this remedy to be available, the contract must have a promise \textit{not} to do something~\cite{lba}.

This need not be in negative terms; it can be as simple as an agreement to use one supplier for a certain product. This implies not using any other suppliers.

Example: non-competition clause. An employee who leaves a company cannot work at another company in the same industry.

If he or she tries to do so, the court may issue an injunction barring the former employee from working at the new company.


\end{frame}


\begin{frame}
\frametitle{Fundamental Breach}

One more thing! Some older literature refers to ``fundamental breach''.

Under this doctrine, a breach of sufficient severity was ruled ``fundamental'', then limitation of liability and exclusion clauses were deemed ineffective.

This situation produced a lot of difficulty in determining whether or not a breach of contract was fundamental or not.

In the 2010 decision in the case \textit{Tercon Contractors Ltd. v. British Columbia (Transportation and Highways)} the Supreme Court eliminated this difficulty.

So, if you find references to this in older material, keep in mind that fundamental breach is no longer a thing.

\end{frame}

\begin{frame}
\frametitle{References \& Disclaimer}
\bibliographystyle{alphaurl}
\setbeamertemplate{bibliography item}{\insertbiblabel}
{\scriptsize
\bibliography{290}
}
\vfill

{\tiny Disclaimer: the material presented in these lectures slides is intended for use in the course ECE~290 at the University of Waterloo and should not be relied upon as legal advice. Any reliance on these course slides by any party for any other purpose are the responsibility of such parties.  The author(s) accept(s) no responsibility for damages, if any, suffered by any party as a result of decisions made or actions based on these course slides for any other purpose than that for which it was intended.\par}


\end{frame}


\end{document}

