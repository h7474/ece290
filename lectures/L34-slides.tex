
\documentclass[letterpaper,hide notes,xcolor={table,svgnames},pdftex,10pt]{beamer}
\def\showexamples{t}


%\usepackage[svgnames]{xcolor}

%% Demo talk
%\documentclass[letterpaper,notes=show]{beamer}

\usecolortheme{crane}
\setbeamertemplate{navigation symbols}{}

\usetheme{MyPittsburgh}
%\usetheme{Frankfurt}

%\usepackage{tipa}

\usepackage{hyperref}
\usepackage{graphicx,xspace}
\usepackage[normalem]{ulem}
\usepackage{multicol}

\newcommand\SF[1]{$\bigstar$\footnote{SF: #1}}

\usepackage[default]{sourcesanspro}
\usepackage[T1]{fontenc}

\newcounter{tmpnumSlide}
\newcounter{tmpnumNote}

% old question code
%\newcommand\question[1]{{$\bigstar$ \small \onlySlide{2}{#1}}}
% \newcommand\nquestion[1]{\ifdefined \presentationonly \textcircled{?} \fi \note{\par{\Large \textbf{?}} #1}}
% \newcommand\nanswer[1]{\note{\par{\Large \textbf{A}} #1}}


 \newcommand\mnote[1]{%
   \addtocounter{tmpnumSlide}{1}
   \ifdefined\showcues {~\tiny\fbox{\arabic{tmpnumSlide}}}\fi
   \note{\setlength{\parskip}{1ex}\addtocounter{tmpnumNote}{1}\textbf{\Large \arabic{tmpnumNote}:} {#1\par}}}

\newcommand\mmnote[1]{\note{\setlength{\parskip}{1ex}#1\par}}

%\newcommand\mnote[2][]{\ifdefined\handoutwithnotes {~\tiny\fbox{#1}}\fi
% \note{\setlength{\parskip}{1ex}\textbf{\Large #1:} #2\par}}

%\newcommand\mnote[2][]{{\tiny\fbox{#1}} \note{\setlength{\parskip}{1ex}\textbf{\Large #1:} #2\par}}

\newcommand\mquestion[2]{{~\color{red}\fbox{?}}\note{\setlength{\parskip}{1ex}\par{\Large \textbf{?}} #1} \note{\setlength{\parskip}{1ex}\par{\Large \textbf{A}} #2\par}\ifdefined \presentationonly \pause \fi}

\newcommand\blackboard[1]{%
\ifdefined   \showblackboard
  {#1}
  \else {\begin{center} \fbox{\colorbox{blue!30}{%
         \begin{minipage}{.95\linewidth}%
           \hspace{\stretch{1}} Some space intentionally left blank; done at the blackboard.%
         \end{minipage}}}\end{center}}%
         \fi%
}



%\newcommand\q{\tikz \node[thick,color=black,shape=circle]{?};}
%\newcommand\q{\ifdefined \presentationonly \textcircled{?} \fi}

\usepackage{listings}
\lstset{%
  keywordstyle=\bfseries,
  aboveskip=15pt,
  belowskip=15pt,
  captionpos=b,
  identifierstyle=\ttfamily,
  escapeinside={(*@}{@*)},
  stringstyle=\ttfamiliy,
  frame=lines,
  numbers=left, basicstyle=\scriptsize, numberstyle=\tiny, stepnumber=0, numbersep=2pt}

\usepackage{siunitx}
\newcommand\sius[1]{\num[group-separator = {,}]{#1}\si{\micro\second}}
\newcommand\sims[1]{\num[group-separator = {,}]{#1}\si{\milli\second}}
\newcommand\sins[1]{\num[group-separator = {,}]{#1}\si{\nano\second}}
\sisetup{group-separator = {,}, group-digits = true}

%% -------------------- tikz --------------------
\usepackage{tikz}
\usetikzlibrary{positioning}
\usetikzlibrary{arrows,backgrounds,automata,decorations.shapes,decorations.pathmorphing,decorations.markings,decorations.text}

\tikzstyle{place}=[circle,draw=blue!50,fill=blue!20,thick, inner sep=0pt,minimum size=6mm]
\tikzstyle{transition}=[rectangle,draw=black!50,fill=black!20,thick, inner sep=0pt,minimum size=4mm]

\tikzstyle{block}=[rectangle,draw=black, thick, inner sep=5pt]
\tikzstyle{bullet}=[circle,draw=black, fill=black, thin, inner sep=2pt]

\tikzstyle{pre}=[<-,shorten <=1pt,>=stealth',semithick]
\tikzstyle{post}=[->,shorten >=1pt,>=stealth',semithick]
\tikzstyle{bi}=[<->,shorten >=1pt,shorten <=1pt, >=stealth',semithick]

\tikzstyle{mut}=[-,>=stealth',semithick]

\tikzstyle{treereset}=[dashed,->, shorten >=1pt,>=stealth',thin]

\usepackage{ifmtarg}
\usepackage{xifthen}
\makeatletter
% new counter to now which frame it is within the sequence
\newcounter{multiframecounter}
% initialize buffer for previously used frame title
\gdef\lastframetitle{\textit{undefined}}
% new environment for a multi-frame
\newenvironment{multiframe}[1][]{%
\ifthenelse{\isempty{#1}}{%
% if no frame title was set via optional parameter,
% only increase sequence counter by 1
\addtocounter{multiframecounter}{1}%
}{%
% new frame title has been provided, thus
% reset sequence counter to 1 and buffer frame title for later use
\setcounter{multiframecounter}{1}%
\gdef\lastframetitle{#1}%
}%
% start conventional frame environment and
% automatically set frame title followed by sequence counter
\begin{frame}%
\frametitle{\lastframetitle~{\normalfont(\arabic{multiframecounter})}}%
}{%
\end{frame}%
}
\makeatother

\makeatletter
\newdimen\tu@tmpa%
\newdimen\ydiffl%
\newdimen\xdiffl%
\newcommand\ydiff[2]{%
    \coordinate (tmpnamea) at (#1);%
    \coordinate (tmpnameb) at (#2);%
    \pgfextracty{\tu@tmpa}{\pgfpointanchor{tmpnamea}{center}}%
    \pgfextracty{\ydiffl}{\pgfpointanchor{tmpnameb}{center}}%
    \advance\ydiffl by -\tu@tmpa%
}
\newcommand\xdiff[2]{%
    \coordinate (tmpnamea) at (#1);%
    \coordinate (tmpnameb) at (#2);%
    \pgfextractx{\tu@tmpa}{\pgfpointanchor{tmpnamea}{center}}%
    \pgfextractx{\xdiffl}{\pgfpointanchor{tmpnameb}{center}}%
    \advance\xdiffl by -\tu@tmpa%
}
\makeatother
\newcommand{\copyrightbox}[3][r]{%
\begin{tikzpicture}%
\node[inner sep=0pt,minimum size=2em](ciimage){#2};
\usefont{OT1}{phv}{n}{n}\fontsize{4}{4}\selectfont
\ydiff{ciimage.south}{ciimage.north}
\xdiff{ciimage.west}{ciimage.east}
\ifthenelse{\equal{#1}{r}}{%
\node[inner sep=0pt,right=1ex of ciimage.south east,anchor=north west,rotate=90]%
{\raggedleft\color{black!50}\parbox{\the\ydiffl}{\raggedright{}#3}};%
}{%
\ifthenelse{\equal{#1}{l}}{%
\node[inner sep=0pt,right=1ex of ciimage.south west,anchor=south west,rotate=90]%
{\raggedleft\color{black!50}\parbox{\the\ydiffl}{\raggedright{}#3}};%
}{%
\node[inner sep=0pt,below=1ex of ciimage.south west,anchor=north west]%
{\raggedleft\color{black!50}\parbox{\the\xdiffl}{\raggedright{}#3}};%
}
}
\end{tikzpicture}
}


%% --------------------

%\usepackage[excludeor]{everyhook}
%\PushPreHook{par}{\setbox0=\lastbox\llap{MUH}}\box0}

%\vspace*{\stretch{1}

%\setbox0=\lastbox \llap{\textbullet\enskip}\box0}

\setlength{\parskip}{\fill}

\newcommand\noskips{\setlength{\parskip}{1ex}}
\newcommand\doskips{\setlength{\parskip}{\fill}}

\newcommand\xx{\par\vspace*{\stretch{1}}\par}
\newcommand\xxs{\par\vspace*{2ex}\par}
\newcommand\tuple[1]{\langle #1 \rangle}
\newcommand\code[1]{{\sf \footnotesize #1}}
\newcommand\ex[1]{\uline{Example:} \ifdefined \presentationonly \pause \fi
  \ifdefined\showexamples#1\xspace\else{\uline{\hspace*{2cm}}}\fi}

\newcommand\ceil[1]{\lceil #1 \rceil}


\AtBeginSection[]
{
   \begin{frame}
       \frametitle{Outline}
       \tableofcontents[currentsection]
   \end{frame}
}



\pgfdeclarelayer{edgelayer}
\pgfdeclarelayer{nodelayer}
\pgfsetlayers{edgelayer,nodelayer,main}

\tikzstyle{none}=[inner sep=0pt]
\tikzstyle{rn}=[circle,fill=Red,draw=Black,line width=0.8 pt]
\tikzstyle{gn}=[circle,fill=Lime,draw=Black,line width=0.8 pt]
\tikzstyle{yn}=[circle,fill=Yellow,draw=Black,line width=0.8 pt]
\tikzstyle{empty}=[circle,fill=White,draw=Black]
\tikzstyle{bw} = [rectangle, draw, fill=blue!20, 
    text width=4em, text centered, rounded corners, minimum height=2em]
    
    \newcommand{\CcNote}[1]{% longname
	This work is licensed under the \textit{Creative Commons #1 3.0 License}.%
}
\newcommand{\CcImageBy}[1]{%
	\includegraphics[scale=#1]{creative_commons/cc_by_30.pdf}%
}
\newcommand{\CcImageSa}[1]{%
	\includegraphics[scale=#1]{creative_commons/cc_sa_30.pdf}%
}
\newcommand{\CcImageNc}[1]{%
	\includegraphics[scale=#1]{creative_commons/cc_nc_30.pdf}%
}
\newcommand{\CcGroupBySa}[2]{% zoom, gap
	\CcImageBy{#1}\hspace*{#2}\CcImageNc{#1}\hspace*{#2}\CcImageSa{#1}%
}
\newcommand{\CcLongnameByNcSa}{Attribution-NonCommercial-ShareAlike}

\newenvironment{changemargin}[1]{% 
  \begin{list}{}{% 
    \setlength{\topsep}{0pt}% 
    \setlength{\leftmargin}{#1}% 
    \setlength{\rightmargin}{1em}
    \setlength{\listparindent}{\parindent}% 
    \setlength{\itemindent}{\parindent}% 
    \setlength{\parsep}{\parskip}% 
  }% 
  \item[]}{\end{list}} 




\title{Lecture 34 --- Whistleblowing }

\author{Jeff Zarnett, based on original by Douglas Harder \\ \small \texttt{jzarnett@uwaterloo.ca} / \texttt{dwharder@uwaterloo.ca}}
\institute{Department of Electrical and Computer Engineering \\
  University of Waterloo}
\date{\today}


\begin{document}

\begin{frame}
  \titlepage

\begin{center}
  \small{Acknowledgments: Douglas Harder~\cite{dwh}, Julie Vale~\cite{jv}}
  \end{center}
\end{frame}



\begin{frame}
\frametitle{Whistleblowing}

A significant amount of this topic is taken directly or paraphrased from PEO's GUIDELINE:  Professional Engineering Practice, a document you can download from PEO's web site.

\url{http://peo.on.ca/index.php/ci_id/22127/la_id/1.htm}

\end{frame}



\begin{frame}
\frametitle{Duty to Report}

Engineers are responsible for ensuring that the products that we design and develop are safe for the public.

What happens if company policy or culture is such that it is inevitably leading to harm?

Harm may take many forms: bodily injury, economic loss, et cetera.

\end{frame}



\begin{frame}
\frametitle{Whistleblowing}

\alert{Whistleblowing} is the term when an employee or insider makes a disclosure of malfeasance or wrongdoing. 

This is sometimes called ``blowing the whistle''.

The disclosure may be internal (within the company, for example, to the CEO) or external (for example, to PEO or to the government).

\end{frame}



\begin{frame}
\frametitle{Duty to Report}

Any time engineers see:
\begin{itemize}
	\item A defect in a design 
	\item An unsafe plan
	\item A faulty conclusion in an analysis
	\item An incomplete evaluation
	\item Poor advice given 
	\item Insufficient direction
\end{itemize}

... within the scope of their practice, there is a duty to report.

\end{frame}



\begin{frame}
\frametitle{Duty to Report}

A duty to report means that there is action you must take failure to do so may be professional misconduct or result in civil liability.

Remember from earlier the discussion of ``Duty of Care''.

A practicing engineer has a duty of care to the public.\\
\quad Hence, a responsibility to act.

Same in other professions; for example, if a psychiatrist believes a patient is likely to commit a murder, the psychiatrist must report this to the police.

\end{frame}



\begin{frame}
\frametitle{Duty to Report - Competence}

The duty to report relates to situations related to engineers' ability to apply judgment based on their professional training, experience and competence.

Note that it is a duty to \textit{report}, not solve.\\
\quad Unless the engineer has the authority to make the changes.

In most cases it is someone else's responsibility and all that is required is to make those others aware of the issue(s). 


\end{frame}



\begin{frame}
\frametitle{Push it to the Limit}

When is it okay to go outside the privacy of a contract or employment?\\
\quad Or in other words ``go public''?

Only when it is a matter that affects the public welfare or public interest.\\
\quad And presumably, other approaches have not succeeded.

\end{frame}



\begin{frame}
\frametitle{Step By Step}

An engineer would proceed as follows:
 
1. Ensure the problem is real and that the harm is a reasonable consequence of the issue.\\ \quad The severity of the consequences will dictate the urgency required.

2. Determine whom should be informed.\\
\quad As a mater of fairness, loyalty and being a faithful trustee or agent, one would always approach one's employer or client first.

3. Advise the employer/client and, if possible, provide possible remedial steps.\\
\quad Ensure that the individual is clearly aware of the consequences.

\end{frame}



\begin{frame}
\frametitle{Decision Time}

At this point, the engineer must decide:

``Will the consequential harm affect the public welfare?''

If the answer is no, it is an internal issue and your responsibilities are met.

\end{frame}



\begin{frame}
\frametitle{Internal or External}
Which of the following are purely internal and do not affect public welfare?

\begin{itemize}
\item A colleague is engaging in racist behaviour, ignoring or belittling those from other cultures or ethnicities
\item The design will have significant consequences on the extensibility of the system, thereby necessarily incurring significant additional expenses in the future
\item There is evidence of bribery between your employer and another client
\item A mechanical engineer is designing the framework for a software system
\item An engineer approved a report prepared by a subordinate that he did not have time to read
\item Any behaviour that is illegal
\end{itemize}


\end{frame}




\begin{frame}
\frametitle{Steps Continue}

If there is the potential for harm to the public welfare, the engineer should take the following additional steps:

4. Follow up with the employer or client within a reasonable period of time.\\
\quad You should inform them that you are obligated to report if the issue is not addressed appropriately.

You are required to do so under the Professional Engineers Act.

5. If the issue is not addressed, discuss the matter with colleagues.\\
\quad Again, you should maintain privacy of contract -- talk to your fellow engineers in-house where at all possible.

6. If the issue is not resolved, follow up with higher management.\\
\quad Again, reiterate that you are obligated to report under the Act


\end{frame}



\begin{frame}
\frametitle{Still Nothing?}

If the issue is not being addressed, it may be necessary to go external.

What about any contractual obligations, employment contracts, or non-disclosure agreements?


\end{frame}



\begin{frame}
\frametitle{NDA}

A non-disclosure agreement is a contract between the parties and an agreement is legally enforceable only if it has legal purpose.

Suppose is an imminent threat to life, health, property, economic interests, the public welfare or the environment.

Any agreement requiring confidentiality on such matters means that such a contract will no longer have legal purpose.

Threats to life, health, property and the environment can be more objectively quantified; what about threats to economic interests \& the public welfare?


\end{frame}



\begin{frame}
\frametitle{Blow the Whistle}

If you feel that there is an immanent threat to the public welfare, the next step is to blow the whistle.

The first step is to approach a government regulatory body or a government ministry or ombudsperson.

There must be an extreme situation before an engineer would approach either the media or a private watchdog agency.

Any professional engineer can always contact PEO at this step of the process; there is a hotline phone number you can call.


\end{frame}



\begin{frame}
\frametitle{Reporting Incompetence or Misconduct}

When must a professional engineer report professional misconduct or incompetence on the part of another engineer?

The Act only requires reporting when a professional engineer determines a situation is either unsafe or detrimental to the public welfare.

If the incompetence or misconduct of a professional engineer becomes threatening, an engineer would issue a complaint as any other person.


\end{frame}



\begin{frame}
\frametitle{How Likely Is This Anyway?}

How often will engineers find themselves in such a position?

How often will they be confronted with corrupt and unprincipled managers or executives bent on throwing all caution and care to the wind...?


It is unlikely that even one student in this class will ever see such behaviour in his or her career -- not impossible, but highly improbable.

\end{frame}



\begin{frame}
\frametitle{Hollywood}

These sorts of things make for good movies but life is rarely so dramatic.

The more likely causes are inaccuracy, carelessness and inattentiveness.

Only occasionally is lack of scientific or mathematical understanding the cause of such harm. 

Malice, greed, or indifference to the public is even less likely.


\end{frame}



\begin{frame}
\frametitle{Care in Engineering}

The very nature of the engineering profession is directly associated with mitigating harm; when that fails, people, property \& the public are injured. 

Thus, engineers must, by their training and obligations, take greater care.

Alpern's principle of proportionate care:
\begin{quote}
When one is in a  position to contribute to greater harm or when one is in a position to play a more critical part in causing harm than is another person, one must exercise greater care to avoid doing so.
\end{quote}

\end{frame}




\begin{frame}
\frametitle{Whose Fault Is It Anyway}

In any large organization, there will be dilution of individual responsibility.

Engineers will change projects, teams, divisions and companies and, these days, seldom will they hold the same position for many years.


Dennis Thompson describes this as a ``problem of many hands''.


Alternatively, should a single individual be held responsible for what is often a systemic problem?



\end{frame}

\begin{frame}
\frametitle{Life as a Whistleblower Ain't Easy}

Sadly, whistleblowers often face retribution or punishment.\\
\quad They are fired, defamed in the press, blacklisted from the industry...

Such reprisals are generally illegal, but that doesn't mean they don't happen.

They may also face legal action based on what they did.\\
\quad Civil or criminal proceedings, depending on the exact circumstances.

If you should ever find yourself in this situation, you should be aware of this.

I sincerely hope that you, or I, could find the courage to do the right thing...

\end{frame}

\begin{frame}
\frametitle{References \& Disclaimer}
\bibliographystyle{alphaurl}
\setbeamertemplate{bibliography item}{\insertbiblabel}
{\scriptsize
\bibliography{290}
}
\vfill

{\tiny Disclaimer: the material presented in these lectures slides is intended for use in the course ECE~290 at the University of Waterloo and should not be relied upon as legal advice. Any reliance on these course slides by any party for any other purpose are the responsibility of such parties.  The author(s) accept(s) no responsibility for damages, if any, suffered by any party as a result of decisions made or actions based on these course slides for any other purpose than that for which it was intended.\par}


\end{frame}


\end{document}

