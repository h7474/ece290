
\documentclass[letterpaper,hide notes,xcolor={table,svgnames},pdftex,10pt]{beamer}
\def\showexamples{t}


%\usepackage[svgnames]{xcolor}

%% Demo talk
%\documentclass[letterpaper,notes=show]{beamer}

\usecolortheme{crane}
\setbeamertemplate{navigation symbols}{}

\usetheme{MyPittsburgh}
%\usetheme{Frankfurt}

%\usepackage{tipa}

\usepackage{hyperref}
\usepackage{graphicx,xspace}
\usepackage[normalem]{ulem}
\usepackage{multicol}

\newcommand\SF[1]{$\bigstar$\footnote{SF: #1}}

\usepackage[default]{sourcesanspro}
\usepackage[T1]{fontenc}

\newcounter{tmpnumSlide}
\newcounter{tmpnumNote}

% old question code
%\newcommand\question[1]{{$\bigstar$ \small \onlySlide{2}{#1}}}
% \newcommand\nquestion[1]{\ifdefined \presentationonly \textcircled{?} \fi \note{\par{\Large \textbf{?}} #1}}
% \newcommand\nanswer[1]{\note{\par{\Large \textbf{A}} #1}}


 \newcommand\mnote[1]{%
   \addtocounter{tmpnumSlide}{1}
   \ifdefined\showcues {~\tiny\fbox{\arabic{tmpnumSlide}}}\fi
   \note{\setlength{\parskip}{1ex}\addtocounter{tmpnumNote}{1}\textbf{\Large \arabic{tmpnumNote}:} {#1\par}}}

\newcommand\mmnote[1]{\note{\setlength{\parskip}{1ex}#1\par}}

%\newcommand\mnote[2][]{\ifdefined\handoutwithnotes {~\tiny\fbox{#1}}\fi
% \note{\setlength{\parskip}{1ex}\textbf{\Large #1:} #2\par}}

%\newcommand\mnote[2][]{{\tiny\fbox{#1}} \note{\setlength{\parskip}{1ex}\textbf{\Large #1:} #2\par}}

\newcommand\mquestion[2]{{~\color{red}\fbox{?}}\note{\setlength{\parskip}{1ex}\par{\Large \textbf{?}} #1} \note{\setlength{\parskip}{1ex}\par{\Large \textbf{A}} #2\par}\ifdefined \presentationonly \pause \fi}

\newcommand\blackboard[1]{%
\ifdefined   \showblackboard
  {#1}
  \else {\begin{center} \fbox{\colorbox{blue!30}{%
         \begin{minipage}{.95\linewidth}%
           \hspace{\stretch{1}} Some space intentionally left blank; done at the blackboard.%
         \end{minipage}}}\end{center}}%
         \fi%
}



%\newcommand\q{\tikz \node[thick,color=black,shape=circle]{?};}
%\newcommand\q{\ifdefined \presentationonly \textcircled{?} \fi}

\usepackage{listings}
\lstset{%
  keywordstyle=\bfseries,
  aboveskip=15pt,
  belowskip=15pt,
  captionpos=b,
  identifierstyle=\ttfamily,
  escapeinside={(*@}{@*)},
  stringstyle=\ttfamiliy,
  frame=lines,
  numbers=left, basicstyle=\scriptsize, numberstyle=\tiny, stepnumber=0, numbersep=2pt}

\usepackage{siunitx}
\newcommand\sius[1]{\num[group-separator = {,}]{#1}\si{\micro\second}}
\newcommand\sims[1]{\num[group-separator = {,}]{#1}\si{\milli\second}}
\newcommand\sins[1]{\num[group-separator = {,}]{#1}\si{\nano\second}}
\sisetup{group-separator = {,}, group-digits = true}

%% -------------------- tikz --------------------
\usepackage{tikz}
\usetikzlibrary{positioning}
\usetikzlibrary{arrows,backgrounds,automata,decorations.shapes,decorations.pathmorphing,decorations.markings,decorations.text}

\tikzstyle{place}=[circle,draw=blue!50,fill=blue!20,thick, inner sep=0pt,minimum size=6mm]
\tikzstyle{transition}=[rectangle,draw=black!50,fill=black!20,thick, inner sep=0pt,minimum size=4mm]

\tikzstyle{block}=[rectangle,draw=black, thick, inner sep=5pt]
\tikzstyle{bullet}=[circle,draw=black, fill=black, thin, inner sep=2pt]

\tikzstyle{pre}=[<-,shorten <=1pt,>=stealth',semithick]
\tikzstyle{post}=[->,shorten >=1pt,>=stealth',semithick]
\tikzstyle{bi}=[<->,shorten >=1pt,shorten <=1pt, >=stealth',semithick]

\tikzstyle{mut}=[-,>=stealth',semithick]

\tikzstyle{treereset}=[dashed,->, shorten >=1pt,>=stealth',thin]

\usepackage{ifmtarg}
\usepackage{xifthen}
\makeatletter
% new counter to now which frame it is within the sequence
\newcounter{multiframecounter}
% initialize buffer for previously used frame title
\gdef\lastframetitle{\textit{undefined}}
% new environment for a multi-frame
\newenvironment{multiframe}[1][]{%
\ifthenelse{\isempty{#1}}{%
% if no frame title was set via optional parameter,
% only increase sequence counter by 1
\addtocounter{multiframecounter}{1}%
}{%
% new frame title has been provided, thus
% reset sequence counter to 1 and buffer frame title for later use
\setcounter{multiframecounter}{1}%
\gdef\lastframetitle{#1}%
}%
% start conventional frame environment and
% automatically set frame title followed by sequence counter
\begin{frame}%
\frametitle{\lastframetitle~{\normalfont(\arabic{multiframecounter})}}%
}{%
\end{frame}%
}
\makeatother

\makeatletter
\newdimen\tu@tmpa%
\newdimen\ydiffl%
\newdimen\xdiffl%
\newcommand\ydiff[2]{%
    \coordinate (tmpnamea) at (#1);%
    \coordinate (tmpnameb) at (#2);%
    \pgfextracty{\tu@tmpa}{\pgfpointanchor{tmpnamea}{center}}%
    \pgfextracty{\ydiffl}{\pgfpointanchor{tmpnameb}{center}}%
    \advance\ydiffl by -\tu@tmpa%
}
\newcommand\xdiff[2]{%
    \coordinate (tmpnamea) at (#1);%
    \coordinate (tmpnameb) at (#2);%
    \pgfextractx{\tu@tmpa}{\pgfpointanchor{tmpnamea}{center}}%
    \pgfextractx{\xdiffl}{\pgfpointanchor{tmpnameb}{center}}%
    \advance\xdiffl by -\tu@tmpa%
}
\makeatother
\newcommand{\copyrightbox}[3][r]{%
\begin{tikzpicture}%
\node[inner sep=0pt,minimum size=2em](ciimage){#2};
\usefont{OT1}{phv}{n}{n}\fontsize{4}{4}\selectfont
\ydiff{ciimage.south}{ciimage.north}
\xdiff{ciimage.west}{ciimage.east}
\ifthenelse{\equal{#1}{r}}{%
\node[inner sep=0pt,right=1ex of ciimage.south east,anchor=north west,rotate=90]%
{\raggedleft\color{black!50}\parbox{\the\ydiffl}{\raggedright{}#3}};%
}{%
\ifthenelse{\equal{#1}{l}}{%
\node[inner sep=0pt,right=1ex of ciimage.south west,anchor=south west,rotate=90]%
{\raggedleft\color{black!50}\parbox{\the\ydiffl}{\raggedright{}#3}};%
}{%
\node[inner sep=0pt,below=1ex of ciimage.south west,anchor=north west]%
{\raggedleft\color{black!50}\parbox{\the\xdiffl}{\raggedright{}#3}};%
}
}
\end{tikzpicture}
}


%% --------------------

%\usepackage[excludeor]{everyhook}
%\PushPreHook{par}{\setbox0=\lastbox\llap{MUH}}\box0}

%\vspace*{\stretch{1}

%\setbox0=\lastbox \llap{\textbullet\enskip}\box0}

\setlength{\parskip}{\fill}

\newcommand\noskips{\setlength{\parskip}{1ex}}
\newcommand\doskips{\setlength{\parskip}{\fill}}

\newcommand\xx{\par\vspace*{\stretch{1}}\par}
\newcommand\xxs{\par\vspace*{2ex}\par}
\newcommand\tuple[1]{\langle #1 \rangle}
\newcommand\code[1]{{\sf \footnotesize #1}}
\newcommand\ex[1]{\uline{Example:} \ifdefined \presentationonly \pause \fi
  \ifdefined\showexamples#1\xspace\else{\uline{\hspace*{2cm}}}\fi}

\newcommand\ceil[1]{\lceil #1 \rceil}


\AtBeginSection[]
{
   \begin{frame}
       \frametitle{Outline}
       \tableofcontents[currentsection]
   \end{frame}
}



\pgfdeclarelayer{edgelayer}
\pgfdeclarelayer{nodelayer}
\pgfsetlayers{edgelayer,nodelayer,main}

\tikzstyle{none}=[inner sep=0pt]
\tikzstyle{rn}=[circle,fill=Red,draw=Black,line width=0.8 pt]
\tikzstyle{gn}=[circle,fill=Lime,draw=Black,line width=0.8 pt]
\tikzstyle{yn}=[circle,fill=Yellow,draw=Black,line width=0.8 pt]
\tikzstyle{empty}=[circle,fill=White,draw=Black]
\tikzstyle{bw} = [rectangle, draw, fill=blue!20, 
    text width=4em, text centered, rounded corners, minimum height=2em]
    
    \newcommand{\CcNote}[1]{% longname
	This work is licensed under the \textit{Creative Commons #1 3.0 License}.%
}
\newcommand{\CcImageBy}[1]{%
	\includegraphics[scale=#1]{creative_commons/cc_by_30.pdf}%
}
\newcommand{\CcImageSa}[1]{%
	\includegraphics[scale=#1]{creative_commons/cc_sa_30.pdf}%
}
\newcommand{\CcImageNc}[1]{%
	\includegraphics[scale=#1]{creative_commons/cc_nc_30.pdf}%
}
\newcommand{\CcGroupBySa}[2]{% zoom, gap
	\CcImageBy{#1}\hspace*{#2}\CcImageNc{#1}\hspace*{#2}\CcImageSa{#1}%
}
\newcommand{\CcLongnameByNcSa}{Attribution-NonCommercial-ShareAlike}

\newenvironment{changemargin}[1]{% 
  \begin{list}{}{% 
    \setlength{\topsep}{0pt}% 
    \setlength{\leftmargin}{#1}% 
    \setlength{\rightmargin}{1em}
    \setlength{\listparindent}{\parindent}% 
    \setlength{\itemindent}{\parindent}% 
    \setlength{\parsep}{\parskip}% 
  }% 
  \item[]}{\end{list}} 




\title{Lecture 20 --- Intellectual Property }

\author{Jeff Zarnett, based on original by Douglas Harder \\ \small \texttt{jzarnett@uwaterloo.ca} / \texttt{dwharder@uwaterloo.ca}}
\institute{Department of Electrical and Computer Engineering \\
  University of Waterloo}
\date{\today}


\begin{document}

\begin{frame}
  \titlepage

\begin{center}
  \small{Acknowledgments: Douglas Harder~\cite{dwh}, Julie Vale~\cite{jv}}
  \end{center}
\end{frame}



\begin{frame}
\frametitle{Property Law}

Property is generally divided into a few distinct areas:

\begin{itemize}
	\item Real property (land)
	\item Personal property (sometimes called \textit{chattel})
\end{itemize}

Personal property can be broken down into tangible and intangible property.

Intellectual property is intangible. 

\end{frame}



\begin{frame}
\frametitle{Intellectual Property}

Intellectual property is divided into:
\begin{itemize}
	\item Rights concerning the reproduction of works (copyright).
	\item Industrial property rights.
\end{itemize}

Industrial property rights are broken down into:

\begin{itemize}
	\item Patents
	\item Trademarks
	\item Industrial design
	\item Trade secrets
\end{itemize}


\end{frame}



\begin{frame}
\frametitle{Intellectual Property}

In Canada, legislation of copyrights and patens is a federal power:

\begin{quote}
91. It shall be lawful for the Queen to make Laws for the Peace, Order, and good Government of Canada; and it is hereby declared that the exclusive Legislative Authority of the Parliament of Canada extends to the Subjects next hereinafter enumerated; that is to say,
\\~\\

22. Patents of Invention and Discovery.\\
23.	Copyrights.
\end{quote}

\end{frame}



\begin{frame}
\frametitle{Copyright}

Copyright gives the creator of an original work the exclusive right to its production, reproduction, performance or publication.

Copyright is generally in force for the life of the author plus anywhere from 50 to 100 years.

It gets worse, because copyrights keep getting extended (hello Disney...)


\textbf{Term of copyright}\\

6. The term for which copyright shall subsist shall, except as otherwise expressly provided by this Act, be the life of the author, the remainder of the calendar year in which the author dies, and a period of fifty years following the end of that calendar year.

\end{frame}



\begin{frame}
\frametitle{Copyright}

What can be copyrighted?


	PART I	Copyright and moral rights in works.


	PART II	Copyright in performer's performances, sound recordings and communication signals and moral rights in performers' performances.


\end{frame}



\begin{frame}
\frametitle{Moral Rights}

What are moral rights?

Copyrights can be sold by the original owner.

The original author should still have the right to be attributed, to require the integrity of the work and with what the work is associated.

For example, would you want a photograph you took associated with an advertisement for the KKK?

\end{frame}



\begin{frame}
\frametitle{Assignment of Copyright}

For the copyright of a work to be assigned to another party, that assignment must be written and signed by the holder of the copyright.

\end{frame}



\begin{frame}
\frametitle{Patents}

A patent is a contract between an inventor and the public.

From Latin, \textit{patere} means ``to lay open''.

The inventor makes the design available to the public.

In return, the inventor has exclusive rights to that invention for a fixed period.

What are the goals of this system?

\end{frame}



\begin{frame}
\frametitle{Patents}

Some possible answers...

To advance science and engineering by making discoveries available for everyone to read and understand.

To reward people who do the research, design, and testing of an invention with exclusive rights so others cannot buy it in a store and then immediately copy it.

If you are cynical: to allow some people to extort companies and stifle progress for their own personal financial gain (and to make lawyers rich)!

\end{frame}




\begin{frame}
\frametitle{Patents of Invention}

The earliest modern patent was to Filippo Brunelleschi who received a 3-year patent for a barge with hoisting gear that carried marble.


The first English patent was to John of Utynam  who received a 20-year monopoly from King Henry VI in 1449 to make stained glass.

\end{frame}



\begin{frame}
\frametitle{Patents in Canada}

In Canada, patents are covered under the Patent Act

\begin{quote}
``invention'' means any new and useful art, process, machine, manufacture or composition of matter, or any new and useful improvement in any art, process, machine, manufacture or composition of matter
\end{quote}

A patent expires 20 years following the day it was filed.

\end{frame}



\begin{frame}
\frametitle{That's New...}

An invention is not patentable if it has either been disclosed or is known to the public.

In Canada and the United States, \alert{relative novelty} allows the invention to be disclosed or known for no more one year prior to filing.

In Europe, \alert{absolute novelty} does not allow any form of novelty.

\end{frame}



\begin{frame}
\frametitle{Space, the Final Frontier...}

In a 1945 letter, Arthur C. Clarke  gave such a complete  description of a  geostationary telecommunications  satellite that when Bell Labs
tried to patent the idea in 1954, it was refused.

\end{frame}



\begin{frame}
\frametitle{``Is It Useful?''}

The test for an invention is that it must be novel (new), non-obvious and it must have utility (be useful).

In  \textit{Apotex Inc. v. Sanofi-Synthelabo Canada Inc.}, 2008, the Supreme Court upheld the test for non-obviousness.

\end{frame}



\begin{frame}
\frametitle{``Obviously!!!''}

The test is:

\begin{enumerate}
\item Identify the notional ``person skilled in the art'' and identify the relevant common general knowledge of that person;
\item Identify the inventive concept of the claim in question or if that cannot readily be done, construe it;
\item Identify what, if any, differences exist between the matter cited as forming part of the ``state of the art'' and the inventive concept of the claim or the claim as construed;
\item Viewed without any knowledge of the alleged invention as claimed, do those differences constitute steps which would have been obvious to the person skilled in the art or do they require any degree of invention?
\end{enumerate}

\end{frame}



\begin{frame}
\frametitle{Software Patents}

Issues with software patents have  been around for years -- the issue is related to a more fundamental question:

What is invented and what is discovered?

Are computer programs invented or discovered?

Is mathematics invented or discovered?

\end{frame}



\begin{frame}
\frametitle{Software Patents}

Software is a collection of algorithms...

Consequently, the question is, is an algorithm mathematics or not?

Can you patent an algorithm?
\end{frame}



\begin{frame}
\frametitle{``This for loop is the property of XYZ Corp...''}

The argument against:

\begin{itemize}
	\item A program is a transcription of an algorithm in a programming language.
	\item Every programming language can be expressed in terms of Church's lambda calculus.
	\item Thus, every program is a transcription of a mathematical expression.
\end{itemize}

\end{frame}



\begin{frame}
\frametitle{Software Patents}

Other issues:
\begin{itemize}
	\item There is no reasonable means of searching software patents.
	\item Failure to enforce it does not prevent them from enforcing it later and suing those who profited from using the patented algorithm.
	\item TThe holder need not make any legitimate attempt to licence the algorithm to others, but can sue programmers using the algorithm later!
\end{itemize}

\end{frame}



\begin{frame}
\frametitle{Counterpoint}

Is not the \textit{non-obvious} combination of mathematical algorithms, with a goal of performing something useful, possibly considered an invention?

\end{frame}



\begin{frame}
\frametitle{It Only Takes One Judge...}

In the United States, the floodgates opened in in 1994. 

It was determined that a mathematical algorithm for anti-aliasing integrated into a digital oscillator satisfied the physical requirement component.

Unfortunately, this led to a vast influx of software patents, many of which were dubious and often had prior art.

\end{frame}



\begin{frame}
\frametitle{RSA Encryption}

The RSA encryption algorithm was patented in the USA with relative novelty.

The algorithm was published in a paper in September 1977, three months prior to the filing in December 1977.


This made patenting in all countries using absolute novelty not possible...

Oddly enough, Clifford Cooks had developed the algorithm in 1973, but it was classified until 1997.

\end{frame}



\begin{frame}
\frametitle{Software Patents}

In 2011, there was one development related to a method of detecting credit card fraud.

The court ruled you cannot patent what amounts to a mental process -- even if programmed on a computer.

This rule only applies, however, if the mathematics is sufficiently complex that the use of a computer is required.

The patent itself was absurd: any method of detecting fraud that used recorded IP addresses would have infringed on this patent...


\end{frame}

\begin{frame}
\frametitle{References \& Disclaimer}
\bibliographystyle{alphaurl}
\setbeamertemplate{bibliography item}{\insertbiblabel}
{\scriptsize
\bibliography{290}
}
\vfill

{\tiny Disclaimer: the material presented in these lectures slides is intended for use in the course ECE~290 at the University of Waterloo and should not be relied upon as legal advice. Any reliance on these course slides by any party for any other purpose are the responsibility of such parties.  The author(s) accept(s) no responsibility for damages, if any, suffered by any party as a result of decisions made or actions based on these course slides for any other purpose than that for which it was intended.\par}


\end{frame}


\end{document}

