
\documentclass[letterpaper,hide notes,xcolor={table,svgnames},pdftex,10pt]{beamer}
\def\showexamples{t}


%\usepackage[svgnames]{xcolor}

%% Demo talk
%\documentclass[letterpaper,notes=show]{beamer}

\usecolortheme{crane}
\setbeamertemplate{navigation symbols}{}

\usetheme{MyPittsburgh}
%\usetheme{Frankfurt}

%\usepackage{tipa}

\usepackage{hyperref}
\usepackage{graphicx,xspace}
\usepackage[normalem]{ulem}
\usepackage{multicol}

\newcommand\SF[1]{$\bigstar$\footnote{SF: #1}}

\usepackage[default]{sourcesanspro}
\usepackage[T1]{fontenc}

\newcounter{tmpnumSlide}
\newcounter{tmpnumNote}

% old question code
%\newcommand\question[1]{{$\bigstar$ \small \onlySlide{2}{#1}}}
% \newcommand\nquestion[1]{\ifdefined \presentationonly \textcircled{?} \fi \note{\par{\Large \textbf{?}} #1}}
% \newcommand\nanswer[1]{\note{\par{\Large \textbf{A}} #1}}


 \newcommand\mnote[1]{%
   \addtocounter{tmpnumSlide}{1}
   \ifdefined\showcues {~\tiny\fbox{\arabic{tmpnumSlide}}}\fi
   \note{\setlength{\parskip}{1ex}\addtocounter{tmpnumNote}{1}\textbf{\Large \arabic{tmpnumNote}:} {#1\par}}}

\newcommand\mmnote[1]{\note{\setlength{\parskip}{1ex}#1\par}}

%\newcommand\mnote[2][]{\ifdefined\handoutwithnotes {~\tiny\fbox{#1}}\fi
% \note{\setlength{\parskip}{1ex}\textbf{\Large #1:} #2\par}}

%\newcommand\mnote[2][]{{\tiny\fbox{#1}} \note{\setlength{\parskip}{1ex}\textbf{\Large #1:} #2\par}}

\newcommand\mquestion[2]{{~\color{red}\fbox{?}}\note{\setlength{\parskip}{1ex}\par{\Large \textbf{?}} #1} \note{\setlength{\parskip}{1ex}\par{\Large \textbf{A}} #2\par}\ifdefined \presentationonly \pause \fi}

\newcommand\blackboard[1]{%
\ifdefined   \showblackboard
  {#1}
  \else {\begin{center} \fbox{\colorbox{blue!30}{%
         \begin{minipage}{.95\linewidth}%
           \hspace{\stretch{1}} Some space intentionally left blank; done at the blackboard.%
         \end{minipage}}}\end{center}}%
         \fi%
}



%\newcommand\q{\tikz \node[thick,color=black,shape=circle]{?};}
%\newcommand\q{\ifdefined \presentationonly \textcircled{?} \fi}

\usepackage{listings}
\lstset{%
  keywordstyle=\bfseries,
  aboveskip=15pt,
  belowskip=15pt,
  captionpos=b,
  identifierstyle=\ttfamily,
  escapeinside={(*@}{@*)},
  stringstyle=\ttfamiliy,
  frame=lines,
  numbers=left, basicstyle=\scriptsize, numberstyle=\tiny, stepnumber=0, numbersep=2pt}

\usepackage{siunitx}
\newcommand\sius[1]{\num[group-separator = {,}]{#1}\si{\micro\second}}
\newcommand\sims[1]{\num[group-separator = {,}]{#1}\si{\milli\second}}
\newcommand\sins[1]{\num[group-separator = {,}]{#1}\si{\nano\second}}
\sisetup{group-separator = {,}, group-digits = true}

%% -------------------- tikz --------------------
\usepackage{tikz}
\usetikzlibrary{positioning}
\usetikzlibrary{arrows,backgrounds,automata,decorations.shapes,decorations.pathmorphing,decorations.markings,decorations.text}

\tikzstyle{place}=[circle,draw=blue!50,fill=blue!20,thick, inner sep=0pt,minimum size=6mm]
\tikzstyle{transition}=[rectangle,draw=black!50,fill=black!20,thick, inner sep=0pt,minimum size=4mm]

\tikzstyle{block}=[rectangle,draw=black, thick, inner sep=5pt]
\tikzstyle{bullet}=[circle,draw=black, fill=black, thin, inner sep=2pt]

\tikzstyle{pre}=[<-,shorten <=1pt,>=stealth',semithick]
\tikzstyle{post}=[->,shorten >=1pt,>=stealth',semithick]
\tikzstyle{bi}=[<->,shorten >=1pt,shorten <=1pt, >=stealth',semithick]

\tikzstyle{mut}=[-,>=stealth',semithick]

\tikzstyle{treereset}=[dashed,->, shorten >=1pt,>=stealth',thin]

\usepackage{ifmtarg}
\usepackage{xifthen}
\makeatletter
% new counter to now which frame it is within the sequence
\newcounter{multiframecounter}
% initialize buffer for previously used frame title
\gdef\lastframetitle{\textit{undefined}}
% new environment for a multi-frame
\newenvironment{multiframe}[1][]{%
\ifthenelse{\isempty{#1}}{%
% if no frame title was set via optional parameter,
% only increase sequence counter by 1
\addtocounter{multiframecounter}{1}%
}{%
% new frame title has been provided, thus
% reset sequence counter to 1 and buffer frame title for later use
\setcounter{multiframecounter}{1}%
\gdef\lastframetitle{#1}%
}%
% start conventional frame environment and
% automatically set frame title followed by sequence counter
\begin{frame}%
\frametitle{\lastframetitle~{\normalfont(\arabic{multiframecounter})}}%
}{%
\end{frame}%
}
\makeatother

\makeatletter
\newdimen\tu@tmpa%
\newdimen\ydiffl%
\newdimen\xdiffl%
\newcommand\ydiff[2]{%
    \coordinate (tmpnamea) at (#1);%
    \coordinate (tmpnameb) at (#2);%
    \pgfextracty{\tu@tmpa}{\pgfpointanchor{tmpnamea}{center}}%
    \pgfextracty{\ydiffl}{\pgfpointanchor{tmpnameb}{center}}%
    \advance\ydiffl by -\tu@tmpa%
}
\newcommand\xdiff[2]{%
    \coordinate (tmpnamea) at (#1);%
    \coordinate (tmpnameb) at (#2);%
    \pgfextractx{\tu@tmpa}{\pgfpointanchor{tmpnamea}{center}}%
    \pgfextractx{\xdiffl}{\pgfpointanchor{tmpnameb}{center}}%
    \advance\xdiffl by -\tu@tmpa%
}
\makeatother
\newcommand{\copyrightbox}[3][r]{%
\begin{tikzpicture}%
\node[inner sep=0pt,minimum size=2em](ciimage){#2};
\usefont{OT1}{phv}{n}{n}\fontsize{4}{4}\selectfont
\ydiff{ciimage.south}{ciimage.north}
\xdiff{ciimage.west}{ciimage.east}
\ifthenelse{\equal{#1}{r}}{%
\node[inner sep=0pt,right=1ex of ciimage.south east,anchor=north west,rotate=90]%
{\raggedleft\color{black!50}\parbox{\the\ydiffl}{\raggedright{}#3}};%
}{%
\ifthenelse{\equal{#1}{l}}{%
\node[inner sep=0pt,right=1ex of ciimage.south west,anchor=south west,rotate=90]%
{\raggedleft\color{black!50}\parbox{\the\ydiffl}{\raggedright{}#3}};%
}{%
\node[inner sep=0pt,below=1ex of ciimage.south west,anchor=north west]%
{\raggedleft\color{black!50}\parbox{\the\xdiffl}{\raggedright{}#3}};%
}
}
\end{tikzpicture}
}


%% --------------------

%\usepackage[excludeor]{everyhook}
%\PushPreHook{par}{\setbox0=\lastbox\llap{MUH}}\box0}

%\vspace*{\stretch{1}

%\setbox0=\lastbox \llap{\textbullet\enskip}\box0}

\setlength{\parskip}{\fill}

\newcommand\noskips{\setlength{\parskip}{1ex}}
\newcommand\doskips{\setlength{\parskip}{\fill}}

\newcommand\xx{\par\vspace*{\stretch{1}}\par}
\newcommand\xxs{\par\vspace*{2ex}\par}
\newcommand\tuple[1]{\langle #1 \rangle}
\newcommand\code[1]{{\sf \footnotesize #1}}
\newcommand\ex[1]{\uline{Example:} \ifdefined \presentationonly \pause \fi
  \ifdefined\showexamples#1\xspace\else{\uline{\hspace*{2cm}}}\fi}

\newcommand\ceil[1]{\lceil #1 \rceil}


\AtBeginSection[]
{
   \begin{frame}
       \frametitle{Outline}
       \tableofcontents[currentsection]
   \end{frame}
}



\pgfdeclarelayer{edgelayer}
\pgfdeclarelayer{nodelayer}
\pgfsetlayers{edgelayer,nodelayer,main}

\tikzstyle{none}=[inner sep=0pt]
\tikzstyle{rn}=[circle,fill=Red,draw=Black,line width=0.8 pt]
\tikzstyle{gn}=[circle,fill=Lime,draw=Black,line width=0.8 pt]
\tikzstyle{yn}=[circle,fill=Yellow,draw=Black,line width=0.8 pt]
\tikzstyle{empty}=[circle,fill=White,draw=Black]
\tikzstyle{bw} = [rectangle, draw, fill=blue!20, 
    text width=4em, text centered, rounded corners, minimum height=2em]
    
    \newcommand{\CcNote}[1]{% longname
	This work is licensed under the \textit{Creative Commons #1 3.0 License}.%
}
\newcommand{\CcImageBy}[1]{%
	\includegraphics[scale=#1]{creative_commons/cc_by_30.pdf}%
}
\newcommand{\CcImageSa}[1]{%
	\includegraphics[scale=#1]{creative_commons/cc_sa_30.pdf}%
}
\newcommand{\CcImageNc}[1]{%
	\includegraphics[scale=#1]{creative_commons/cc_nc_30.pdf}%
}
\newcommand{\CcGroupBySa}[2]{% zoom, gap
	\CcImageBy{#1}\hspace*{#2}\CcImageNc{#1}\hspace*{#2}\CcImageSa{#1}%
}
\newcommand{\CcLongnameByNcSa}{Attribution-NonCommercial-ShareAlike}

\newenvironment{changemargin}[1]{% 
  \begin{list}{}{% 
    \setlength{\topsep}{0pt}% 
    \setlength{\leftmargin}{#1}% 
    \setlength{\rightmargin}{1em}
    \setlength{\listparindent}{\parindent}% 
    \setlength{\itemindent}{\parindent}% 
    \setlength{\parsep}{\parskip}% 
  }% 
  \item[]}{\end{list}} 




\title{Lecture 33 --- PEO Code of Ethics }

\author{Jeff Zarnett, based on original by Douglas Harder \\ \small \texttt{jzarnett@uwaterloo.ca} / \texttt{dwharder@uwaterloo.ca}}
\institute{Department of Electrical and Computer Engineering \\
  University of Waterloo}
\date{\today}


\begin{document}

\begin{frame}
  \titlepage

\begin{center}
  \small{Acknowledgments: Douglas Harder~\cite{dwh}, Julie Vale~\cite{jv}}
  \end{center}
\end{frame}



\begin{frame}
\frametitle{Enforceable Codes of Ethics}


In most jurisdictions, the Code of Ethics is enforceable.\\
\quad A violation of the code of ethics is professional misconduct.


In Ontario, the Code of Ethics is not enforceable.\\
\quad Instead, it lists items considered to be best practices.

Professional misconduct is defined in section 72 of the regulations.

\end{frame}




\begin{frame}
\frametitle{Always Seven, There Are}

The Code of Ethics focuses on seven areas (most of which are about interactions with others):

\begin{itemize}
	\item Personal conduct.
	\item Standards of behaviour.
	\item Employers.
	\item Clients.
	\item Other professionals.
	\item Other practitioners of engineering.
	\item The profession of engineering itself.
\end{itemize}

The code of ethics is written into the Regulations (section 77)

\end{frame}



\begin{frame}
\frametitle{Personal Conduct}

1. It is the duty of a practitioner to the public, to the practitioner's employer, to the practitioner's clients, to other members of the practitioner's profession, and to the practitioner to act at all times with,

\begin{enumerate}[i)]
\item fairness and loyalty to the practitioner's associates, employer, clients, subordinates and employees,

\item fidelity to public needs,

\item devotion to high ideals of personal honour and professional integrity,

\item knowledge of developments in the area of professional engineering relevant to any services 1that are undertaken, and

\item competence in the performance of any professional engineering services that are undertaken.
\end{enumerate}


\end{frame}



\begin{frame}
\frametitle{Standards of Behaviour}

2. A practitioner shall,

\begin{enumerate}[i)]
\item regard the practitioner's duty to public welfare as paramount,

\item endeavour at all times to enhance the public regard for the practitioner's profession by extending the public knowledge thereof and discouraging untrue, unfair or exaggerated statements with respect to professional engineering,

\item not express publicly, or while the practitioner is serving as a witness before a court, commission or other tribunal, opinions on professional engineering matters that are not founded on adequate knowledge and honest conviction,

\item endeavour to keep the practitioner's licence, temporary licence, provisional licence, limited licence or certificate of authorization, as the case may be, permanently displayed in the practitioner's place of business.
\end{enumerate}

\end{frame}



\begin{frame}
\frametitle{Employers}

3. A practitioner shall act in professional engineering matters for the practitioner's employer as a faithful agent or trustee and shall regard as confidential information obtained by the practitioner as to the business affairs, technical methods or processes of an employer and avoid or disclose a conflict of interest that might influence the practitioner's actions or judgment.


\end{frame}



\begin{frame}
\frametitle{Clients}

4. A practitioner must disclose immediately to the practitioner's client any interest, direct or indirect, that might be construed as prejudicial in any way to the professional judgment of the practitioner in rendering service to the client.

\end{frame}



\begin{frame}
\frametitle{Employers \& Clients}

It is possible for an engineer to act as both an employee while at the same time soliciting contracts from clients.


The work for the clients must, of course, be entirely independent of the work for the employer.

An employee engineer does not require a Certificate of Authorization, but soliciting clients does.

\end{frame}



\begin{frame}
\frametitle{Employers \& Clients}

5. A practitioner who is an employee-engineer and is contracting in the practitioner's own name to perform professional engineering work for other than the practitioner's employer, must provide the practitioner's client with a written statement of the nature of the practitioner's status as an employee and the attendant limitations on the practitioner's services to the client, must satisfy the practitioner that the work will not conflict with the practitioner's duty to the practitioner's employer, and must inform the practitioner's employer of the work.


(This is sometimes referred to as the ``Moonlighting Clause''.)

\end{frame}




\begin{frame}
\frametitle{Other Professionals}

6. A practitioner must co-operate in working with other professionals engaged on a project.


\end{frame}



\begin{frame}
\frametitle{Other Practitioners}

7. A practitioner shall,

\begin{enumerate}[i)]
\item act towards other practitioners with courtesy and good faith,

\item  not accept an engagement to review the work of another practitioner for the same employer except with the knowledge of the other practitioner or except where the connection of the other practitioner with the work has been terminated,

\item  not maliciously injure the reputation or business of another practitioner,

\item  not attempt to gain an advantage over other practitioners by paying or accepting a commission in securing professional engineering work, and

\item  give proper credit for engineering work, uphold the principle of adequate compensation for engineering work, provide opportunity for professional development and advancement of the practitioner's associates and subordinates, and extend the effectiveness of the profession through the interchange of engineering information and experience.
\end{enumerate}


\end{frame}



\begin{frame}
\frametitle{The Profession}

8. A practitioner shall maintain the honour and integrity of the practitioner's profession and without fear or favour expose before the proper tribunals unprofessional, dishonest or unethical conduct by any other practitioner. 

\end{frame}



\begin{frame}
\frametitle{The Code of Ethics}

Recall that the Code of Ethics is deontological in nature

Engineers, in general, prefer deontological statements. 

As long as the rules are followed, one is not behaving unethically -- that is, one is not engaging in professional misconduct.

Clear rules are appreciated, but unfortunately, human affairs are messy.


\end{frame}



\begin{frame}
\frametitle{It Gets Complicated}

Consider however the issue with utilitarianism: It is very close to ethics engineering.

\begin{itemize}
\item What are the requirements for a successful outcome?
\item What are the metrics in this particular case?
\item What are the criteria by which we will consider a reasonable action?
\item What are the available options?
\item Are there examples from which experience can be drawn from?
\end{itemize}

In some cases, there may be no ethical action. What then?

\end{frame}



\begin{frame}
\frametitle{Other Codes of Ethics}

In many jurisdictions, not following the Code of Ethics is by definition professional misconduct.


For example, the Alberta Code of Ethics is such a code.

\end{frame}



\begin{frame}
\frametitle{Alberta}

\textbf{CODE OF ETHICS}\\
\quad Established pursuant to section 20(1)(k) of the Engineering, Geological and Geophysical Professions Act,

\textbf{Preamble}\\
1. Professional engineers, geologists and geophysicists shall recognize that professional ethics is founded upon integrity, competence, dignity and devotion to service. This concept shall guide their conduct at all times.

2. Practitioners shall, in their areas of practice, hold paramount the health, safety and welfare of the public and have regard for the environment.


\end{frame}




\begin{frame}
\frametitle{Alberta}

3. Practitioners shall undertake only work that they are competent to perform by virtue of their training and experience.

4. Practitioners shall conduct themselves with integrity, honesty, fairness and objectivity in their professional activities.

5. Practitioners shall comply with applicable statutes, regulations and bylaws in their professional practices.

6. Practitioners shall uphold and enhance the honour, dignity and reputation of their professions and thus the ability of the professions to serve the public interest.

\end{frame}

\begin{frame}
\frametitle{References \& Disclaimer}
\bibliographystyle{alphaurl}
\setbeamertemplate{bibliography item}{\insertbiblabel}
{\scriptsize
\bibliography{290}
}
\vfill

{\tiny Disclaimer: the material presented in these lectures slides is intended for use in the course ECE~290 at the University of Waterloo and should not be relied upon as legal advice. Any reliance on these course slides by any party for any other purpose are the responsibility of such parties.  The author(s) accept(s) no responsibility for damages, if any, suffered by any party as a result of decisions made or actions based on these course slides for any other purpose than that for which it was intended.\par}


\end{frame}


\end{document}

