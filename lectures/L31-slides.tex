
\documentclass[letterpaper,hide notes,xcolor={table,svgnames},pdftex,10pt]{beamer}
\def\showexamples{t}


%\usepackage[svgnames]{xcolor}

%% Demo talk
%\documentclass[letterpaper,notes=show]{beamer}

\usecolortheme{crane}
\setbeamertemplate{navigation symbols}{}

\usetheme{MyPittsburgh}
%\usetheme{Frankfurt}

%\usepackage{tipa}

\usepackage{hyperref}
\usepackage{graphicx,xspace}
\usepackage[normalem]{ulem}
\usepackage{multicol}

\newcommand\SF[1]{$\bigstar$\footnote{SF: #1}}

\usepackage[default]{sourcesanspro}
\usepackage[T1]{fontenc}

\newcounter{tmpnumSlide}
\newcounter{tmpnumNote}

% old question code
%\newcommand\question[1]{{$\bigstar$ \small \onlySlide{2}{#1}}}
% \newcommand\nquestion[1]{\ifdefined \presentationonly \textcircled{?} \fi \note{\par{\Large \textbf{?}} #1}}
% \newcommand\nanswer[1]{\note{\par{\Large \textbf{A}} #1}}


 \newcommand\mnote[1]{%
   \addtocounter{tmpnumSlide}{1}
   \ifdefined\showcues {~\tiny\fbox{\arabic{tmpnumSlide}}}\fi
   \note{\setlength{\parskip}{1ex}\addtocounter{tmpnumNote}{1}\textbf{\Large \arabic{tmpnumNote}:} {#1\par}}}

\newcommand\mmnote[1]{\note{\setlength{\parskip}{1ex}#1\par}}

%\newcommand\mnote[2][]{\ifdefined\handoutwithnotes {~\tiny\fbox{#1}}\fi
% \note{\setlength{\parskip}{1ex}\textbf{\Large #1:} #2\par}}

%\newcommand\mnote[2][]{{\tiny\fbox{#1}} \note{\setlength{\parskip}{1ex}\textbf{\Large #1:} #2\par}}

\newcommand\mquestion[2]{{~\color{red}\fbox{?}}\note{\setlength{\parskip}{1ex}\par{\Large \textbf{?}} #1} \note{\setlength{\parskip}{1ex}\par{\Large \textbf{A}} #2\par}\ifdefined \presentationonly \pause \fi}

\newcommand\blackboard[1]{%
\ifdefined   \showblackboard
  {#1}
  \else {\begin{center} \fbox{\colorbox{blue!30}{%
         \begin{minipage}{.95\linewidth}%
           \hspace{\stretch{1}} Some space intentionally left blank; done at the blackboard.%
         \end{minipage}}}\end{center}}%
         \fi%
}



%\newcommand\q{\tikz \node[thick,color=black,shape=circle]{?};}
%\newcommand\q{\ifdefined \presentationonly \textcircled{?} \fi}

\usepackage{listings}
\lstset{%
  keywordstyle=\bfseries,
  aboveskip=15pt,
  belowskip=15pt,
  captionpos=b,
  identifierstyle=\ttfamily,
  escapeinside={(*@}{@*)},
  stringstyle=\ttfamiliy,
  frame=lines,
  numbers=left, basicstyle=\scriptsize, numberstyle=\tiny, stepnumber=0, numbersep=2pt}

\usepackage{siunitx}
\newcommand\sius[1]{\num[group-separator = {,}]{#1}\si{\micro\second}}
\newcommand\sims[1]{\num[group-separator = {,}]{#1}\si{\milli\second}}
\newcommand\sins[1]{\num[group-separator = {,}]{#1}\si{\nano\second}}
\sisetup{group-separator = {,}, group-digits = true}

%% -------------------- tikz --------------------
\usepackage{tikz}
\usetikzlibrary{positioning}
\usetikzlibrary{arrows,backgrounds,automata,decorations.shapes,decorations.pathmorphing,decorations.markings,decorations.text}

\tikzstyle{place}=[circle,draw=blue!50,fill=blue!20,thick, inner sep=0pt,minimum size=6mm]
\tikzstyle{transition}=[rectangle,draw=black!50,fill=black!20,thick, inner sep=0pt,minimum size=4mm]

\tikzstyle{block}=[rectangle,draw=black, thick, inner sep=5pt]
\tikzstyle{bullet}=[circle,draw=black, fill=black, thin, inner sep=2pt]

\tikzstyle{pre}=[<-,shorten <=1pt,>=stealth',semithick]
\tikzstyle{post}=[->,shorten >=1pt,>=stealth',semithick]
\tikzstyle{bi}=[<->,shorten >=1pt,shorten <=1pt, >=stealth',semithick]

\tikzstyle{mut}=[-,>=stealth',semithick]

\tikzstyle{treereset}=[dashed,->, shorten >=1pt,>=stealth',thin]

\usepackage{ifmtarg}
\usepackage{xifthen}
\makeatletter
% new counter to now which frame it is within the sequence
\newcounter{multiframecounter}
% initialize buffer for previously used frame title
\gdef\lastframetitle{\textit{undefined}}
% new environment for a multi-frame
\newenvironment{multiframe}[1][]{%
\ifthenelse{\isempty{#1}}{%
% if no frame title was set via optional parameter,
% only increase sequence counter by 1
\addtocounter{multiframecounter}{1}%
}{%
% new frame title has been provided, thus
% reset sequence counter to 1 and buffer frame title for later use
\setcounter{multiframecounter}{1}%
\gdef\lastframetitle{#1}%
}%
% start conventional frame environment and
% automatically set frame title followed by sequence counter
\begin{frame}%
\frametitle{\lastframetitle~{\normalfont(\arabic{multiframecounter})}}%
}{%
\end{frame}%
}
\makeatother

\makeatletter
\newdimen\tu@tmpa%
\newdimen\ydiffl%
\newdimen\xdiffl%
\newcommand\ydiff[2]{%
    \coordinate (tmpnamea) at (#1);%
    \coordinate (tmpnameb) at (#2);%
    \pgfextracty{\tu@tmpa}{\pgfpointanchor{tmpnamea}{center}}%
    \pgfextracty{\ydiffl}{\pgfpointanchor{tmpnameb}{center}}%
    \advance\ydiffl by -\tu@tmpa%
}
\newcommand\xdiff[2]{%
    \coordinate (tmpnamea) at (#1);%
    \coordinate (tmpnameb) at (#2);%
    \pgfextractx{\tu@tmpa}{\pgfpointanchor{tmpnamea}{center}}%
    \pgfextractx{\xdiffl}{\pgfpointanchor{tmpnameb}{center}}%
    \advance\xdiffl by -\tu@tmpa%
}
\makeatother
\newcommand{\copyrightbox}[3][r]{%
\begin{tikzpicture}%
\node[inner sep=0pt,minimum size=2em](ciimage){#2};
\usefont{OT1}{phv}{n}{n}\fontsize{4}{4}\selectfont
\ydiff{ciimage.south}{ciimage.north}
\xdiff{ciimage.west}{ciimage.east}
\ifthenelse{\equal{#1}{r}}{%
\node[inner sep=0pt,right=1ex of ciimage.south east,anchor=north west,rotate=90]%
{\raggedleft\color{black!50}\parbox{\the\ydiffl}{\raggedright{}#3}};%
}{%
\ifthenelse{\equal{#1}{l}}{%
\node[inner sep=0pt,right=1ex of ciimage.south west,anchor=south west,rotate=90]%
{\raggedleft\color{black!50}\parbox{\the\ydiffl}{\raggedright{}#3}};%
}{%
\node[inner sep=0pt,below=1ex of ciimage.south west,anchor=north west]%
{\raggedleft\color{black!50}\parbox{\the\xdiffl}{\raggedright{}#3}};%
}
}
\end{tikzpicture}
}


%% --------------------

%\usepackage[excludeor]{everyhook}
%\PushPreHook{par}{\setbox0=\lastbox\llap{MUH}}\box0}

%\vspace*{\stretch{1}

%\setbox0=\lastbox \llap{\textbullet\enskip}\box0}

\setlength{\parskip}{\fill}

\newcommand\noskips{\setlength{\parskip}{1ex}}
\newcommand\doskips{\setlength{\parskip}{\fill}}

\newcommand\xx{\par\vspace*{\stretch{1}}\par}
\newcommand\xxs{\par\vspace*{2ex}\par}
\newcommand\tuple[1]{\langle #1 \rangle}
\newcommand\code[1]{{\sf \footnotesize #1}}
\newcommand\ex[1]{\uline{Example:} \ifdefined \presentationonly \pause \fi
  \ifdefined\showexamples#1\xspace\else{\uline{\hspace*{2cm}}}\fi}

\newcommand\ceil[1]{\lceil #1 \rceil}


\AtBeginSection[]
{
   \begin{frame}
       \frametitle{Outline}
       \tableofcontents[currentsection]
   \end{frame}
}



\pgfdeclarelayer{edgelayer}
\pgfdeclarelayer{nodelayer}
\pgfsetlayers{edgelayer,nodelayer,main}

\tikzstyle{none}=[inner sep=0pt]
\tikzstyle{rn}=[circle,fill=Red,draw=Black,line width=0.8 pt]
\tikzstyle{gn}=[circle,fill=Lime,draw=Black,line width=0.8 pt]
\tikzstyle{yn}=[circle,fill=Yellow,draw=Black,line width=0.8 pt]
\tikzstyle{empty}=[circle,fill=White,draw=Black]
\tikzstyle{bw} = [rectangle, draw, fill=blue!20, 
    text width=4em, text centered, rounded corners, minimum height=2em]
    
    \newcommand{\CcNote}[1]{% longname
	This work is licensed under the \textit{Creative Commons #1 3.0 License}.%
}
\newcommand{\CcImageBy}[1]{%
	\includegraphics[scale=#1]{creative_commons/cc_by_30.pdf}%
}
\newcommand{\CcImageSa}[1]{%
	\includegraphics[scale=#1]{creative_commons/cc_sa_30.pdf}%
}
\newcommand{\CcImageNc}[1]{%
	\includegraphics[scale=#1]{creative_commons/cc_nc_30.pdf}%
}
\newcommand{\CcGroupBySa}[2]{% zoom, gap
	\CcImageBy{#1}\hspace*{#2}\CcImageNc{#1}\hspace*{#2}\CcImageSa{#1}%
}
\newcommand{\CcLongnameByNcSa}{Attribution-NonCommercial-ShareAlike}

\newenvironment{changemargin}[1]{% 
  \begin{list}{}{% 
    \setlength{\topsep}{0pt}% 
    \setlength{\leftmargin}{#1}% 
    \setlength{\rightmargin}{1em}
    \setlength{\listparindent}{\parindent}% 
    \setlength{\itemindent}{\parindent}% 
    \setlength{\parsep}{\parskip}% 
  }% 
  \item[]}{\end{list}} 




\title{Lecture 31 --- Ethics: Modern Ethical Theories }

\author{Jeff Zarnett \\ \small \texttt{jzarnett@uwaterloo.ca}}
\institute{Department of Electrical and Computer Engineering \\
  University of Waterloo}
\date{\today}


\begin{document}

\begin{frame}
  \titlepage

\begin{center}
  \small{Acknowledgments: Douglas Harder~\cite{dwh}, Julie Vale~\cite{jv}}
  \end{center}
\end{frame}



\begin{frame}
\frametitle{Agree to Disagree}

In this course you will be exposed to different ethical theories and different points of view. So far we have already seen the opinions of others...

You may agree with some of the theories under discussion, disagree with others...

You may nevertheless be required to answer questions based on these theories, regardless of your personal beliefs. 

This is not a natural science where there are correct answers (``the world is round, not flat''); reasonable people can agree to disagree civilly.


\end{frame}



\begin{frame}
\frametitle{With Support}

Therefore, we must understand that you know how to make an ethical decision by making a reasoned decision based on underlying assumptions.

Your opinion in class in invaluable -- it will make for fascinating discussions.


Your opinion on an examination, without a logical support, is worth nothing.

``Because I said so''-type arguments and ``I feel like this is true'' are not adequate reasons to support the assertions.

\end{frame}



\begin{frame}
\frametitle{Deontology}

\alert{Deontology} is the science of duty; the branch of knowledge that deals with moral obligations; ethics.

Deontology determines the morality of an action, behaviour, character, or desire based on is adherence to a rule or rules.


It is also known as duty-based ethics.


\end{frame}



\begin{frame}
\frametitle{Deontology}

Deontology was the first modern ethical theory. 

It is based heavily on concepts such as moral absolutism \& divine command theory.

It posits that there are rules or duties that must be followed and that by fulfilling one�s duties or following these rules one is ethical.

\end{frame}



\begin{frame}
\frametitle{Deontology}

Immanuel Kant (1724-1804) argued an action is good only if it is the consequence of duty as opposed to desire or need.

He also argued that good intentions, rather than good outcomes, are what make an act moral.

After all, bad things may happen by accident, but intent is important.

Kant saw a need for universal principles that guided human behaviour and conscience.

This led him to introduce the idea of a \alert{categorical imperative}.

\end{frame}



\begin{frame}
\frametitle{Deontology}

An action can only be moral if it is a consequence of following a categorical imperative (i.e, a universally applicable principle).

\begin{itemize}
\item It is performed out of a sense of moral duty.
\item Any maxim in harmony with the categorical imperative must be universally applicable to all beings with free will.
\item All actions that are performed according to a maxim that cannot be applied universally do not have moral worth.
\end{itemize}

\end{frame}



\begin{frame}
\frametitle{Deontology}

Kant formulated three maxims:

\begin{enumerate}
\item Act only on the maxim whereby thou canst at the same time will that it should become a universal law.

\item So act as to treat humanity, whether in thine own person or in that of any other, in every case as an end withal, never as means only.

\item A rational being must always regard himself as giving laws either as member or as sovereign in a kingdom of ends which is rendered possible by the freedom of will.
\end{enumerate}

\end{frame}

\begin{frame}
\frametitle{Deontology}

One possible way that this duty based morality can be abused is ``I was only following orders''.

\end{frame}


\begin{frame}
\frametitle{Deontology}

Deontology has continued to evolve as a theory of normative ethics.

	Frances Kamm of Harvard continues to use this approach.

But first some background...

\end{frame}



\begin{frame}
\frametitle{Necessity and Sufficiency}

A little formal logic for you.

A condition $a$ is sufficient for $z$ if $a \rightarrow z$.

But it it also possible that $b \rightarrow z$, $c \rightarrow z$.

Example: Event $z$ is you have enough money to go to the concert. 

This could be caused by:

\begin{itemize}
	\item Your parents give you money ($a$)
	\item You find money on the street ($b$)
	\item You win the lottery ($c$)
	\item etc
\end{itemize}

So $ a \vee b \vee c \vee ... \rightarrow z$

\end{frame}



\begin{frame}
\frametitle{Necessity and Sufficiency}

There is also the contrapositive: $\neg z \rightarrow \neg a \wedge \neg b \wedge \neg c...$

If you don't have the money, you didn't get it from your parents, or find it on the street, or win the lottery, or...


\end{frame}



\begin{frame}
\frametitle{Necessity and Sufficiency}

A condition $a$ is necessary for $z$ if $\neg a \rightarrow \neg z$.

But it is also possible that $\neg b \rightarrow \neg z$...\\
\quad $\neg a \vee \neg b \vee \neg c ... \rightarrow \neg z$

Consider getting your engineering license as $z$.

All of these elements are necessary:

\begin{itemize}
	\item Engineering education ($a$)
	\item Work experience ($b$)
	\item Completion of the PPE ($c$)
\end{itemize}


The absence of any one of those necessary elements means the license will not be granted.

The contrapositive: $z \rightarrow a \wedge b \wedge c$

\end{frame}



\begin{frame}
\frametitle{Necessity and Sufficiency}

If both $a \rightarrow z$ and $\neg a \rightarrow \neg z$ then it follows that $z \rightarrow a$ and we say $a$ is necessary and sufficient for $z$.

We can also say $a$ if and only if $z$ (or $a$ iff $b$).

\end{frame}



\begin{frame}
\frametitle{Frances Kamm}

Frances Kamm introduced the The Principle of Permissible Harm.

One may harm in order to save more if and only if the harm is an effect or an aspect of the greater good itself.

Evil is contrasted with wrong or harm as the involvement of persons without their consent when foreseeably this will lead to a wrong or harm to them.

\end{frame}



\begin{frame}
\frametitle{Frances Kamm}

\textbf{The Doctrine of Productive Purity}

1. If an evil cannot be at least initially sufficiently justified, it cannot be justified by the greater good that it is necessary (given our act) to causally produce.� However, such an evil can be justified by the greater good whose component(s) cause it, even if the evil is causally necessary to help sustain the greater good or its components.


\end{frame}

\begin{frame}
\frametitle{Frances Kamm}

\textbf{The Doctrine of Productive Purity}

2. In order for an act to be permissible, it should be possible for any evil side effect (except possibly indirect side effects) of what we do, or evil causal means that we must use (given our act) to bring about the greater good, to be at least the effect of a [greater good that] is working itself out (or the effect of means that are noncausally related to that greater good that is working itself out).

\end{frame}



\begin{frame}
\frametitle{Consequentialism}


After Immanuel Kant described his ethical philosophies, others began using a different approach: they focused on the consequences of the action.

It is the consequences that will form the basis of any judgement about the ethical correctness of the action.

A morally right act or omission is one that has good consequences.

Such ethical theories are \alert{consequentialist} or \alert{teleological}.

\end{frame}



\begin{frame}
\frametitle{Consequentialism}

How do we define consequences?

The same outcome may be viewed as positive by one person and negative by another.

There may be vastly different short and long term consequences...

\end{frame}



\begin{frame}
\frametitle{State Consequentialism}

Consider M\`oz\v{i}, a Chinese philosopher from the 5th century BCE


He introduced the idea of state consequentialism:


\begin{quote}
It is the business of the benevolent man to seek to promote what is beneficial to the world and to eliminate what is harmful, and to provide a model for the world. What benefits he will carry out; what does not benefit men he will leave alone.
\end{quote}

He argued against the role ethics of Confucius.


\end{frame}



\begin{frame}
\frametitle{Utilitarianism}

More recently, Jeremy Bentham (1748 - 1832) introduced the concept of \alert{utilitarianism}.

We must develop a measure of the \alert{utility} of the consequences of an action.

A means of quantitatively, or at least qualitatively, determining whether consequences are ``good'' (and which courses of action are better).

He advocated for animal rights, gender equality, and acceptance of homosexuality (all ideas very much ahead of the times in which he lived!)

\end{frame}



\begin{frame}
\frametitle{Utilitarianism}

Bentham had a great deal of ideas that were ahead of his time:

\begin{itemize}
	\item Separation of church and state
	\item Freedom of expression
	\item Equal rights for women
	\item Animal rights
	\item The right to divorce
	\item Decriminalizing homosexuality
	\item Abolition of slavery
	\item Abolition of the death penalty
	\item Individual rights
\end{itemize}

\end{frame}



\begin{frame}
\frametitle{Let's Make It a Math Problem}

Bentham introduced the concept of \textit{felicific calculus} (math is fun!)

The happiness of a consequence can be judged by:

\begin{itemize}
	\item Intensity
	\item Duration
	\item Certainty
	\item Proximity
	\item Fecundity
	\item Purity
	\item Extent
\end{itemize}

In simple terms, the greatest increase in good for the greatest number.

\end{frame}



\begin{frame}
\frametitle{The Final Frontier}

``The needs of the many outweigh the needs of the few.''

``Or the one.''


\end{frame}



\begin{frame}
\frametitle{Utilitarianism}

One way that utilitarianism can be simplified, incorrectly, is the argument that ``the ends justify the means''.


\end{frame}


\begin{frame}
\frametitle{References \& Disclaimer}
\bibliographystyle{alphaurl}
\setbeamertemplate{bibliography item}{\insertbiblabel}
{\scriptsize
\bibliography{290}
}
\vfill

{\tiny Disclaimer: the material presented in these lectures slides is intended for use in the course ECE~290 at the University of Waterloo and should not be relied upon as legal advice. Any reliance on these course slides by any party for any other purpose are the responsibility of such parties.  The author(s) accept(s) no responsibility for damages, if any, suffered by any party as a result of decisions made or actions based on these course slides for any other purpose than that for which it was intended.\par}


\end{frame}


\end{document}

