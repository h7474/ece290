
\documentclass[letterpaper,hide notes,xcolor={table,svgnames},pdftex,10pt]{beamer}
\def\showexamples{t}


%\usepackage[svgnames]{xcolor}

%% Demo talk
%\documentclass[letterpaper,notes=show]{beamer}

\usecolortheme{crane}
\setbeamertemplate{navigation symbols}{}

\usetheme{MyPittsburgh}
%\usetheme{Frankfurt}

%\usepackage{tipa}

\usepackage{hyperref}
\usepackage{graphicx,xspace}
\usepackage[normalem]{ulem}
\usepackage{multicol}

\newcommand\SF[1]{$\bigstar$\footnote{SF: #1}}

\usepackage[default]{sourcesanspro}
\usepackage[T1]{fontenc}

\newcounter{tmpnumSlide}
\newcounter{tmpnumNote}

% old question code
%\newcommand\question[1]{{$\bigstar$ \small \onlySlide{2}{#1}}}
% \newcommand\nquestion[1]{\ifdefined \presentationonly \textcircled{?} \fi \note{\par{\Large \textbf{?}} #1}}
% \newcommand\nanswer[1]{\note{\par{\Large \textbf{A}} #1}}


 \newcommand\mnote[1]{%
   \addtocounter{tmpnumSlide}{1}
   \ifdefined\showcues {~\tiny\fbox{\arabic{tmpnumSlide}}}\fi
   \note{\setlength{\parskip}{1ex}\addtocounter{tmpnumNote}{1}\textbf{\Large \arabic{tmpnumNote}:} {#1\par}}}

\newcommand\mmnote[1]{\note{\setlength{\parskip}{1ex}#1\par}}

%\newcommand\mnote[2][]{\ifdefined\handoutwithnotes {~\tiny\fbox{#1}}\fi
% \note{\setlength{\parskip}{1ex}\textbf{\Large #1:} #2\par}}

%\newcommand\mnote[2][]{{\tiny\fbox{#1}} \note{\setlength{\parskip}{1ex}\textbf{\Large #1:} #2\par}}

\newcommand\mquestion[2]{{~\color{red}\fbox{?}}\note{\setlength{\parskip}{1ex}\par{\Large \textbf{?}} #1} \note{\setlength{\parskip}{1ex}\par{\Large \textbf{A}} #2\par}\ifdefined \presentationonly \pause \fi}

\newcommand\blackboard[1]{%
\ifdefined   \showblackboard
  {#1}
  \else {\begin{center} \fbox{\colorbox{blue!30}{%
         \begin{minipage}{.95\linewidth}%
           \hspace{\stretch{1}} Some space intentionally left blank; done at the blackboard.%
         \end{minipage}}}\end{center}}%
         \fi%
}



%\newcommand\q{\tikz \node[thick,color=black,shape=circle]{?};}
%\newcommand\q{\ifdefined \presentationonly \textcircled{?} \fi}

\usepackage{listings}
\lstset{%
  keywordstyle=\bfseries,
  aboveskip=15pt,
  belowskip=15pt,
  captionpos=b,
  identifierstyle=\ttfamily,
  escapeinside={(*@}{@*)},
  stringstyle=\ttfamiliy,
  frame=lines,
  numbers=left, basicstyle=\scriptsize, numberstyle=\tiny, stepnumber=0, numbersep=2pt}

\usepackage{siunitx}
\newcommand\sius[1]{\num[group-separator = {,}]{#1}\si{\micro\second}}
\newcommand\sims[1]{\num[group-separator = {,}]{#1}\si{\milli\second}}
\newcommand\sins[1]{\num[group-separator = {,}]{#1}\si{\nano\second}}
\sisetup{group-separator = {,}, group-digits = true}

%% -------------------- tikz --------------------
\usepackage{tikz}
\usetikzlibrary{positioning}
\usetikzlibrary{arrows,backgrounds,automata,decorations.shapes,decorations.pathmorphing,decorations.markings,decorations.text}

\tikzstyle{place}=[circle,draw=blue!50,fill=blue!20,thick, inner sep=0pt,minimum size=6mm]
\tikzstyle{transition}=[rectangle,draw=black!50,fill=black!20,thick, inner sep=0pt,minimum size=4mm]

\tikzstyle{block}=[rectangle,draw=black, thick, inner sep=5pt]
\tikzstyle{bullet}=[circle,draw=black, fill=black, thin, inner sep=2pt]

\tikzstyle{pre}=[<-,shorten <=1pt,>=stealth',semithick]
\tikzstyle{post}=[->,shorten >=1pt,>=stealth',semithick]
\tikzstyle{bi}=[<->,shorten >=1pt,shorten <=1pt, >=stealth',semithick]

\tikzstyle{mut}=[-,>=stealth',semithick]

\tikzstyle{treereset}=[dashed,->, shorten >=1pt,>=stealth',thin]

\usepackage{ifmtarg}
\usepackage{xifthen}
\makeatletter
% new counter to now which frame it is within the sequence
\newcounter{multiframecounter}
% initialize buffer for previously used frame title
\gdef\lastframetitle{\textit{undefined}}
% new environment for a multi-frame
\newenvironment{multiframe}[1][]{%
\ifthenelse{\isempty{#1}}{%
% if no frame title was set via optional parameter,
% only increase sequence counter by 1
\addtocounter{multiframecounter}{1}%
}{%
% new frame title has been provided, thus
% reset sequence counter to 1 and buffer frame title for later use
\setcounter{multiframecounter}{1}%
\gdef\lastframetitle{#1}%
}%
% start conventional frame environment and
% automatically set frame title followed by sequence counter
\begin{frame}%
\frametitle{\lastframetitle~{\normalfont(\arabic{multiframecounter})}}%
}{%
\end{frame}%
}
\makeatother

\makeatletter
\newdimen\tu@tmpa%
\newdimen\ydiffl%
\newdimen\xdiffl%
\newcommand\ydiff[2]{%
    \coordinate (tmpnamea) at (#1);%
    \coordinate (tmpnameb) at (#2);%
    \pgfextracty{\tu@tmpa}{\pgfpointanchor{tmpnamea}{center}}%
    \pgfextracty{\ydiffl}{\pgfpointanchor{tmpnameb}{center}}%
    \advance\ydiffl by -\tu@tmpa%
}
\newcommand\xdiff[2]{%
    \coordinate (tmpnamea) at (#1);%
    \coordinate (tmpnameb) at (#2);%
    \pgfextractx{\tu@tmpa}{\pgfpointanchor{tmpnamea}{center}}%
    \pgfextractx{\xdiffl}{\pgfpointanchor{tmpnameb}{center}}%
    \advance\xdiffl by -\tu@tmpa%
}
\makeatother
\newcommand{\copyrightbox}[3][r]{%
\begin{tikzpicture}%
\node[inner sep=0pt,minimum size=2em](ciimage){#2};
\usefont{OT1}{phv}{n}{n}\fontsize{4}{4}\selectfont
\ydiff{ciimage.south}{ciimage.north}
\xdiff{ciimage.west}{ciimage.east}
\ifthenelse{\equal{#1}{r}}{%
\node[inner sep=0pt,right=1ex of ciimage.south east,anchor=north west,rotate=90]%
{\raggedleft\color{black!50}\parbox{\the\ydiffl}{\raggedright{}#3}};%
}{%
\ifthenelse{\equal{#1}{l}}{%
\node[inner sep=0pt,right=1ex of ciimage.south west,anchor=south west,rotate=90]%
{\raggedleft\color{black!50}\parbox{\the\ydiffl}{\raggedright{}#3}};%
}{%
\node[inner sep=0pt,below=1ex of ciimage.south west,anchor=north west]%
{\raggedleft\color{black!50}\parbox{\the\xdiffl}{\raggedright{}#3}};%
}
}
\end{tikzpicture}
}


%% --------------------

%\usepackage[excludeor]{everyhook}
%\PushPreHook{par}{\setbox0=\lastbox\llap{MUH}}\box0}

%\vspace*{\stretch{1}

%\setbox0=\lastbox \llap{\textbullet\enskip}\box0}

\setlength{\parskip}{\fill}

\newcommand\noskips{\setlength{\parskip}{1ex}}
\newcommand\doskips{\setlength{\parskip}{\fill}}

\newcommand\xx{\par\vspace*{\stretch{1}}\par}
\newcommand\xxs{\par\vspace*{2ex}\par}
\newcommand\tuple[1]{\langle #1 \rangle}
\newcommand\code[1]{{\sf \footnotesize #1}}
\newcommand\ex[1]{\uline{Example:} \ifdefined \presentationonly \pause \fi
  \ifdefined\showexamples#1\xspace\else{\uline{\hspace*{2cm}}}\fi}

\newcommand\ceil[1]{\lceil #1 \rceil}


\AtBeginSection[]
{
   \begin{frame}
       \frametitle{Outline}
       \tableofcontents[currentsection]
   \end{frame}
}



\pgfdeclarelayer{edgelayer}
\pgfdeclarelayer{nodelayer}
\pgfsetlayers{edgelayer,nodelayer,main}

\tikzstyle{none}=[inner sep=0pt]
\tikzstyle{rn}=[circle,fill=Red,draw=Black,line width=0.8 pt]
\tikzstyle{gn}=[circle,fill=Lime,draw=Black,line width=0.8 pt]
\tikzstyle{yn}=[circle,fill=Yellow,draw=Black,line width=0.8 pt]
\tikzstyle{empty}=[circle,fill=White,draw=Black]
\tikzstyle{bw} = [rectangle, draw, fill=blue!20, 
    text width=4em, text centered, rounded corners, minimum height=2em]
    
    \newcommand{\CcNote}[1]{% longname
	This work is licensed under the \textit{Creative Commons #1 3.0 License}.%
}
\newcommand{\CcImageBy}[1]{%
	\includegraphics[scale=#1]{creative_commons/cc_by_30.pdf}%
}
\newcommand{\CcImageSa}[1]{%
	\includegraphics[scale=#1]{creative_commons/cc_sa_30.pdf}%
}
\newcommand{\CcImageNc}[1]{%
	\includegraphics[scale=#1]{creative_commons/cc_nc_30.pdf}%
}
\newcommand{\CcGroupBySa}[2]{% zoom, gap
	\CcImageBy{#1}\hspace*{#2}\CcImageNc{#1}\hspace*{#2}\CcImageSa{#1}%
}
\newcommand{\CcLongnameByNcSa}{Attribution-NonCommercial-ShareAlike}

\newenvironment{changemargin}[1]{% 
  \begin{list}{}{% 
    \setlength{\topsep}{0pt}% 
    \setlength{\leftmargin}{#1}% 
    \setlength{\rightmargin}{1em}
    \setlength{\listparindent}{\parindent}% 
    \setlength{\itemindent}{\parindent}% 
    \setlength{\parsep}{\parskip}% 
  }% 
  \item[]}{\end{list}} 




\title{Lecture 19 --- Tort: Limitation of Liability }

\author{Jeff Zarnett \\ \small \texttt{jzarnett@uwaterloo.ca}}
\institute{Department of Electrical and Computer Engineering \\
  University of Waterloo}
\date{\today}


\begin{document}

\begin{frame}
  \titlepage

\begin{center}
  \small{Acknowledgments: Douglas Harder~\cite{dwh}, Julie Vale~\cite{jv}}
  \end{center}
\end{frame}




\begin{frame}
\frametitle{Is It Over?}

How long can a party be held liable for negligence?

How many years can a party be held accountable for a breach of contract?


This was a matter of common-law precedent over many years, and the various and differing decisions made it difficult to determine when liability ends.

\end{frame}



\begin{frame}
\frametitle{General Principles}

The goal is that after the passage of sufficient time, a person can no longer be held responsible for actions taken in the past. 

After a specified period of time, litigation related to the issue cannot proceed.\\
\quad Because it is prohibited by the statute, it is called \alert{statute barred}.

With that in mind, plaintiffs cannot ``sit on'' the issue.\\
\quad If they have discovered grounds for a claim, they must proceed expeditiously.\\
\quad Otherwise, they lose their right to make the claim.

\end{frame}



\begin{frame}
\frametitle{The 2002 Limitations Act}

In 2002 the Government of Ontario introduced the Limitations Act.

This introduced an ultimate 15-year limitation period for the commencement of any action, which begins to run as soon as the project is completed.

Any action that begins after that point is statute barred...

(Possible exception: if a minor is affected the limitation period may not begin running until that person reaches the age of majority.)

\end{frame}



\begin{frame}
\frametitle{The 2002 Limitations Act}

No action may commence more than two years after the claim is discovered.

There are exceptions, for example: there is no limitation period in respect of an environmental claim that has not been discovered.

\end{frame}



\begin{frame}
\frametitle{Discovery}


A claim is discovered on the earlier of,
\begin{enumerate}[(a)]
	\item the day on which the person with the claim first knew,
		\begin{enumerate}[i.]
			\item that the injury, loss or damage had occurred,
			\item that the injury, loss or damage was caused by or contributed to by an act or omission, and
			\item that the act or omission was that of the person against whom the claim is mad, and
			\item that, having regard to the nature of the injury, loss or damage, a proceeding would be an appropriate means to seek to remedy it;
		\end{enumerate}
	\item the day on which a reasonable person with the abilities and in the circumstances of the person with the claim first ought to have known of the matters referred to in clause (a).
\end{enumerate}

\end{frame}



\begin{frame}
\frametitle{Don't Ask, Don't Get}

There were objections to not allowing business agreements to vary the limitation periods.

In 2006, the Act was amended to allow business agreements to:

\begin{itemize}
	\item Shorten or extend the 2-year limitation period.
	\item Shorten the 15-year ultimate limitation period.
	\item Extend or suspend the 15-year ultimate limitation period, but only if the relevant claim was discovered prior to 15 years.
\end{itemize}
\end{frame}


\begin{frame}
\frametitle{Business Agreement}

What is a business agreement?

A \alert{business agreement} means an agreement made by parties none of whom is a consumer as defined in the Consumer Protection Act, 2002.

Who is a consumer?

A \alert{consumer} means an individual acting for personal, family or household purposes and does not include a person who is acting for business purposes.

\end{frame}



\begin{frame}
\frametitle{Discovery}

In \textit{Kamloops (City) v. Nielsen}, 1984, the Supreme Court of Canada articulated the concept of discovery.

A house was being built by a contractor on loose fill.

City by-laws allowed for discretion in performing inspections.

The foundations were found to be insufficient by a city inspector.

Stop work orders and warnings were issued but not enforced.

The house was sold to the contractor's parents who were made aware of the deficiencies, but who chose not to litigate.

\end{frame}



\begin{frame}
\frametitle{Discovery}

The Nielsen family purchased the house when an inspection failed to find any deficiencies.

Later, the foundations subsided and the Nielsens sued the city and contractor.

The city claimed that there was no duty of care \& the action was statute barred.

Who was correct?

\end{frame}



\begin{frame}
\frametitle{Discovery}

The Nielsen family was vindicated by the court.

The city had a duty of care to the citizens in discharging its duty of regulation of construction; a duty the city took on by passing regulation legislation. 

The city allowed construction to continue in spite of the builder's negligence.

The court said:

\begin{quote}
Sections 3(1)(a) and 6(3) of the Limitations Act postponed the running of time until the acquisition of knowledge or means of knowledge of the facts giving rise to the cause of action. Plaintiff was not barred in the circumstances of this case by the failure of the first purchasers to litigate.
\end{quote}

\end{frame}



\begin{frame}
\frametitle{Concurrent Tortfeasors}

The city and contractor were found to be concurrent tortfeasors.

The contractor was assigned 75\% of the fault; the city 25\%.

Why?

\end{frame}


\begin{frame}
\frametitle{Exceptions}

Previously, the courts have often allowed for extensions beyond the expiration of the limitation period:

(1) A reasonable explanation for the delay.

(2) The potential defendants were aware of the claim before the expiration date.

Such extensions fall under the common law doctrine of special circumstances.

\end{frame}



\begin{frame}
\frametitle{Extensions}

In \textit{York Condominium Corp. No. 382 v. Jay-M Holdings Ltd.}, 2007, the judge indicated that the purpose of the Act is:


\begin{quote}
   to balance the right of claimants to sue with the right of defendants to have some certainty and finality in managing their affairs.
\end{quote}


\end{frame}



\begin{frame}
\frametitle{Extensions Example}

\textit{Joseph v. Paramount Canada 's Wonderland}, 2008

The plaintiff was injured on the grounds on September 5th, 2004, after the introduction of the Limitations Act, 2002.

While documentation was forwarded to the defendant soon thereafter, the lawyer for the Mr. Joseph failed to issue the claim until October 31st, 2006.

The lawyer had left instructions with his assistant to file prior to September 5th.

The assistant was under the impression that the old six-year limitation period was still in effect.

Should the case be allowed?

\end{frame}



\begin{frame}
\frametitle{Extensions Example}

The court ruled no:

\begin{quote}
The question to be answered now is whether the legislature intended to preserve the court 's common law discretion to extend limitation periods under the new Act by applying the doctrine of special circumstances. As a matter of statutory interpretation, I have concluded that the answer must be no.
\end{quote}

The goal of the act is to provide certainty by consolidating limitation periods.

\end{frame}


\begin{frame}
\frametitle{References \& Disclaimer}
\bibliographystyle{alphaurl}
\setbeamertemplate{bibliography item}{\insertbiblabel}
{\scriptsize
\bibliography{290}
}
\vfill

{\tiny Disclaimer: the material presented in these lectures slides is intended for use in the course ECE~290 at the University of Waterloo and should not be relied upon as legal advice. Any reliance on these course slides by any party for any other purpose are the responsibility of such parties.  The author(s) accept(s) no responsibility for damages, if any, suffered by any party as a result of decisions made or actions based on these course slides for any other purpose than that for which it was intended.\par}


\end{frame}



\end{document}

