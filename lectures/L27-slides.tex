
\documentclass[letterpaper,hide notes,xcolor={table,svgnames},pdftex,10pt]{beamer}
\def\showexamples{t}


%\usepackage[svgnames]{xcolor}

%% Demo talk
%\documentclass[letterpaper,notes=show]{beamer}

\usecolortheme{crane}
\setbeamertemplate{navigation symbols}{}

\usetheme{MyPittsburgh}
%\usetheme{Frankfurt}

%\usepackage{tipa}

\usepackage{hyperref}
\usepackage{graphicx,xspace}
\usepackage[normalem]{ulem}
\usepackage{multicol}

\newcommand\SF[1]{$\bigstar$\footnote{SF: #1}}

\usepackage[default]{sourcesanspro}
\usepackage[T1]{fontenc}

\newcounter{tmpnumSlide}
\newcounter{tmpnumNote}

% old question code
%\newcommand\question[1]{{$\bigstar$ \small \onlySlide{2}{#1}}}
% \newcommand\nquestion[1]{\ifdefined \presentationonly \textcircled{?} \fi \note{\par{\Large \textbf{?}} #1}}
% \newcommand\nanswer[1]{\note{\par{\Large \textbf{A}} #1}}


 \newcommand\mnote[1]{%
   \addtocounter{tmpnumSlide}{1}
   \ifdefined\showcues {~\tiny\fbox{\arabic{tmpnumSlide}}}\fi
   \note{\setlength{\parskip}{1ex}\addtocounter{tmpnumNote}{1}\textbf{\Large \arabic{tmpnumNote}:} {#1\par}}}

\newcommand\mmnote[1]{\note{\setlength{\parskip}{1ex}#1\par}}

%\newcommand\mnote[2][]{\ifdefined\handoutwithnotes {~\tiny\fbox{#1}}\fi
% \note{\setlength{\parskip}{1ex}\textbf{\Large #1:} #2\par}}

%\newcommand\mnote[2][]{{\tiny\fbox{#1}} \note{\setlength{\parskip}{1ex}\textbf{\Large #1:} #2\par}}

\newcommand\mquestion[2]{{~\color{red}\fbox{?}}\note{\setlength{\parskip}{1ex}\par{\Large \textbf{?}} #1} \note{\setlength{\parskip}{1ex}\par{\Large \textbf{A}} #2\par}\ifdefined \presentationonly \pause \fi}

\newcommand\blackboard[1]{%
\ifdefined   \showblackboard
  {#1}
  \else {\begin{center} \fbox{\colorbox{blue!30}{%
         \begin{minipage}{.95\linewidth}%
           \hspace{\stretch{1}} Some space intentionally left blank; done at the blackboard.%
         \end{minipage}}}\end{center}}%
         \fi%
}



%\newcommand\q{\tikz \node[thick,color=black,shape=circle]{?};}
%\newcommand\q{\ifdefined \presentationonly \textcircled{?} \fi}

\usepackage{listings}
\lstset{%
  keywordstyle=\bfseries,
  aboveskip=15pt,
  belowskip=15pt,
  captionpos=b,
  identifierstyle=\ttfamily,
  escapeinside={(*@}{@*)},
  stringstyle=\ttfamiliy,
  frame=lines,
  numbers=left, basicstyle=\scriptsize, numberstyle=\tiny, stepnumber=0, numbersep=2pt}

\usepackage{siunitx}
\newcommand\sius[1]{\num[group-separator = {,}]{#1}\si{\micro\second}}
\newcommand\sims[1]{\num[group-separator = {,}]{#1}\si{\milli\second}}
\newcommand\sins[1]{\num[group-separator = {,}]{#1}\si{\nano\second}}
\sisetup{group-separator = {,}, group-digits = true}

%% -------------------- tikz --------------------
\usepackage{tikz}
\usetikzlibrary{positioning}
\usetikzlibrary{arrows,backgrounds,automata,decorations.shapes,decorations.pathmorphing,decorations.markings,decorations.text}

\tikzstyle{place}=[circle,draw=blue!50,fill=blue!20,thick, inner sep=0pt,minimum size=6mm]
\tikzstyle{transition}=[rectangle,draw=black!50,fill=black!20,thick, inner sep=0pt,minimum size=4mm]

\tikzstyle{block}=[rectangle,draw=black, thick, inner sep=5pt]
\tikzstyle{bullet}=[circle,draw=black, fill=black, thin, inner sep=2pt]

\tikzstyle{pre}=[<-,shorten <=1pt,>=stealth',semithick]
\tikzstyle{post}=[->,shorten >=1pt,>=stealth',semithick]
\tikzstyle{bi}=[<->,shorten >=1pt,shorten <=1pt, >=stealth',semithick]

\tikzstyle{mut}=[-,>=stealth',semithick]

\tikzstyle{treereset}=[dashed,->, shorten >=1pt,>=stealth',thin]

\usepackage{ifmtarg}
\usepackage{xifthen}
\makeatletter
% new counter to now which frame it is within the sequence
\newcounter{multiframecounter}
% initialize buffer for previously used frame title
\gdef\lastframetitle{\textit{undefined}}
% new environment for a multi-frame
\newenvironment{multiframe}[1][]{%
\ifthenelse{\isempty{#1}}{%
% if no frame title was set via optional parameter,
% only increase sequence counter by 1
\addtocounter{multiframecounter}{1}%
}{%
% new frame title has been provided, thus
% reset sequence counter to 1 and buffer frame title for later use
\setcounter{multiframecounter}{1}%
\gdef\lastframetitle{#1}%
}%
% start conventional frame environment and
% automatically set frame title followed by sequence counter
\begin{frame}%
\frametitle{\lastframetitle~{\normalfont(\arabic{multiframecounter})}}%
}{%
\end{frame}%
}
\makeatother

\makeatletter
\newdimen\tu@tmpa%
\newdimen\ydiffl%
\newdimen\xdiffl%
\newcommand\ydiff[2]{%
    \coordinate (tmpnamea) at (#1);%
    \coordinate (tmpnameb) at (#2);%
    \pgfextracty{\tu@tmpa}{\pgfpointanchor{tmpnamea}{center}}%
    \pgfextracty{\ydiffl}{\pgfpointanchor{tmpnameb}{center}}%
    \advance\ydiffl by -\tu@tmpa%
}
\newcommand\xdiff[2]{%
    \coordinate (tmpnamea) at (#1);%
    \coordinate (tmpnameb) at (#2);%
    \pgfextractx{\tu@tmpa}{\pgfpointanchor{tmpnamea}{center}}%
    \pgfextractx{\xdiffl}{\pgfpointanchor{tmpnameb}{center}}%
    \advance\xdiffl by -\tu@tmpa%
}
\makeatother
\newcommand{\copyrightbox}[3][r]{%
\begin{tikzpicture}%
\node[inner sep=0pt,minimum size=2em](ciimage){#2};
\usefont{OT1}{phv}{n}{n}\fontsize{4}{4}\selectfont
\ydiff{ciimage.south}{ciimage.north}
\xdiff{ciimage.west}{ciimage.east}
\ifthenelse{\equal{#1}{r}}{%
\node[inner sep=0pt,right=1ex of ciimage.south east,anchor=north west,rotate=90]%
{\raggedleft\color{black!50}\parbox{\the\ydiffl}{\raggedright{}#3}};%
}{%
\ifthenelse{\equal{#1}{l}}{%
\node[inner sep=0pt,right=1ex of ciimage.south west,anchor=south west,rotate=90]%
{\raggedleft\color{black!50}\parbox{\the\ydiffl}{\raggedright{}#3}};%
}{%
\node[inner sep=0pt,below=1ex of ciimage.south west,anchor=north west]%
{\raggedleft\color{black!50}\parbox{\the\xdiffl}{\raggedright{}#3}};%
}
}
\end{tikzpicture}
}


%% --------------------

%\usepackage[excludeor]{everyhook}
%\PushPreHook{par}{\setbox0=\lastbox\llap{MUH}}\box0}

%\vspace*{\stretch{1}

%\setbox0=\lastbox \llap{\textbullet\enskip}\box0}

\setlength{\parskip}{\fill}

\newcommand\noskips{\setlength{\parskip}{1ex}}
\newcommand\doskips{\setlength{\parskip}{\fill}}

\newcommand\xx{\par\vspace*{\stretch{1}}\par}
\newcommand\xxs{\par\vspace*{2ex}\par}
\newcommand\tuple[1]{\langle #1 \rangle}
\newcommand\code[1]{{\sf \footnotesize #1}}
\newcommand\ex[1]{\uline{Example:} \ifdefined \presentationonly \pause \fi
  \ifdefined\showexamples#1\xspace\else{\uline{\hspace*{2cm}}}\fi}

\newcommand\ceil[1]{\lceil #1 \rceil}


\AtBeginSection[]
{
   \begin{frame}
       \frametitle{Outline}
       \tableofcontents[currentsection]
   \end{frame}
}



\pgfdeclarelayer{edgelayer}
\pgfdeclarelayer{nodelayer}
\pgfsetlayers{edgelayer,nodelayer,main}

\tikzstyle{none}=[inner sep=0pt]
\tikzstyle{rn}=[circle,fill=Red,draw=Black,line width=0.8 pt]
\tikzstyle{gn}=[circle,fill=Lime,draw=Black,line width=0.8 pt]
\tikzstyle{yn}=[circle,fill=Yellow,draw=Black,line width=0.8 pt]
\tikzstyle{empty}=[circle,fill=White,draw=Black]
\tikzstyle{bw} = [rectangle, draw, fill=blue!20, 
    text width=4em, text centered, rounded corners, minimum height=2em]
    
    \newcommand{\CcNote}[1]{% longname
	This work is licensed under the \textit{Creative Commons #1 3.0 License}.%
}
\newcommand{\CcImageBy}[1]{%
	\includegraphics[scale=#1]{creative_commons/cc_by_30.pdf}%
}
\newcommand{\CcImageSa}[1]{%
	\includegraphics[scale=#1]{creative_commons/cc_sa_30.pdf}%
}
\newcommand{\CcImageNc}[1]{%
	\includegraphics[scale=#1]{creative_commons/cc_nc_30.pdf}%
}
\newcommand{\CcGroupBySa}[2]{% zoom, gap
	\CcImageBy{#1}\hspace*{#2}\CcImageNc{#1}\hspace*{#2}\CcImageSa{#1}%
}
\newcommand{\CcLongnameByNcSa}{Attribution-NonCommercial-ShareAlike}

\newenvironment{changemargin}[1]{% 
  \begin{list}{}{% 
    \setlength{\topsep}{0pt}% 
    \setlength{\leftmargin}{#1}% 
    \setlength{\rightmargin}{1em}
    \setlength{\listparindent}{\parindent}% 
    \setlength{\itemindent}{\parindent}% 
    \setlength{\parsep}{\parskip}% 
  }% 
  \item[]}{\end{list}} 




\title{Lecture 27 --- Engineering Co-op \& Sexual Harassment }

\author{Jeff Zarnett, based on original by Douglas Harder \\ \small \texttt{jzarnett@uwaterloo.ca} / \texttt{dwharder@uwaterloo.ca}}
\institute{Department of Electrical and Computer Engineering \\
  University of Waterloo}
\date{\today}


\begin{document}

\begin{frame}
  \titlepage

\begin{center}
  \small{Acknowledgments: Douglas Harder~\cite{dwh}, Julie Vale~\cite{jv}}
  \end{center}
\end{frame}

\part{Engineering Co-op}

\begin{frame}
\partpage
\end{frame}




\begin{frame}
\frametitle{But I am but a Lowly Co-op Student}


We will look at the use of the word ``engineer'' with respect to undergraduate students in engineering programs.

\begin{itemize}
	\item The term is restricted
	\item What is a violation?
	\item What can you do?
	\item Student Membership Program
	\item Engineering Intern
\end{itemize}

\end{frame}



\begin{frame}
\frametitle{What is Protected?}

In Canada, the term ``engineer'' is protected.

In the United Kingdom, the term ``chartered engineer'' is protected  -- but not the term ``engineer''.

(Humorously, when I visited the UK in 2016 I put down ``Professional Engineer'' on my landing card and was asked what the difference is...)

In the United States, the term ``engineer'' is protected in many states, but there are more cases of industrial exemptions.

\end{frame}



\begin{frame}
\frametitle{Professional Engineers Act}

\textbf{Offence, use of term ``professional engineer'', etc.}\\
40 (2)(a.1) Every person who is not a holder of a licence or a temporary licence and who uses the title ``engineer'' or an abbreviation of that title in a manner that will lead to the belief that the person may engage in the practice of professional engineering is guilty of an offence and on conviction is liable for the first offence to a fine of not more than \$10,000 and for each subsequent offence to a fine of not more than \$25,000.

\end{frame}



\begin{frame}
\frametitle{Professional Engineers Act}

\textbf{Onus of proof}\\
40 (2.1) In a proceeding for an alleged contravention of clause (2) (a.1), the burden of proving that the use of the title or abbreviation will not lead to the belief referred to is on the defendant, unless the defendant's use of the title or abbreviation is authorized or required by an Act or regulation.

Authorized exceptions include: Flight Engineer, Train Engineer, Sound Engineer, Aircraft Maintenance Engineer...

\end{frame}



\begin{frame}
\frametitle{What Else is Acceptable?}

Suppose your co-op job is outside of Canada in a jurisdiction where the term engineer is not protected?

Example: you had a co-op job in England; your title was ``Software Engineer''.

This is the title you held as an employee and consequently, you may continue to use this title on your resume.

As long as there is no further indication that you can provide services within the practice of software engineering.

Any such listing in a resume must make it clear that the job was in a jurisdiction outside of Canada.

\end{frame}



\begin{frame}
\frametitle{What is a Violation?}

From Engineers Canada's web site:

Until you become licensed, it is against the law in Canada for you:

\begin{enumerate}
\item to approve [that is, seal, sign and date] engineering drawings or reports,
\item to use the title ``engineer'' or ``professional engineer'' (or any title like it), or
\item to offer any engineering services to the public.
\end{enumerate}

	You may do most other technical work legally, subject to other professional laws (such as the architecture or land surveyors acts).


\end{frame}



\begin{frame}
\frametitle{Don't Do It}

You applied for a co-op job in Canada and your title is ``software engineer''?\\
\quad You are in violation of the Professional Engineers Act

Your resume or LinkedIn states that you were employed in Canada with a title that contains the term engineer?\\
\quad You are in violation of the Professional Engineers Act

\end{frame}



\begin{frame}
\frametitle{Pay Up}

Has PEO ever applied for an order against an individual who called himself an engineer on the job?

In the January/February 2003 edition of the Gazette, the Association gained a court order against Dan Stolarchuk.

He among other things, described himself as a ``Field Applications Engineer'' and a ``R.D. Engineer'' in a resume.

He was ordered to pay PEO its costs at \$6,750

\end{frame}



\begin{frame}
\frametitle{No CoA = You Will Pay}

Has PEO ever found a business holding a Certificate of Authorization guilty of professional misconduct in relation to this?

In the March/April 2008 edition of the Gazette, the Association found The Environment Management Group Ltd. guilty of professional misconduct

Among other things, they issued business cards for non-engineering staff, upon which was `Member of Professional Engineers Ontario.''

For this and other actions, EMG was required to pay \$2,500 and was given a reprimand, which was recorded for twelve months.

\end{frame}



\begin{frame}
\frametitle{What Can You Do?}

First, approach your employer, and if nothing else, direct them to these slides.\\
\quad Or to the author... 

If your employer has any questions, they can contact PEO directly.

If your employer changes your job title, you must direct them to change your job title on JobMine.

\end{frame}



\begin{frame}
\frametitle{``What is the sound of a hair being split?''}

The term ``engineering'' is not protected in Canada -- consequently, you may be able to use it instead of ``engineer''.

If you are using the term ``engineering'', you cannot, however, use any of the words ``professional'', ``consultant'' or ``specialist''.

These are highly restricted terms specified in the Professional Engineers Act.

Any suggestion that you are, for example, a consulting engineer, will command the attention of PEO.


\end{frame}



\begin{frame}
\frametitle{CTRL+F, CTRL+R}

Other terms or phrases that may replace the term ``engineer'' in a job title include:


\begin{tabular}{l l l}
Developer & Engineering Developer  & Manager\\
Programmer & Engineering Programmer & Team Lead\\
Analyst & Engineering Analyst & Leader\\
Operator & Engineering Operator & Coordinator\\
Specialist & Consultant & \\
\end{tabular}

\end{frame}



\begin{frame}
\frametitle{Now? Now is Too Late...}

What if the job is over -- what about my resume?

It would not be considered misrepresentation if you change an illegal job title that appears on a resume.

The change must, however, be minimal in effect.

Suggestions:
\begin{itemize}
\item Replace ``engineer'' with ``engineering'' if possible
\item Otherwise, replace ``engineer'' with ``developer'', ``analyst'', ``operator'' or ``programmer''
\item Do not use a term such as ``specialist'' or ``consultant'' -- they suggest additional responsibilities
\end{itemize}

\end{frame}



\begin{frame}
\frametitle{Student Terms}

What can a student call him or herself?

The terms ``computer engineering student'' and ``candidate for computer engineering'' are acceptable.

The terms ``student engineer'' or ``engineer candidate'' or anything with the phrase ``computer engineer'' is not acceptable.

You can also join PEO's Student Membership Program

\url{http://www.engineeringstudents.peo.on.ca/}

\end{frame}



\begin{frame}
\frametitle{SMP is not Symmetric Multiprocessing}

As of January 2013: We're \#1! About 3829 UW students enrolled.\\
\quad About double the next highest university (McMaster).

\end{frame}




\begin{frame}
\frametitle{Intern / EIT}


Once you graduate, you can apply for the position of engineering intern and use the initials EIT after your name.

(It used to be ``Engineer In Training'', hence the acronym EIT).

You can also vote in regional chapter elections, attend the annual meeting of the Association, or be appointed to a committee.

\end{frame}

\part{Sexual Harassment}

\begin{frame}
\partpage
\end{frame}



\begin{frame}
\frametitle{Harassment}

Recall that it is professional misconduct to harass another person while engaged in the practice of professional engineering:

	72. (2)  For the purposes of the Act and this Regulation,
``professional misconduct'' means\\
(n)	harassment; that is, engaging in a course of vexatious 	comment or conduct that is known or ought reasonably to be 	known as unwelcome and that might reasonably be regarded 	as interfering in a professional engineering relationship

\end{frame}



\begin{frame}
\frametitle{Not Acceptable}

More generally, sexual harassment is not acceptable at any time:

	From the Canada Labour Code:

\textbf{Definition of ``sexual harassment''}\\
	247.1 In this Division, ``sexual harassment'' means any conduct, comment, gesture or contact of a sexual nature
\begin{enumerate}[(a)]
\item that is likely to cause offence or humiliation to any employee; or
\item that might, on reasonable grounds, be perceived by that employee as placing a condition of a sexual nature on employment or on any opportunity for training or promotion.
\end{enumerate}

The fight against sexual harassment is about respect for others � a fundamental value in Canadian society.


\end{frame}



\begin{frame}
\frametitle{At Waterloo}

What is sexual harassment at Waterloo?


Unwanted attention of a sexual or gender related nature (verbal, non-verbal, physical)  --  jokes, touching, suggestive remarks, leering, or demands for sexual favours, threats...

\url{http://uwaterloo.ca/conflict-management-human-rights/sexual-harassment}

\end{frame}



\begin{frame}
\frametitle{Examples at Waterloo}

\begin{itemize}
\item Even though you said ``no'', another resident continues to ask you out and follow you around.
\item Your workterm supervisor continually brushes up against you.
\item You said no to a date, and now the person in charge of your course will not answer your course related questions.
\item You continually receive x-rated e-mail messages, and now you are reluctant to log on to your account.
\item The students in residence (or class, or club, etc.) tease and hassle you about being LGBTQ.
\end{itemize}


\end{frame}



\begin{frame}
\frametitle{UW Policy 33}

\textbf{Sexual Harassment} includes comment or conduct where acceptance of sexual advances is a condition of education or employment, or where rejection of sexual advances negatively impacts decisions that concern the recipient (e.g., grades, performance evaluation or any academic or employment decisions) or where unwelcome sexual advances, comment, conduct or communications interfere with the recipient's work or study.

\end{frame}



\begin{frame}
\frametitle{Ontario Human Rights Code}

Under the Ontario Human Rights Code, sexual harassment is ``engaging in a course of vexatious comment or conduct that is known or ought to be known to be unwelcome.'' 

\end{frame}



\begin{frame}
\frametitle{Human Rights Commission Listing}

\begin{itemize}
\item Asking for sex in exchange for a benefit or a favour
\item Repeatedly asking for dates, and not taking ``no'' for an answer
\item Demanding hugs
\item Making unnecessary physical contact, including unwanted touching
\item Using rude or insulting language or making comments toward women (or men, depending on the circumstances)
\item Calling people sex-specific derogatory names
\item Making sex-related comments about a person's physical characteristics or actions
\item Saying or doing something because you think a person does not conform to sex-role stereotypes
\item Posting or sharing pornography, sexual pictures or cartoons, sexually explicit graffiti, or other sexual images (including online)
\item Making sexual jokes
\item Bragging about sexual prowess
\end{itemize}

\end{frame}



\begin{frame}
\frametitle{Awkward}

Suppose you are asked out and you respond in the negative  --  how do you respond to that person in the future in the work place?

The issue is one of interest versus infatuation.

A person interested in someone else will ask them out, and on receiving an answer of ``no'' will look elsewhere.

One who becomes infatuated in another is less likely to accept ``no''  --  they have spent too much time thinking about the person in the first place.

Some people will have had previous approaches by such a character, in which case, future approaches may equally uncomfortable  --  please remember this...


\end{frame}



\begin{frame}
\frametitle{Workplace Romance}

Consequently:
Can you ask someone out?\\
\quad Yes, but use judgement.

If the person is says ``no'', can you ask again?\\
\quad No  --  the ball is in their court.

What if he/she is playing ``hard to get''?\\
\quad It's their issue now  --  life is not a Hollywood movie.

Do you ask subordinates or someone you are supervising out?\\
\quad No  --  you are in a position of power.

In all cases, evaluate the risks. Is it really worth it?

If you do date someone at work:
\begin{itemize}
	\item Do not expect any form of favouritism from your partner
	\item Do not bring work home
\end{itemize}

\end{frame}



\begin{frame}
\frametitle{Bonus Disclaimer}

I'm neither a dating coach nor an expert in relationships.

I therefore disclaim all responsibility for any reliance on this advice...


\end{frame}


\begin{frame}
\frametitle{References \& Disclaimer}
\bibliographystyle{alphaurl}
\setbeamertemplate{bibliography item}{\insertbiblabel}
{\scriptsize
\bibliography{290}
}
\vfill

{\tiny Disclaimer: the material presented in these lectures slides is intended for use in the course ECE~290 at the University of Waterloo and should not be relied upon as legal advice. Any reliance on these course slides by any party for any other purpose are the responsibility of such parties.  The author(s) accept(s) no responsibility for damages, if any, suffered by any party as a result of decisions made or actions based on these course slides for any other purpose than that for which it was intended.\par}


\end{frame}


\end{document}

