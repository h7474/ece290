
\documentclass[letterpaper,hide notes,xcolor={table,svgnames},pdftex,10pt]{beamer}
\def\showexamples{t}


%\usepackage[svgnames]{xcolor}

%% Demo talk
%\documentclass[letterpaper,notes=show]{beamer}

\usecolortheme{crane}
\setbeamertemplate{navigation symbols}{}

\usetheme{MyPittsburgh}
%\usetheme{Frankfurt}

%\usepackage{tipa}

\usepackage{hyperref}
\usepackage{graphicx,xspace}
\usepackage[normalem]{ulem}
\usepackage{multicol}

\newcommand\SF[1]{$\bigstar$\footnote{SF: #1}}

\usepackage[default]{sourcesanspro}
\usepackage[T1]{fontenc}

\newcounter{tmpnumSlide}
\newcounter{tmpnumNote}

% old question code
%\newcommand\question[1]{{$\bigstar$ \small \onlySlide{2}{#1}}}
% \newcommand\nquestion[1]{\ifdefined \presentationonly \textcircled{?} \fi \note{\par{\Large \textbf{?}} #1}}
% \newcommand\nanswer[1]{\note{\par{\Large \textbf{A}} #1}}


 \newcommand\mnote[1]{%
   \addtocounter{tmpnumSlide}{1}
   \ifdefined\showcues {~\tiny\fbox{\arabic{tmpnumSlide}}}\fi
   \note{\setlength{\parskip}{1ex}\addtocounter{tmpnumNote}{1}\textbf{\Large \arabic{tmpnumNote}:} {#1\par}}}

\newcommand\mmnote[1]{\note{\setlength{\parskip}{1ex}#1\par}}

%\newcommand\mnote[2][]{\ifdefined\handoutwithnotes {~\tiny\fbox{#1}}\fi
% \note{\setlength{\parskip}{1ex}\textbf{\Large #1:} #2\par}}

%\newcommand\mnote[2][]{{\tiny\fbox{#1}} \note{\setlength{\parskip}{1ex}\textbf{\Large #1:} #2\par}}

\newcommand\mquestion[2]{{~\color{red}\fbox{?}}\note{\setlength{\parskip}{1ex}\par{\Large \textbf{?}} #1} \note{\setlength{\parskip}{1ex}\par{\Large \textbf{A}} #2\par}\ifdefined \presentationonly \pause \fi}

\newcommand\blackboard[1]{%
\ifdefined   \showblackboard
  {#1}
  \else {\begin{center} \fbox{\colorbox{blue!30}{%
         \begin{minipage}{.95\linewidth}%
           \hspace{\stretch{1}} Some space intentionally left blank; done at the blackboard.%
         \end{minipage}}}\end{center}}%
         \fi%
}



%\newcommand\q{\tikz \node[thick,color=black,shape=circle]{?};}
%\newcommand\q{\ifdefined \presentationonly \textcircled{?} \fi}

\usepackage{listings}
\lstset{%
  keywordstyle=\bfseries,
  aboveskip=15pt,
  belowskip=15pt,
  captionpos=b,
  identifierstyle=\ttfamily,
  escapeinside={(*@}{@*)},
  stringstyle=\ttfamiliy,
  frame=lines,
  numbers=left, basicstyle=\scriptsize, numberstyle=\tiny, stepnumber=0, numbersep=2pt}

\usepackage{siunitx}
\newcommand\sius[1]{\num[group-separator = {,}]{#1}\si{\micro\second}}
\newcommand\sims[1]{\num[group-separator = {,}]{#1}\si{\milli\second}}
\newcommand\sins[1]{\num[group-separator = {,}]{#1}\si{\nano\second}}
\sisetup{group-separator = {,}, group-digits = true}

%% -------------------- tikz --------------------
\usepackage{tikz}
\usetikzlibrary{positioning}
\usetikzlibrary{arrows,backgrounds,automata,decorations.shapes,decorations.pathmorphing,decorations.markings,decorations.text}

\tikzstyle{place}=[circle,draw=blue!50,fill=blue!20,thick, inner sep=0pt,minimum size=6mm]
\tikzstyle{transition}=[rectangle,draw=black!50,fill=black!20,thick, inner sep=0pt,minimum size=4mm]

\tikzstyle{block}=[rectangle,draw=black, thick, inner sep=5pt]
\tikzstyle{bullet}=[circle,draw=black, fill=black, thin, inner sep=2pt]

\tikzstyle{pre}=[<-,shorten <=1pt,>=stealth',semithick]
\tikzstyle{post}=[->,shorten >=1pt,>=stealth',semithick]
\tikzstyle{bi}=[<->,shorten >=1pt,shorten <=1pt, >=stealth',semithick]

\tikzstyle{mut}=[-,>=stealth',semithick]

\tikzstyle{treereset}=[dashed,->, shorten >=1pt,>=stealth',thin]

\usepackage{ifmtarg}
\usepackage{xifthen}
\makeatletter
% new counter to now which frame it is within the sequence
\newcounter{multiframecounter}
% initialize buffer for previously used frame title
\gdef\lastframetitle{\textit{undefined}}
% new environment for a multi-frame
\newenvironment{multiframe}[1][]{%
\ifthenelse{\isempty{#1}}{%
% if no frame title was set via optional parameter,
% only increase sequence counter by 1
\addtocounter{multiframecounter}{1}%
}{%
% new frame title has been provided, thus
% reset sequence counter to 1 and buffer frame title for later use
\setcounter{multiframecounter}{1}%
\gdef\lastframetitle{#1}%
}%
% start conventional frame environment and
% automatically set frame title followed by sequence counter
\begin{frame}%
\frametitle{\lastframetitle~{\normalfont(\arabic{multiframecounter})}}%
}{%
\end{frame}%
}
\makeatother

\makeatletter
\newdimen\tu@tmpa%
\newdimen\ydiffl%
\newdimen\xdiffl%
\newcommand\ydiff[2]{%
    \coordinate (tmpnamea) at (#1);%
    \coordinate (tmpnameb) at (#2);%
    \pgfextracty{\tu@tmpa}{\pgfpointanchor{tmpnamea}{center}}%
    \pgfextracty{\ydiffl}{\pgfpointanchor{tmpnameb}{center}}%
    \advance\ydiffl by -\tu@tmpa%
}
\newcommand\xdiff[2]{%
    \coordinate (tmpnamea) at (#1);%
    \coordinate (tmpnameb) at (#2);%
    \pgfextractx{\tu@tmpa}{\pgfpointanchor{tmpnamea}{center}}%
    \pgfextractx{\xdiffl}{\pgfpointanchor{tmpnameb}{center}}%
    \advance\xdiffl by -\tu@tmpa%
}
\makeatother
\newcommand{\copyrightbox}[3][r]{%
\begin{tikzpicture}%
\node[inner sep=0pt,minimum size=2em](ciimage){#2};
\usefont{OT1}{phv}{n}{n}\fontsize{4}{4}\selectfont
\ydiff{ciimage.south}{ciimage.north}
\xdiff{ciimage.west}{ciimage.east}
\ifthenelse{\equal{#1}{r}}{%
\node[inner sep=0pt,right=1ex of ciimage.south east,anchor=north west,rotate=90]%
{\raggedleft\color{black!50}\parbox{\the\ydiffl}{\raggedright{}#3}};%
}{%
\ifthenelse{\equal{#1}{l}}{%
\node[inner sep=0pt,right=1ex of ciimage.south west,anchor=south west,rotate=90]%
{\raggedleft\color{black!50}\parbox{\the\ydiffl}{\raggedright{}#3}};%
}{%
\node[inner sep=0pt,below=1ex of ciimage.south west,anchor=north west]%
{\raggedleft\color{black!50}\parbox{\the\xdiffl}{\raggedright{}#3}};%
}
}
\end{tikzpicture}
}


%% --------------------

%\usepackage[excludeor]{everyhook}
%\PushPreHook{par}{\setbox0=\lastbox\llap{MUH}}\box0}

%\vspace*{\stretch{1}

%\setbox0=\lastbox \llap{\textbullet\enskip}\box0}

\setlength{\parskip}{\fill}

\newcommand\noskips{\setlength{\parskip}{1ex}}
\newcommand\doskips{\setlength{\parskip}{\fill}}

\newcommand\xx{\par\vspace*{\stretch{1}}\par}
\newcommand\xxs{\par\vspace*{2ex}\par}
\newcommand\tuple[1]{\langle #1 \rangle}
\newcommand\code[1]{{\sf \footnotesize #1}}
\newcommand\ex[1]{\uline{Example:} \ifdefined \presentationonly \pause \fi
  \ifdefined\showexamples#1\xspace\else{\uline{\hspace*{2cm}}}\fi}

\newcommand\ceil[1]{\lceil #1 \rceil}


\AtBeginSection[]
{
   \begin{frame}
       \frametitle{Outline}
       \tableofcontents[currentsection]
   \end{frame}
}



\pgfdeclarelayer{edgelayer}
\pgfdeclarelayer{nodelayer}
\pgfsetlayers{edgelayer,nodelayer,main}

\tikzstyle{none}=[inner sep=0pt]
\tikzstyle{rn}=[circle,fill=Red,draw=Black,line width=0.8 pt]
\tikzstyle{gn}=[circle,fill=Lime,draw=Black,line width=0.8 pt]
\tikzstyle{yn}=[circle,fill=Yellow,draw=Black,line width=0.8 pt]
\tikzstyle{empty}=[circle,fill=White,draw=Black]
\tikzstyle{bw} = [rectangle, draw, fill=blue!20, 
    text width=4em, text centered, rounded corners, minimum height=2em]
    
    \newcommand{\CcNote}[1]{% longname
	This work is licensed under the \textit{Creative Commons #1 3.0 License}.%
}
\newcommand{\CcImageBy}[1]{%
	\includegraphics[scale=#1]{creative_commons/cc_by_30.pdf}%
}
\newcommand{\CcImageSa}[1]{%
	\includegraphics[scale=#1]{creative_commons/cc_sa_30.pdf}%
}
\newcommand{\CcImageNc}[1]{%
	\includegraphics[scale=#1]{creative_commons/cc_nc_30.pdf}%
}
\newcommand{\CcGroupBySa}[2]{% zoom, gap
	\CcImageBy{#1}\hspace*{#2}\CcImageNc{#1}\hspace*{#2}\CcImageSa{#1}%
}
\newcommand{\CcLongnameByNcSa}{Attribution-NonCommercial-ShareAlike}

\newenvironment{changemargin}[1]{% 
  \begin{list}{}{% 
    \setlength{\topsep}{0pt}% 
    \setlength{\leftmargin}{#1}% 
    \setlength{\rightmargin}{1em}
    \setlength{\listparindent}{\parindent}% 
    \setlength{\itemindent}{\parindent}% 
    \setlength{\parsep}{\parskip}% 
  }% 
  \item[]}{\end{list}} 




\title{Lecture 10 --- Discharging Contracts }

\author{Jeff Zarnett \\ \small \texttt{jzarnett@uwaterloo.ca}}
\institute{Department of Electrical and Computer Engineering \\
  University of Waterloo}
\date{\today}


\begin{document}

\begin{frame}
  \titlepage

\begin{center}
  \small{Acknowledgments: Douglas Harder~\cite{dwh}, Julie Vale~\cite{jv}}
  \end{center}
\end{frame}




\begin{frame}
\frametitle{All Good Things...}

Contracts eventually come to an end, but they can do so in one of several ways.

This is not the same as completion.

The official term is \alert{discharge}.

Black's Law Dictionary says: ``to cancel or unloose the obligation of a contract; to make an agreement or contract null and inoperative''.

\end{frame}


\begin{frame}
\frametitle{Discharge}

Ways in which a contract may be discharged:

\begin{itemize}
	\item Performance
	\item Agreement
	\item Pursuant to expressed terms
	\item Replacement (including Novation)
	\item Operation of law
	\item Impossibility, impracticality, frustration
	\item Breach of Contract
\end{itemize}

We will examine each of these in turn.

\end{frame}



\begin{frame}
\frametitle{Performance}

Performance is what everyone hopes for when the contract is written up.\\
\quad The parties have performed their respective obligations satisfactorily.

This requires that both parties have completed their parts; it is not enough that one party fulfills its obligations.

As long as any obligation remains unfulfilled, the contract is still in force.

\end{frame}



\begin{frame}
\frametitle{Warranty}

Contracts may have additional terms that go beyond the scope of performance, especially when related to warranties~\cite{lpe}.

The initial performance might be services and materials, but warranties go beyond that.

Construction and supply contracts will often include warranties whereby a party will remedy certain defects under specified limitations.

\end{frame}



\begin{frame}
\frametitle{Tender of Performance}

Occasionally, one party attempts to perform but the other party refuses to a accept the performance~\cite{lba}.

This is called \alert{tender of performance}, whether or not it as accepted.

If the seller tenders delivery of goods, but the buyer does not accept them, the seller is not obligated to attempt delivery again and may sue for breach.

It is enough that the seller has honestly and fully made the effort to fulfill its end of the bargain.

\end{frame}


\begin{frame}
\frametitle{Agreement to Discharge}

Remember that parties are free to amend the agreement if they so choose.

Thus, they can also agree to terminate the contract if they so desire.

This agreement is itself a contract, as it modifies the obligations of the parties.

It therefore needs to have all the elements of a valid contract...\\
\quad (And none of the elements, like duress, that make it invalid!)

In the agreement to discharge, what counts as consideration?

\end{frame}



\begin{frame}
\frametitle{Agreement to Discharge}

Remember that consideration is the benefit each party receives for the contract.

If both parties have unfulfilled obligations, the consideration they receive is relief from the obligations they would otherwise have had to perform.

What if one party has done all of its part but the other has not?

\end{frame}



\begin{frame}
\frametitle{Agreement to Discharge}

If one party has completed all its obligations but the other has not, then alternative consideration is required.

This could be in any form that normal consideration takes (e.g., monetary).

Remember that use of a seal is a valid form of consideration.

Without a valid agreement to discharge, failure to perform is technically a breach of the contract.


\end{frame}



\begin{frame}
\frametitle{Expressed Terms}

A contract may contain terms that allow the contract to be discharged.

It is advisable to include such provisions that allow one or all parties to terminate the contract given certain events.

A contract may be terminated if one party becomes bankrupt, for example.

An engineer may have provisions to terminate a contract if the contractor fails to substantially comply with requirements of the contract~\cite{lpe}.


\end{frame}



\begin{frame}
\frametitle{Material Alteration of Terms}

As we know, a contract may be modified when the parties agree to it.

If a modification is big enough, it is a ``material alteration of the terms''~\cite{lba}.

When this bar is cleared, it amounts to the discharge of the old contract and its replacement with a new one.

A minor alteration would not have this sweeping effect.

What are some examples of changes that would result in novation?

What are some examples of changes that will not?

\end{frame}



\begin{frame}
\frametitle{Accord and Satisfaction}

Sometimes a promisor finds he cannot perform an obligation in the contract.

He may offer the promisee a monetary payment or some other substitute instead, if the promisee will agree to this.

For example, the seller cannot obtain the promised goods, but may offer instead a similar, but different product at a lower price~\cite{lba}.

If the buyer agrees, that will be fine; this is a compromise out of court.

The updated agreement replace the original.

\end{frame}

\begin{frame}
\frametitle{Novation}

Novation is similar, in that it is another form of replacement.

This occurs when one of the parties in the contract is exchanged for another.

Suppose one business purchases another. The purchaser will take on the assets and liabilities of the purchased company.

If creditors accept the new owner as their debtor the liability of the former owner is discharged and the liability of the new owner is established~\cite{lba}.

Creditors may do so either by explicit agreement or by asking for or receiving payment from the new owner.

\end{frame}



\begin{frame}
\frametitle{Replacement}

However the replacement of the contract talks please, it generally requires agreement of all the parties.

The new contract enters into effect and the old is discharged.

The new contract requires consideration, which is usually the discharge of the old contract (but if not, alternate consideration is permitted).

\end{frame}



\begin{frame}
\frametitle{Operation of Law}

There are certain conditions under which a contract will legally be discharged automatically.

\begin{itemize}
	\item By the death of one of the parties (in a personal service contract)
	\item Bankruptcy (in some circumstances)
	\item By the rights and liabilities becoming owed by the same person (e.g., a company purchases a supplier)
\end{itemize}

\end{frame}



\begin{frame}
\frametitle{Other Forms of Discharge}

Sometimes ``stuff'' happens, and circumstances radically change.

English common law originally held parties responsible for foreseeing unusual and unlikely circumstances and writing them into a contract~\cite{lba}.

The law now excuses people from fulfilling their obligations if circumstances change and it is, informally, ``not their fault''.

For this, there are closely related but distinct forms of discharge:

\begin{enumerate}
	\item Discharge by impossibility
	\item Discharge by impracticality
	\item Discharge by frustration
\end{enumerate}

\end{frame}



\begin{frame}
\frametitle{Discharge by Impossibility}

Impossibility requires that it be objectively impossible to satisfy the obligations.

It is not enough for it to be difficult or costly, but must be literally impossible.

Example: \textit{Taylor v Caldwell}, 1863.

Taylor rented a music hall; it burned down 1 week before the performance.

The plaintiff claimed breach of contract: the defendant did not rent the hall.

When a party to the contract dies, the obligations are not passed on; this principle was extended here: it is impossible to rent a hall that burned down.

\end{frame}



\begin{frame}
\frametitle{Discharge by Impracticality}

Impracticality is a subject discharge when it becomes unfeasibly difficult or expensive to perform an obligation.

This is very vague. Courts like to establish tests for these scenarios.

The US test is:

\begin{enumerate}
	\item There must be an occurrence of a condition, the non-occurrence of which was a basic assumption of the contract.
	\item The occurrence must make performance extremely expensive or difficult.
	\item This difficulty was not anticipated by the parties.
\end{enumerate}

\end{frame}



\begin{frame}
\frametitle{Discharge by Impracticality}

Subjective tests like that are difficult to apply.\\
\quad They require the judgement of the court, and may result in appeals...

So-called ``Hell or High Water'' clauses can be added to prevent discharge by impossibility or impracticality.

These may appear, for example, in leasing agreements.\\
\quad The tenant must continue to pay regardless of difficulties encountered.

Whether these clauses will be upheld by the courts is a matter of debate...

\end{frame}



\begin{frame}
\frametitle{Discharge by Frustration}

When the purpose of one party entering into the contract is eliminated, the contract may be discharged by frustration.


\begin{quote}
   Where, after a contract is made, a party's principal purpose is substantially frustrated without his fault by the occurrence of an event the non-occurrence of which was a basic assumption on which the contract was made, his remaining duties to render performance are discharged, unless the language or circumstances [of the contract] indicate the contrary.
\end{quote}

All parties are left as is at the time of discharge -- no attempt is made to put the parties back to their original state.


\end{frame}



\begin{frame}
\frametitle{Discharge by Frustration}

Frustration is seldom used. It is for exceptional, rare circumstances.

The doctrine of frustration was developed in \textit{Chandler v. Webster} where the
defendant hired a room to watch the coronation of King Edward VII.

The event was subsequently cancelled due to appendicitis.

\end{frame}



\begin{frame}
\frametitle{Discharge by Frustration}

Sometimes, performance remains possible, but circumstances mean that the contract as written makes no sense.

In \textit{Metropolitan Water Board v. Dick, Kerr and Co. Ltd.}, 1917, a contract entered into in July of 1914 was halted in 1916 by an order of the Ministry of Munitions.

(The ministry had this authority because of wartime laws.)

After the war ended, the water board attempted to insist that the contract be carried out.

Prices and conditions of supply were different from what they had been in 1916.

The court found that these significant changes rendered the contract frustrated.

\end{frame}



\begin{frame}
\frametitle{Frustration vs Common Mistake}

In the discussion about common mistake, we have declared that if an object of the contract does not exist when the contract is formed, it is a common mistake.

Frustration requires that the contract become impossible or purposeless \textit{after the agreement was made}~\cite{lba}.

\end{frame}



\begin{frame}
\frametitle{``I will make it (il)legal.''}

Another way a contract can become frustrated is a change in the law that renders the obligations impossible.

Case: \textit{Denny, Mott \& Dickson Ltd v James B Fraser \& Co Ltd.}, 1944.

A contract signed prior to 1939 was declared discharged by frustration as the Control of Timber (No 4) Order 1939 prevented further transactions.

\end{frame}



\begin{frame}
\frametitle{Self-Induced Frustration}

It should be obvious that a party cannot wilfully disable itself from performing, and then claim successfully that the contract is frustrated.

This is \alert{self-induced frustration}~\cite{lba}.

Suppose A has contracted with B to transport soil in a remote northern area.\\
\quad Let's assume A has only one truck in the area. 

Let's look at three scenarios from~\cite{lba}.

\end{frame}



\begin{frame}
\frametitle{Three Scenarios}

1: A discovers he's made a bad deal. He sells the dump truck and cannot fulfill the contract.

2. After the deal is made, A's truck is stolen and wrecked.

3. A's truck breaks down because A has not done adequate maintenance on it.

Are these cases where the contract is frustrated?

In all three cases, A cannot fulfill the contract.

\end{frame}



\begin{frame}
\frametitle{Three Scenarios}

1: Very straight forward: this is self-induced frustration and the court will not relieve A from his conduct by declaring the contract frustrated.

2: This situation is no fault of A's and the contract is discharged.

3: The breakdown is A's fault\footnote{The idea of fault in the breakdown is a part of the section on tort law, which is coming soon...} and his failure to deliver the soil will be considered a breach of contract.

\end{frame}



\begin{frame}
\frametitle{``Acts of God''}

Any natural event that makes performance impractical or impossible is said to be an ``Act of God''.

``Natural'' is defined as anything that is beyond human control where no one is responsible.

Examples include earthquakes, volcanic eruptions, and other natural disasters.


There must be no human action involved, so an earthquake caused by mining or drilling would not fall under this provision.

\end{frame}



\begin{frame}
\frametitle{\textit{Force Majeure}}

\alert{Force majeure} is latin for ``superior force'' and it refers to extraordinary events or circumstances beyond anyone's control.

Examples include war, labour strike, riot, crime (or perhaps even an Act of God).

Common law does not recognize \textit{force majeure} on its own, so it must be explicitly written into the contract. 

\end{frame}



\begin{frame}
\frametitle{Discharge Examples}

\textit{Davis Contractors Ltd. v Fareham Urban District Council}

In this 1956 case, the contract was to build 78 houses in an 8-month period.

Due to a labour shortage, the project ended up taking 22 months.

Was this contract discharged?

\end{frame}



\begin{frame}
\frametitle{Discharge Examples}

No - the contract was not frustrated or rendered impossible; it was just more difficult than was expected.


\textit{Swanson Construction Co. Ltd. v Government of Manitoba}, 1963

A contractor ended up completing work in winter conditions instead of the planned summer conditions because the site was not ready on time.

You may have guessed that this contract was not discharged either.

The contractor should've taken this possibility into account in preparing his bid.

\end{frame}



\begin{frame}
\frametitle{Discharge by Breach}

Apart from discharging a contract through performance, the next most common means that contracts are discharged is through a breach of contract.

One or more parties fail(s) to perform at least one obligation under the terms of the contract.

Breach of contract is complicated and will therefore receive its own topic.

\end{frame}


\begin{frame}
\frametitle{References \& Disclaimer}
\bibliographystyle{alphaurl}
\setbeamertemplate{bibliography item}{\insertbiblabel}
{\scriptsize
\bibliography{290}
}
\vfill

{\tiny Disclaimer: the material presented in these lectures slides is intended for use in the course ECE~290 at the University of Waterloo and should not be relied upon as legal advice. Any reliance on these course slides by any party for any other purpose are the responsibility of such parties.  The author(s) accept(s) no responsibility for damages, if any, suffered by any party as a result of decisions made or actions based on these course slides for any other purpose than that for which it was intended.\par}


\end{frame}


\end{document}

